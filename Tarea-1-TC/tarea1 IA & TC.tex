\documentclass{article}
\usepackage{standalone}
\usepackage{import}
\usepackage[margin=1in,headheight=57pt,headsep=0.1in]{geometry}
%\usepackage{xcolor}
\usepackage{tikz}
\usetikzlibrary{automata, positioning, arrows}


%\usepackage[framemethod=TikZ]{mdframed}
\usepackage{fancyhdr}
\usepackage{hw-shortcuts}
\usepackage[spanish]{babel}
% ------------- %
% HEADER/FOOTER %
% ------------- %
\setlength\parindent{0pt}
\setlength\headheight{30pt}
\headsep=0.25in
\lhead{\textbf{\coursetitle{}}}
\rhead{\textbf{\course}}


% ----------------- %
% BOXED EXPRESSIONS %
% ----------------- %
\newenvironment{prop}[1][]{
\ifstrempty{#1}
    {} % Don't add header
    {\mdfsetup{
        frametitle={
            \tikz[baseline=(current bounding box.east),outer sep=0pt]
            \node[anchor=east,rectangle,fill=blue!30]
            {Proposición #1};}}
    } % Header
    \mdfsetup{
        innertopmargin=10pt,linecolor=blue!30,
        linewidth=2pt,topline=true,
        frametitleaboveskip=\dimexpr-\ht\strutbox\relax
    }
    \begin{mdframed}
}{
    \end{mdframed}
}

\newenvironment{question}[1][]{
\ifstrempty{#1}
    {}
    {\mdfsetup{
        frametitle={
            \tikz[baseline=(current bounding box.east),outer sep=0pt]
            \node[anchor=east,rectangle,fill=red!30]
            {Problema #1};}}
    }
    \mdfsetup{
        innertopmargin=10pt,linecolor=red!30,
        linewidth=2pt,topline=true,
        frametitleaboveskip=\dimexpr - \ht\strutbox\relax
    }
    \begin{mdframed}
}{
    \end{mdframed}
}

\newenvironment{lem}[1][]{
\ifstrempty{#1}
    {}
    {\mdfsetup{
        frametitle={
            \tikz[baseline=(current bounding box.east),outer sep=0pt]
            \node[anchor=east,rectangle,fill=green!30]
            {Lema #1};}}
    }\newcommand{\coNP}{\ensuremath{\mathsf{coNP}}}

    % Commonly-used reductions add semantic clarity
    \newcommand{\CookReducesTo}{\ensuremath{\leq^P_T}}
    \newcommand{\KarpReducesTo}{\ensuremath{\leq^P_m}}
    \mdfsetup{
        innertopmargin=10pt,linecolor=green!30,
        linewidth=2pt,topline=true,
        frametitleaboveskip=\dimexpr - \ht\strutbox\relax
    }
    \begin{mdframed}
}{
    \end{mdframed}
}
%-------%
% FONTS %
%-------%

% Skip lines and don't indent
\setlength{\parindent}{0em}
\setlength{\parskip}{1em}

% Import math fonts and symbols
\RequirePackage{amsfonts}
\RequirePackage{amsmath,amsthm,amssymb}

% Palatino typeface -- includes (old-style) text figures
\RequirePackage[osf]{mathpazo}

% Hipervínculos
\hypersetup{
    colorlinks=true,
    linkcolor=RobRed,
    filecolor=magenta,      
    urlcolor=cyan,
    pdfauthor={R. Vasquez},
    pdftitle={Tarea 1},
    citecolor=RobRed
}



% ---------- %
% PARAMETERS %
% ---------- %
\newcommand\course{IA \& TC}
\newcommand\coursetitle{Tarea 1}
\newcommand\prof{Jesús Rodríguez Viorato}
\newcommand\semester{Agosto-Diciembre 2021}
\newcommand\name{Roberto Vásquez Martínez}

% -------- %
% DOCUMENT %
% -------- %
\tikzset{
->, % makes the edges directed
%>=stealth’, % makes the arrow heads bold
node distance=3cm, % specifies the minimum distance between two nodes. Change if necessary.
every state/.style={thick, fill=gray!10}, % sets the properties for each ’state’ node
initial text=$ $, % sets the text that appears on the start arrow
}
\begin{document}
\begin{titlepage}
    \centering
    \vspace*{\fill}
    \textsc{\LARGE{\coursetitle}}\\[0.3cm]
    \textsc{\Large{Profesor: \prof}}\\[0.3cm]
    \textsc{\large{\course}}\\[0.3cm]
    \textsc{\large{\semester}}\\
    \vspace*{\fill}
    \textsc{\name}
\end{titlepage}
\pagebreak
\setcounter{page}{1}

\pagestyle{fancy}

%%%%%%%%%%%%%%
% Problema 1 %
%%%%%%%%%%%%%%
\begin{question}[1]
(30 pts) Dibuja un autómata que acepte el lenguaje de todas las cadenas en $\{0,1\}^{\ast}$ que tengan un múltiplo de 3 de 1's
\end{question}

\Proof 
Sea $L_1$ el lenguaje que consiste en todas las cadenas en $\{0,1\}^{\ast}$ con un múltiplo de 3 de 1's.

Para cada cadena en $\{0,1\}^{\ast}$ tenemos que el número de $1's$ en la cadena puede ser congruente con $0,1,2$ módulo $3$, los estados $q_0,q_1$ y $q_2$ representarán cada una de estas posibilidades. 

Por lo que pide el problema, queremos que nuestro estado aceptor sea el $q_0$. Si estoy en $q_0$ para regresar a este camino por siguiente vez consecutiva se debió agregar exactamente tres 1's más. Con esta heurística el autómata finito deseado, que denotaremos por $M_1$, es el siguiente:

\begin{figure}[h!]
    \centering 
    \begin{tikzpicture}
        \node[state,accepting, initial] (q0) {$q_0$};
        \node[state, right of=q0] (q1) {$q_1$};
        \node[state, right of=q1] (q2) {$q_2$};
        \draw (q0) edge[loop above] node{0} (q0)
              (q1) edge[loop above] node{0} (q1)
              (q2) edge[loop above] node{0} (q2)
              (q0) edge[above] node{1} (q1)
              (q1) edge[above] node{1} (q2)
              (q2) edge[bend left, below] node{1} (q0);
    \end{tikzpicture}
    \caption{Diagrama que ilustra al autómata $M_1$}
\end{figure}

y esto es lo que queríamos. \qed
\clearpage
%%%%%%%%%%%%%%
% Problema 2 %
%%%%%%%%%%%%%%
\begin{question}[2]
    (40 pts) Dibuja un autómata que acepte el lenguaje de todas las cadenas de $\{0,1\}^{\ast}$ que representen números en binario divisibles por 3.
\end{question}

\Proof 
Sea $L_2$ el lenguaje de los números en expansión binaria divisibles por 3. 

De manera similar al problema anterior, para cada número en expansión binaria este pertenece a una clase de equivalencia módulo 3. Los estados $p_0,p_1$ y $p_2$ representarán las clases de equivalencia $0,1$ y $2$ módulo 3. 

Enlistando consecutivamente los números en su expansión binaria investigamos las transiciones posibles a cada uno de los estados. El estado aceptor es $p_0$ pues nos interesan los números múltiplos de 3; denotamos por $M_2$ al autómata que acepta el lenguaje $L_2$ y lo representamos en el siguiente diagrama:

\begin{figure}[h!]
    \centering 
    \begin{tikzpicture}
        \node[state,accepting, initial] (p0) {$p_0$};
        \node[state, right of=p0] (p1) {$p_1$};
        \node[state, right of=p1] (p2) {$p_2$};
        \draw (p0) edge[loop above] node{0} (p0)
              (p0) edge[bend left,above] node{1} (p1)

              (p1) edge[bend left, above] node{0} (p2)
              (p1) edge[bend left,below] node{1} (p0)

              (p2) edge[bend left, below] node{0} (p1)
              (p2) edge[loop above] node{1} (p2);
    \end{tikzpicture}
    \caption{Diagrama que ilustra al autómata $M_2$}
\end{figure}

y este es el autómata buscado. \qed

\clearpage
%%%%%%%%%%%%%%
% Problema 3 %
%%%%%%%%%%%%%%
\begin{question}[3]
    (30 pts) Construye el autómata producto de los autómatas que encontraste en los problemas 1 y 2. Esto para obtener un autómada que reconozca el lenguaje de las cadenas de 0's y 1's que en binario sean divisibles por 3 pero que NO tengan un múltiplo de 3 de 1's.
\end{question}

\Proof 
Estamos buscando el autómata que acepte el lenguaje $L_2-L_1$. Sean $Q_1$ y $Q_2$ los estados de los autómatas $M_1$ y $M_2$ construidos en los problemas anteriores. Además, sean $A_1$ y $A_2$ los estados aceptores correspondientes. 

Construiremos el autómata producto, denotado por $M$, que acepte al lenguaje $L_2-L_1$. Sea $Q=Q_2\times Q_1$ el conjunto de estados del autómata producto, si $A$ es el conjunto de estados aceptores entonces como estamos interesados en el lenguaje $L_2-L_1$ se tiene que 
\[
     A=\cbr{(p,q)\in Q\ :\ p\in A_2\y q\not\in A_1}=\{(p_0,q_1),(p_0,q_2)\}.
\]

Considerando el conjunto de estados aceptores $A$ y la función de transición para el autómata producto, representamos a través del siguiente diagrama al autómata $M$ que acepta el lenguaje $L_2-L_1$:

\begin{figure}[h!]
    \centering 
    \begin{tikzpicture}
        \node[state] (11) {$(p_1,q_1)$};
        \node[state,accepting, left of=11] (01) {$(p_0,q_1)$};
        \node[state, right of=11] (21) {$(p_2,q_1)$};
        \node[state, initial, below of=01] (00) {$(p_0,q_0)$};
        \node[state, below of=11] (10) {$(p_1,q_0)$};
        \node[state, below of=21] (20) {$(p_2,q_0)$};
        \node[state,accepting, above of=01] (02) {$(p_0,q_2)$};
        \node[state, above of=11] (12) {$(p_1,q_2)$};
        \node[state, above of=21] (22) {$(p_2,q_2)$};
        \draw (00) edge[loop below] node{0} (00)
              (00) edge[bend right,below] node{1} (11)
              (10) edge[bend right,below] node{0} (20)
              (10) edge[bend left,below] node{1} (01)
              (20) edge[bend right,above] node{0} (10)
              (20) edge[right] node{1} (21)

              (01) edge[loop left] node{0} (01)
              (01) edge[bend right,below] node{1} (12)
              (11) edge[bend right,below] node{0} (21)
              (11) edge[bend left,below] node{1} (02)
              (21) edge[bend right, above] node{0} (11)
              (21) edge[right] node{1} (22)

              (02) edge[loop left] node{0} (02)
              (02) edge[left] node{1} (10)
              (12) edge[bend right, below] node{0} (22)
              (12) edge[right] node{1} (00)
              (22) edge[bend right, above] node{0} (12)
              (22) edge[bend left, right] node{1} (20);


        %      (p1) edge[bend left, above] node{0} (p2)
        %      (p1) edge[bend left,below] node{1} (p0)

        %      (p2) edge[bend left, below] node{0} (p1)
        %      (p2) edge[loop above] node{1} (p2);
    \end{tikzpicture}
    \caption{Diagrama que ilustra al autómata $M$}
\end{figure}

y este es el autómata deseado. \qed



\end{document}


