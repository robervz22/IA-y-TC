\documentclass[11pt]{article}

    \usepackage[breakable]{tcolorbox}
    \usepackage{parskip} % Stop auto-indenting (to mimic markdown behaviour)
    
    \usepackage{iftex}
    \ifPDFTeX
    	\usepackage[T1]{fontenc}
    	\usepackage{mathpazo}
    \else
    	\usepackage{fontspec}
    \fi

    % Basic figure setup, for now with no caption control since it's done
    % automatically by Pandoc (which extracts ![](path) syntax from Markdown).
    \usepackage{graphicx}
    % Maintain compatibility with old templates. Remove in nbconvert 6.0
    \let\Oldincludegraphics\includegraphics
    % Ensure that by default, figures have no caption (until we provide a
    % proper Figure object with a Caption API and a way to capture that
    % in the conversion process - todo).
    \usepackage{caption}
    \DeclareCaptionFormat{nocaption}{}
    \captionsetup{format=nocaption,aboveskip=0pt,belowskip=0pt}

    \usepackage[Export]{adjustbox} % Used to constrain images to a maximum size
    \adjustboxset{max size={0.9\linewidth}{0.9\paperheight}}
    \usepackage{float}
    \floatplacement{figure}{H} % forces figures to be placed at the correct location
    \usepackage{xcolor} % Allow colors to be defined
    \usepackage{enumerate} % Needed for markdown enumerations to work
    \usepackage{geometry} % Used to adjust the document margins
    \usepackage{amsmath} % Equations
    \usepackage{amssymb} % Equations
    \usepackage{textcomp} % defines textquotesingle
    % Hack from http://tex.stackexchange.com/a/47451/13684:
    \AtBeginDocument{%
        \def\PYZsq{\textquotesingle}% Upright quotes in Pygmentized code
    }
    \usepackage{upquote} % Upright quotes for verbatim code
    \usepackage{eurosym} % defines \euro
    \usepackage[mathletters]{ucs} % Extended unicode (utf-8) support
    \usepackage{fancyvrb} % verbatim replacement that allows latex
    \usepackage{grffile} % extends the file name processing of package graphics 
                         % to support a larger range
    \makeatletter % fix for grffile with XeLaTeX
    \def\Gread@@xetex#1{%
      \IfFileExists{"\Gin@base".bb}%
      {\Gread@eps{\Gin@base.bb}}%
      {\Gread@@xetex@aux#1}%
    }
    \makeatother

    % The hyperref package gives us a pdf with properly built
    % internal navigation ('pdf bookmarks' for the table of contents,
    % internal cross-reference links, web links for URLs, etc.)
    \usepackage{hyperref}
    % The default LaTeX title has an obnoxious amount of whitespace. By default,
    % titling removes some of it. It also provides customization options.
    \usepackage{titling}
    \usepackage{longtable} % longtable support required by pandoc >1.10
    \usepackage{booktabs}  % table support for pandoc > 1.12.2
    \usepackage[inline]{enumitem} % IRkernel/repr support (it uses the enumerate* environment)
    \usepackage[normalem]{ulem} % ulem is needed to support strikethroughs (\sout)
                                % normalem makes italics be italics, not underlines
    \usepackage{mathrsfs}
    

    
    % Colors for the hyperref package
    \definecolor{urlcolor}{rgb}{0,.145,.698}
    \definecolor{linkcolor}{rgb}{.71,0.21,0.01}
    \definecolor{citecolor}{rgb}{.12,.54,.11}

    % ANSI colors
    \definecolor{ansi-black}{HTML}{3E424D}
    \definecolor{ansi-black-intense}{HTML}{282C36}
    \definecolor{ansi-red}{HTML}{E75C58}
    \definecolor{ansi-red-intense}{HTML}{B22B31}
    \definecolor{ansi-green}{HTML}{00A250}
    \definecolor{ansi-green-intense}{HTML}{007427}
    \definecolor{ansi-yellow}{HTML}{DDB62B}
    \definecolor{ansi-yellow-intense}{HTML}{B27D12}
    \definecolor{ansi-blue}{HTML}{208FFB}
    \definecolor{ansi-blue-intense}{HTML}{0065CA}
    \definecolor{ansi-magenta}{HTML}{D160C4}
    \definecolor{ansi-magenta-intense}{HTML}{A03196}
    \definecolor{ansi-cyan}{HTML}{60C6C8}
    \definecolor{ansi-cyan-intense}{HTML}{258F8F}
    \definecolor{ansi-white}{HTML}{C5C1B4}
    \definecolor{ansi-white-intense}{HTML}{A1A6B2}
    \definecolor{ansi-default-inverse-fg}{HTML}{FFFFFF}
    \definecolor{ansi-default-inverse-bg}{HTML}{000000}

    % commands and environments needed by pandoc snippets
    % extracted from the output of `pandoc -s`
    \providecommand{\tightlist}{%
      \setlength{\itemsep}{0pt}\setlength{\parskip}{0pt}}
    \DefineVerbatimEnvironment{Highlighting}{Verbatim}{commandchars=\\\{\}}
    % Add ',fontsize=\small' for more characters per line
    \newenvironment{Shaded}{}{}
    \newcommand{\KeywordTok}[1]{\textcolor[rgb]{0.00,0.44,0.13}{\textbf{{#1}}}}
    \newcommand{\DataTypeTok}[1]{\textcolor[rgb]{0.56,0.13,0.00}{{#1}}}
    \newcommand{\DecValTok}[1]{\textcolor[rgb]{0.25,0.63,0.44}{{#1}}}
    \newcommand{\BaseNTok}[1]{\textcolor[rgb]{0.25,0.63,0.44}{{#1}}}
    \newcommand{\FloatTok}[1]{\textcolor[rgb]{0.25,0.63,0.44}{{#1}}}
    \newcommand{\CharTok}[1]{\textcolor[rgb]{0.25,0.44,0.63}{{#1}}}
    \newcommand{\StringTok}[1]{\textcolor[rgb]{0.25,0.44,0.63}{{#1}}}
    \newcommand{\CommentTok}[1]{\textcolor[rgb]{0.38,0.63,0.69}{\textit{{#1}}}}
    \newcommand{\OtherTok}[1]{\textcolor[rgb]{0.00,0.44,0.13}{{#1}}}
    \newcommand{\AlertTok}[1]{\textcolor[rgb]{1.00,0.00,0.00}{\textbf{{#1}}}}
    \newcommand{\FunctionTok}[1]{\textcolor[rgb]{0.02,0.16,0.49}{{#1}}}
    \newcommand{\RegionMarkerTok}[1]{{#1}}
    \newcommand{\ErrorTok}[1]{\textcolor[rgb]{1.00,0.00,0.00}{\textbf{{#1}}}}
    \newcommand{\NormalTok}[1]{{#1}}
    
    % Additional commands for more recent versions of Pandoc
    \newcommand{\ConstantTok}[1]{\textcolor[rgb]{0.53,0.00,0.00}{{#1}}}
    \newcommand{\SpecialCharTok}[1]{\textcolor[rgb]{0.25,0.44,0.63}{{#1}}}
    \newcommand{\VerbatimStringTok}[1]{\textcolor[rgb]{0.25,0.44,0.63}{{#1}}}
    \newcommand{\SpecialStringTok}[1]{\textcolor[rgb]{0.73,0.40,0.53}{{#1}}}
    \newcommand{\ImportTok}[1]{{#1}}
    \newcommand{\DocumentationTok}[1]{\textcolor[rgb]{0.73,0.13,0.13}{\textit{{#1}}}}
    \newcommand{\AnnotationTok}[1]{\textcolor[rgb]{0.38,0.63,0.69}{\textbf{\textit{{#1}}}}}
    \newcommand{\CommentVarTok}[1]{\textcolor[rgb]{0.38,0.63,0.69}{\textbf{\textit{{#1}}}}}
    \newcommand{\VariableTok}[1]{\textcolor[rgb]{0.10,0.09,0.49}{{#1}}}
    \newcommand{\ControlFlowTok}[1]{\textcolor[rgb]{0.00,0.44,0.13}{\textbf{{#1}}}}
    \newcommand{\OperatorTok}[1]{\textcolor[rgb]{0.40,0.40,0.40}{{#1}}}
    \newcommand{\BuiltInTok}[1]{{#1}}
    \newcommand{\ExtensionTok}[1]{{#1}}
    \newcommand{\PreprocessorTok}[1]{\textcolor[rgb]{0.74,0.48,0.00}{{#1}}}
    \newcommand{\AttributeTok}[1]{\textcolor[rgb]{0.49,0.56,0.16}{{#1}}}
    \newcommand{\InformationTok}[1]{\textcolor[rgb]{0.38,0.63,0.69}{\textbf{\textit{{#1}}}}}
    \newcommand{\WarningTok}[1]{\textcolor[rgb]{0.38,0.63,0.69}{\textbf{\textit{{#1}}}}}
    
    
    % Define a nice break command that doesn't care if a line doesn't already
    % exist.
    \def\br{\hspace*{\fill} \\* }
    % Math Jax compatibility definitions
    \def\gt{>}
    \def\lt{<}
    \let\Oldtex\TeX
    \let\Oldlatex\LaTeX
    \renewcommand{\TeX}{\textrm{\Oldtex}}
    \renewcommand{\LaTeX}{\textrm{\Oldlatex}}
    % Document parameters
    % Document title
    \title{Tarea\_4\_IA}
    
    
    
    
    
% Pygments definitions
\makeatletter
\def\PY@reset{\let\PY@it=\relax \let\PY@bf=\relax%
    \let\PY@ul=\relax \let\PY@tc=\relax%
    \let\PY@bc=\relax \let\PY@ff=\relax}
\def\PY@tok#1{\csname PY@tok@#1\endcsname}
\def\PY@toks#1+{\ifx\relax#1\empty\else%
    \PY@tok{#1}\expandafter\PY@toks\fi}
\def\PY@do#1{\PY@bc{\PY@tc{\PY@ul{%
    \PY@it{\PY@bf{\PY@ff{#1}}}}}}}
\def\PY#1#2{\PY@reset\PY@toks#1+\relax+\PY@do{#2}}

\expandafter\def\csname PY@tok@w\endcsname{\def\PY@tc##1{\textcolor[rgb]{0.73,0.73,0.73}{##1}}}
\expandafter\def\csname PY@tok@c\endcsname{\let\PY@it=\textit\def\PY@tc##1{\textcolor[rgb]{0.25,0.50,0.50}{##1}}}
\expandafter\def\csname PY@tok@cp\endcsname{\def\PY@tc##1{\textcolor[rgb]{0.74,0.48,0.00}{##1}}}
\expandafter\def\csname PY@tok@k\endcsname{\let\PY@bf=\textbf\def\PY@tc##1{\textcolor[rgb]{0.00,0.50,0.00}{##1}}}
\expandafter\def\csname PY@tok@kp\endcsname{\def\PY@tc##1{\textcolor[rgb]{0.00,0.50,0.00}{##1}}}
\expandafter\def\csname PY@tok@kt\endcsname{\def\PY@tc##1{\textcolor[rgb]{0.69,0.00,0.25}{##1}}}
\expandafter\def\csname PY@tok@o\endcsname{\def\PY@tc##1{\textcolor[rgb]{0.40,0.40,0.40}{##1}}}
\expandafter\def\csname PY@tok@ow\endcsname{\let\PY@bf=\textbf\def\PY@tc##1{\textcolor[rgb]{0.67,0.13,1.00}{##1}}}
\expandafter\def\csname PY@tok@nb\endcsname{\def\PY@tc##1{\textcolor[rgb]{0.00,0.50,0.00}{##1}}}
\expandafter\def\csname PY@tok@nf\endcsname{\def\PY@tc##1{\textcolor[rgb]{0.00,0.00,1.00}{##1}}}
\expandafter\def\csname PY@tok@nc\endcsname{\let\PY@bf=\textbf\def\PY@tc##1{\textcolor[rgb]{0.00,0.00,1.00}{##1}}}
\expandafter\def\csname PY@tok@nn\endcsname{\let\PY@bf=\textbf\def\PY@tc##1{\textcolor[rgb]{0.00,0.00,1.00}{##1}}}
\expandafter\def\csname PY@tok@ne\endcsname{\let\PY@bf=\textbf\def\PY@tc##1{\textcolor[rgb]{0.82,0.25,0.23}{##1}}}
\expandafter\def\csname PY@tok@nv\endcsname{\def\PY@tc##1{\textcolor[rgb]{0.10,0.09,0.49}{##1}}}
\expandafter\def\csname PY@tok@no\endcsname{\def\PY@tc##1{\textcolor[rgb]{0.53,0.00,0.00}{##1}}}
\expandafter\def\csname PY@tok@nl\endcsname{\def\PY@tc##1{\textcolor[rgb]{0.63,0.63,0.00}{##1}}}
\expandafter\def\csname PY@tok@ni\endcsname{\let\PY@bf=\textbf\def\PY@tc##1{\textcolor[rgb]{0.60,0.60,0.60}{##1}}}
\expandafter\def\csname PY@tok@na\endcsname{\def\PY@tc##1{\textcolor[rgb]{0.49,0.56,0.16}{##1}}}
\expandafter\def\csname PY@tok@nt\endcsname{\let\PY@bf=\textbf\def\PY@tc##1{\textcolor[rgb]{0.00,0.50,0.00}{##1}}}
\expandafter\def\csname PY@tok@nd\endcsname{\def\PY@tc##1{\textcolor[rgb]{0.67,0.13,1.00}{##1}}}
\expandafter\def\csname PY@tok@s\endcsname{\def\PY@tc##1{\textcolor[rgb]{0.73,0.13,0.13}{##1}}}
\expandafter\def\csname PY@tok@sd\endcsname{\let\PY@it=\textit\def\PY@tc##1{\textcolor[rgb]{0.73,0.13,0.13}{##1}}}
\expandafter\def\csname PY@tok@si\endcsname{\let\PY@bf=\textbf\def\PY@tc##1{\textcolor[rgb]{0.73,0.40,0.53}{##1}}}
\expandafter\def\csname PY@tok@se\endcsname{\let\PY@bf=\textbf\def\PY@tc##1{\textcolor[rgb]{0.73,0.40,0.13}{##1}}}
\expandafter\def\csname PY@tok@sr\endcsname{\def\PY@tc##1{\textcolor[rgb]{0.73,0.40,0.53}{##1}}}
\expandafter\def\csname PY@tok@ss\endcsname{\def\PY@tc##1{\textcolor[rgb]{0.10,0.09,0.49}{##1}}}
\expandafter\def\csname PY@tok@sx\endcsname{\def\PY@tc##1{\textcolor[rgb]{0.00,0.50,0.00}{##1}}}
\expandafter\def\csname PY@tok@m\endcsname{\def\PY@tc##1{\textcolor[rgb]{0.40,0.40,0.40}{##1}}}
\expandafter\def\csname PY@tok@gh\endcsname{\let\PY@bf=\textbf\def\PY@tc##1{\textcolor[rgb]{0.00,0.00,0.50}{##1}}}
\expandafter\def\csname PY@tok@gu\endcsname{\let\PY@bf=\textbf\def\PY@tc##1{\textcolor[rgb]{0.50,0.00,0.50}{##1}}}
\expandafter\def\csname PY@tok@gd\endcsname{\def\PY@tc##1{\textcolor[rgb]{0.63,0.00,0.00}{##1}}}
\expandafter\def\csname PY@tok@gi\endcsname{\def\PY@tc##1{\textcolor[rgb]{0.00,0.63,0.00}{##1}}}
\expandafter\def\csname PY@tok@gr\endcsname{\def\PY@tc##1{\textcolor[rgb]{1.00,0.00,0.00}{##1}}}
\expandafter\def\csname PY@tok@ge\endcsname{\let\PY@it=\textit}
\expandafter\def\csname PY@tok@gs\endcsname{\let\PY@bf=\textbf}
\expandafter\def\csname PY@tok@gp\endcsname{\let\PY@bf=\textbf\def\PY@tc##1{\textcolor[rgb]{0.00,0.00,0.50}{##1}}}
\expandafter\def\csname PY@tok@go\endcsname{\def\PY@tc##1{\textcolor[rgb]{0.53,0.53,0.53}{##1}}}
\expandafter\def\csname PY@tok@gt\endcsname{\def\PY@tc##1{\textcolor[rgb]{0.00,0.27,0.87}{##1}}}
\expandafter\def\csname PY@tok@err\endcsname{\def\PY@bc##1{\setlength{\fboxsep}{0pt}\fcolorbox[rgb]{1.00,0.00,0.00}{1,1,1}{\strut ##1}}}
\expandafter\def\csname PY@tok@kc\endcsname{\let\PY@bf=\textbf\def\PY@tc##1{\textcolor[rgb]{0.00,0.50,0.00}{##1}}}
\expandafter\def\csname PY@tok@kd\endcsname{\let\PY@bf=\textbf\def\PY@tc##1{\textcolor[rgb]{0.00,0.50,0.00}{##1}}}
\expandafter\def\csname PY@tok@kn\endcsname{\let\PY@bf=\textbf\def\PY@tc##1{\textcolor[rgb]{0.00,0.50,0.00}{##1}}}
\expandafter\def\csname PY@tok@kr\endcsname{\let\PY@bf=\textbf\def\PY@tc##1{\textcolor[rgb]{0.00,0.50,0.00}{##1}}}
\expandafter\def\csname PY@tok@bp\endcsname{\def\PY@tc##1{\textcolor[rgb]{0.00,0.50,0.00}{##1}}}
\expandafter\def\csname PY@tok@fm\endcsname{\def\PY@tc##1{\textcolor[rgb]{0.00,0.00,1.00}{##1}}}
\expandafter\def\csname PY@tok@vc\endcsname{\def\PY@tc##1{\textcolor[rgb]{0.10,0.09,0.49}{##1}}}
\expandafter\def\csname PY@tok@vg\endcsname{\def\PY@tc##1{\textcolor[rgb]{0.10,0.09,0.49}{##1}}}
\expandafter\def\csname PY@tok@vi\endcsname{\def\PY@tc##1{\textcolor[rgb]{0.10,0.09,0.49}{##1}}}
\expandafter\def\csname PY@tok@vm\endcsname{\def\PY@tc##1{\textcolor[rgb]{0.10,0.09,0.49}{##1}}}
\expandafter\def\csname PY@tok@sa\endcsname{\def\PY@tc##1{\textcolor[rgb]{0.73,0.13,0.13}{##1}}}
\expandafter\def\csname PY@tok@sb\endcsname{\def\PY@tc##1{\textcolor[rgb]{0.73,0.13,0.13}{##1}}}
\expandafter\def\csname PY@tok@sc\endcsname{\def\PY@tc##1{\textcolor[rgb]{0.73,0.13,0.13}{##1}}}
\expandafter\def\csname PY@tok@dl\endcsname{\def\PY@tc##1{\textcolor[rgb]{0.73,0.13,0.13}{##1}}}
\expandafter\def\csname PY@tok@s2\endcsname{\def\PY@tc##1{\textcolor[rgb]{0.73,0.13,0.13}{##1}}}
\expandafter\def\csname PY@tok@sh\endcsname{\def\PY@tc##1{\textcolor[rgb]{0.73,0.13,0.13}{##1}}}
\expandafter\def\csname PY@tok@s1\endcsname{\def\PY@tc##1{\textcolor[rgb]{0.73,0.13,0.13}{##1}}}
\expandafter\def\csname PY@tok@mb\endcsname{\def\PY@tc##1{\textcolor[rgb]{0.40,0.40,0.40}{##1}}}
\expandafter\def\csname PY@tok@mf\endcsname{\def\PY@tc##1{\textcolor[rgb]{0.40,0.40,0.40}{##1}}}
\expandafter\def\csname PY@tok@mh\endcsname{\def\PY@tc##1{\textcolor[rgb]{0.40,0.40,0.40}{##1}}}
\expandafter\def\csname PY@tok@mi\endcsname{\def\PY@tc##1{\textcolor[rgb]{0.40,0.40,0.40}{##1}}}
\expandafter\def\csname PY@tok@il\endcsname{\def\PY@tc##1{\textcolor[rgb]{0.40,0.40,0.40}{##1}}}
\expandafter\def\csname PY@tok@mo\endcsname{\def\PY@tc##1{\textcolor[rgb]{0.40,0.40,0.40}{##1}}}
\expandafter\def\csname PY@tok@ch\endcsname{\let\PY@it=\textit\def\PY@tc##1{\textcolor[rgb]{0.25,0.50,0.50}{##1}}}
\expandafter\def\csname PY@tok@cm\endcsname{\let\PY@it=\textit\def\PY@tc##1{\textcolor[rgb]{0.25,0.50,0.50}{##1}}}
\expandafter\def\csname PY@tok@cpf\endcsname{\let\PY@it=\textit\def\PY@tc##1{\textcolor[rgb]{0.25,0.50,0.50}{##1}}}
\expandafter\def\csname PY@tok@c1\endcsname{\let\PY@it=\textit\def\PY@tc##1{\textcolor[rgb]{0.25,0.50,0.50}{##1}}}
\expandafter\def\csname PY@tok@cs\endcsname{\let\PY@it=\textit\def\PY@tc##1{\textcolor[rgb]{0.25,0.50,0.50}{##1}}}

\def\PYZbs{\char`\\}
\def\PYZus{\char`\_}
\def\PYZob{\char`\{}
\def\PYZcb{\char`\}}
\def\PYZca{\char`\^}
\def\PYZam{\char`\&}
\def\PYZlt{\char`\<}
\def\PYZgt{\char`\>}
\def\PYZsh{\char`\#}
\def\PYZpc{\char`\%}
\def\PYZdl{\char`\$}
\def\PYZhy{\char`\-}
\def\PYZsq{\char`\'}
\def\PYZdq{\char`\"}
\def\PYZti{\char`\~}
% for compatibility with earlier versions
\def\PYZat{@}
\def\PYZlb{[}
\def\PYZrb{]}
\makeatother


    % For linebreaks inside Verbatim environment from package fancyvrb. 
    \makeatletter
        \newbox\Wrappedcontinuationbox 
        \newbox\Wrappedvisiblespacebox 
        \newcommand*\Wrappedvisiblespace {\textcolor{red}{\textvisiblespace}} 
        \newcommand*\Wrappedcontinuationsymbol {\textcolor{red}{\llap{\tiny$\m@th\hookrightarrow$}}} 
        \newcommand*\Wrappedcontinuationindent {3ex } 
        \newcommand*\Wrappedafterbreak {\kern\Wrappedcontinuationindent\copy\Wrappedcontinuationbox} 
        % Take advantage of the already applied Pygments mark-up to insert 
        % potential linebreaks for TeX processing. 
        %        {, <, #, %, $, ' and ": go to next line. 
        %        _, }, ^, &, >, - and ~: stay at end of broken line. 
        % Use of \textquotesingle for straight quote. 
        \newcommand*\Wrappedbreaksatspecials {% 
            \def\PYGZus{\discretionary{\char`\_}{\Wrappedafterbreak}{\char`\_}}% 
            \def\PYGZob{\discretionary{}{\Wrappedafterbreak\char`\{}{\char`\{}}% 
            \def\PYGZcb{\discretionary{\char`\}}{\Wrappedafterbreak}{\char`\}}}% 
            \def\PYGZca{\discretionary{\char`\^}{\Wrappedafterbreak}{\char`\^}}% 
            \def\PYGZam{\discretionary{\char`\&}{\Wrappedafterbreak}{\char`\&}}% 
            \def\PYGZlt{\discretionary{}{\Wrappedafterbreak\char`\<}{\char`\<}}% 
            \def\PYGZgt{\discretionary{\char`\>}{\Wrappedafterbreak}{\char`\>}}% 
            \def\PYGZsh{\discretionary{}{\Wrappedafterbreak\char`\#}{\char`\#}}% 
            \def\PYGZpc{\discretionary{}{\Wrappedafterbreak\char`\%}{\char`\%}}% 
            \def\PYGZdl{\discretionary{}{\Wrappedafterbreak\char`\$}{\char`\$}}% 
            \def\PYGZhy{\discretionary{\char`\-}{\Wrappedafterbreak}{\char`\-}}% 
            \def\PYGZsq{\discretionary{}{\Wrappedafterbreak\textquotesingle}{\textquotesingle}}% 
            \def\PYGZdq{\discretionary{}{\Wrappedafterbreak\char`\"}{\char`\"}}% 
            \def\PYGZti{\discretionary{\char`\~}{\Wrappedafterbreak}{\char`\~}}% 
        } 
        % Some characters . , ; ? ! / are not pygmentized. 
        % This macro makes them "active" and they will insert potential linebreaks 
        \newcommand*\Wrappedbreaksatpunct {% 
            \lccode`\~`\.\lowercase{\def~}{\discretionary{\hbox{\char`\.}}{\Wrappedafterbreak}{\hbox{\char`\.}}}% 
            \lccode`\~`\,\lowercase{\def~}{\discretionary{\hbox{\char`\,}}{\Wrappedafterbreak}{\hbox{\char`\,}}}% 
            \lccode`\~`\;\lowercase{\def~}{\discretionary{\hbox{\char`\;}}{\Wrappedafterbreak}{\hbox{\char`\;}}}% 
            \lccode`\~`\:\lowercase{\def~}{\discretionary{\hbox{\char`\:}}{\Wrappedafterbreak}{\hbox{\char`\:}}}% 
            \lccode`\~`\?\lowercase{\def~}{\discretionary{\hbox{\char`\?}}{\Wrappedafterbreak}{\hbox{\char`\?}}}% 
            \lccode`\~`\!\lowercase{\def~}{\discretionary{\hbox{\char`\!}}{\Wrappedafterbreak}{\hbox{\char`\!}}}% 
            \lccode`\~`\/\lowercase{\def~}{\discretionary{\hbox{\char`\/}}{\Wrappedafterbreak}{\hbox{\char`\/}}}% 
            \catcode`\.\active
            \catcode`\,\active 
            \catcode`\;\active
            \catcode`\:\active
            \catcode`\?\active
            \catcode`\!\active
            \catcode`\/\active 
            \lccode`\~`\~ 	
        }
    \makeatother

    \let\OriginalVerbatim=\Verbatim
    \makeatletter
    \renewcommand{\Verbatim}[1][1]{%
        %\parskip\z@skip
        \sbox\Wrappedcontinuationbox {\Wrappedcontinuationsymbol}%
        \sbox\Wrappedvisiblespacebox {\FV@SetupFont\Wrappedvisiblespace}%
        \def\FancyVerbFormatLine ##1{\hsize\linewidth
            \vtop{\raggedright\hyphenpenalty\z@\exhyphenpenalty\z@
                \doublehyphendemerits\z@\finalhyphendemerits\z@
                \strut ##1\strut}%
        }%
        % If the linebreak is at a space, the latter will be displayed as visible
        % space at end of first line, and a continuation symbol starts next line.
        % Stretch/shrink are however usually zero for typewriter font.
        \def\FV@Space {%
            \nobreak\hskip\z@ plus\fontdimen3\font minus\fontdimen4\font
            \discretionary{\copy\Wrappedvisiblespacebox}{\Wrappedafterbreak}
            {\kern\fontdimen2\font}%
        }%
        
        % Allow breaks at special characters using \PYG... macros.
        \Wrappedbreaksatspecials
        % Breaks at punctuation characters . , ; ? ! and / need catcode=\active 	
        \OriginalVerbatim[#1,codes*=\Wrappedbreaksatpunct]%
    }
    \makeatother

    % Exact colors from NB
    \definecolor{incolor}{HTML}{303F9F}
    \definecolor{outcolor}{HTML}{D84315}
    \definecolor{cellborder}{HTML}{CFCFCF}
    \definecolor{cellbackground}{HTML}{F7F7F7}
    
    % prompt
    \makeatletter
    \newcommand{\boxspacing}{\kern\kvtcb@left@rule\kern\kvtcb@boxsep}
    \makeatother
    \newcommand{\prompt}[4]{
        \ttfamily\llap{{\color{#2}[#3]:\hspace{3pt}#4}}\vspace{-\baselineskip}
    }
    

    
    % Prevent overflowing lines due to hard-to-break entities
    \sloppy 
    % Setup hyperref package
    \hypersetup{
      breaklinks=true,  % so long urls are correctly broken across lines
      colorlinks=true,
      urlcolor=urlcolor,
      linkcolor=linkcolor,
      citecolor=citecolor,
      }
    % Slightly bigger margins than the latex defaults
    
    \geometry{verbose,tmargin=1in,bmargin=1in,lmargin=1in,rmargin=1in}
    
    

\begin{document}
    \title{Tarea 4 IA}
    \author{Roberto Vásquez Martínez \\ Profesor: Arturo Hernández Aguirre}
    \date{12/Diciembre/2021}
    \maketitle
    
    
    \hypertarget{perceptruxf3n-multicapas}{%
\section{Perceptrón Multicapas}\label{perceptruxf3n-multicapas}}

    \hypertarget{datos}{%
\subsection{Datos}\label{datos}}

    A continuación generaremos la muestra que nos servirá para entrenar y
validar el perceptrón multicapa que construiremos. Generaremos 100
puntos en \(\mathbb{R}^2\) que corresponderán a dos clases, la clase
positiva y la negativa.

Un punto \((X_1,X_2)\in\mathbb{R}^2\) pertenecerá a la clase positiva si
\((X_1,X_2)\) se encuentra en el conjunto \(\mathcal{R}\) definido como
\begin{equation*}
    \mathcal{R}=\{(x_1,x_2)\in\mathbb{R}^2\ |\ 2\leq x_1\leq 8\text{ y } 2\leq x_2\leq 6\},
\end{equation*} de lo contrario \((X_1,X_2)\) pertenecerá a la clase
negativa.

    Por lo tanto, la muestra que utilizaremos para entrenar y validar el
perceptrón será \begin{equation*}
    D=\{(X_{1i},X_{2i},Y_i)\in\mathbb{R}^3\ |\ \text{ con }Y_i=\mathbf{1}_{\mathcal{R}}(X_{1i},X_{2i})\text{ y }i=1,2,\dots,100\},
\end{equation*}

donde \(\mathbf{1}_{\mathcal{R}}\) es la función indicadora del conjunto
\(\mathcal{R}\).

Sean \(\mathcal{P},\mathcal{N}\subset D\) los subconjuntos que
corresponden a la clase positiva y negativa respectivamente. De manera
similar, consideremos los subconjuntos \(E,V\subset D\) los conjuntos de
entrenamiento y validación.

La muestra \(D\) cumplirá las siguientes propiedades.

\begin{enumerate}
\def\labelenumi{\arabic{enumi}.}
\tightlist
\item
  \((2,6),(8,6),(8,2),(2,2)\in E\subset D\), es decir, los vértices del
  rectángulo \(\mathcal{R}\) forman parte del conjunto de entrenamiento
  y la clase positiva.
\item
  \(|\mathcal{P}|=|\mathcal{N}|=50\).
\item
  \(|E|=0.7 |D|\), \(|V|=0.3 |D|\) y \(E,V\) forman una partición de
  \(D\), es decir, \(D=E\cup V\) y \(E\cap V=\emptyset\).
\end{enumerate}

    A continuación generamos el conjunto \(D\) con las propiedades
anteriores.

    \begin{tcolorbox}[breakable, size=fbox, boxrule=1pt, pad at break*=1mm,colback=cellbackground, colframe=cellborder]
\prompt{In}{incolor}{1}{\boxspacing}
\begin{Verbatim}[commandchars=\\\{\}]
\PY{k+kn}{import} \PY{n+nn}{numpy} \PY{k}{as} \PY{n+nn}{np} 

\PY{n}{FAR\PYZus{}BORDER}\PY{o}{=}\PY{l+m+mf}{0.2}
\PY{n}{SEED}\PY{o}{=}\PY{l+m+mi}{2}
\PY{n}{MAX\PYZus{}EPOCHS}\PY{o}{=}\PY{l+m+mi}{1000}


\PY{c+c1}{\PYZsh{} Positive class}
\PY{n}{np}\PY{o}{.}\PY{n}{random}\PY{o}{.}\PY{n}{seed}\PY{p}{(}\PY{n}{SEED}\PY{p}{)}
\PY{n}{V}\PY{o}{=}\PY{n}{np}\PY{o}{.}\PY{n}{array}\PY{p}{(}\PY{p}{[}\PY{p}{(}\PY{l+m+mf}{2.0}\PY{p}{,}\PY{l+m+mf}{2.0}\PY{p}{)}\PY{p}{,}\PY{p}{(}\PY{l+m+mf}{8.0}\PY{p}{,}\PY{l+m+mf}{2.0}\PY{p}{)}\PY{p}{,}\PY{p}{(}\PY{l+m+mf}{8.0}\PY{p}{,}\PY{l+m+mf}{6.0}\PY{p}{)}\PY{p}{,}\PY{p}{(}\PY{l+m+mf}{2.0}\PY{p}{,}\PY{l+m+mf}{6.0}\PY{p}{)}\PY{p}{]}\PY{p}{)} \PY{c+c1}{\PYZsh{} Vertexes}
\PY{n}{Px}\PY{p}{,}\PY{n}{Py}\PY{o}{=}\PY{n}{np}\PY{o}{.}\PY{n}{random}\PY{o}{.}\PY{n}{uniform}\PY{p}{(}\PY{l+m+mi}{2}\PY{p}{,}\PY{l+m+mi}{8}\PY{p}{,}\PY{n}{size}\PY{o}{=}\PY{l+m+mi}{46}\PY{p}{)}\PY{p}{,} \PY{n}{np}\PY{o}{.}\PY{n}{random}\PY{o}{.}\PY{n}{uniform}\PY{p}{(}\PY{l+m+mi}{2}\PY{p}{,}\PY{l+m+mi}{6}\PY{p}{,}\PY{n}{size}\PY{o}{=}\PY{l+m+mi}{46}\PY{p}{)}
\PY{n}{P\PYZus{}random}\PY{o}{=}\PY{p}{[}\PY{p}{(}\PY{n}{Px}\PY{p}{[}\PY{n}{i}\PY{p}{]}\PY{p}{,}\PY{n}{Py}\PY{p}{[}\PY{n}{i}\PY{p}{]}\PY{p}{)} \PY{k}{for} \PY{n}{i} \PY{o+ow}{in} \PY{n+nb}{range}\PY{p}{(}\PY{n+nb}{len}\PY{p}{(}\PY{n}{Px}\PY{p}{)}\PY{p}{)}\PY{p}{]}
\PY{n}{P}\PY{o}{=}\PY{n}{np}\PY{o}{.}\PY{n}{concatenate}\PY{p}{(}\PY{p}{(}\PY{n}{P\PYZus{}random}\PY{p}{,}\PY{n}{V}\PY{p}{)}\PY{p}{)}

\PY{c+c1}{\PYZsh{} Negative class}
\PY{n}{C}\PY{o}{=}\PY{p}{(}\PY{l+m+mi}{5}\PY{p}{,}\PY{l+m+mi}{4}\PY{p}{)} \PY{c+c1}{\PYZsh{} Center of the Rectangle}
\PY{n}{height\PYZus{}C}\PY{p}{,}\PY{n}{width\PYZus{}C}\PY{o}{=} \PY{l+m+mf}{7.0}\PY{p}{,}\PY{l+m+mf}{5.0}
\PY{n}{N}\PY{o}{=}\PY{p}{[}\PY{p}{]}
\PY{l+s+sd}{\PYZsq{}\PYZsq{}\PYZsq{}First Quadrant\PYZsq{}\PYZsq{}\PYZsq{}}
\PY{n}{num\PYZus{}quadrant}\PY{o}{=}\PY{l+m+mi}{0}
\PY{k}{while} \PY{n}{num\PYZus{}quadrant}\PY{o}{\PYZlt{}}\PY{l+m+mi}{13}\PY{p}{:} 
    \PY{n}{nx}\PY{o}{=}\PY{n}{np}\PY{o}{.}\PY{n}{random}\PY{o}{.}\PY{n}{uniform}\PY{p}{(}\PY{n}{C}\PY{p}{[}\PY{l+m+mi}{0}\PY{p}{]}\PY{p}{,}\PY{n}{C}\PY{p}{[}\PY{l+m+mi}{0}\PY{p}{]}\PY{o}{+}\PY{n}{height\PYZus{}C}\PY{p}{,}\PY{l+m+mi}{1}\PY{p}{)}\PY{p}{[}\PY{l+m+mi}{0}\PY{p}{]}
    \PY{n}{ny}\PY{o}{=}\PY{n}{np}\PY{o}{.}\PY{n}{random}\PY{o}{.}\PY{n}{uniform}\PY{p}{(}\PY{n}{C}\PY{p}{[}\PY{l+m+mi}{1}\PY{p}{]}\PY{p}{,}\PY{n}{C}\PY{p}{[}\PY{l+m+mi}{1}\PY{p}{]}\PY{o}{+}\PY{n}{width\PYZus{}C}\PY{p}{,}\PY{l+m+mi}{1}\PY{p}{)}\PY{p}{[}\PY{l+m+mi}{0}\PY{p}{]}
    \PY{k}{if} \PY{n}{nx}\PY{o}{\PYZgt{}}\PY{l+m+mi}{8} \PY{o+ow}{or} \PY{n}{ny}\PY{o}{\PYZgt{}}\PY{l+m+mi}{6}\PY{p}{:}
        \PY{n}{n}\PY{o}{=}\PY{p}{(}\PY{n}{nx}\PY{o}{+}\PY{n}{FAR\PYZus{}BORDER}\PY{p}{,}\PY{n}{ny}\PY{o}{+}\PY{n}{FAR\PYZus{}BORDER}\PY{p}{)}
        \PY{n}{N}\PY{o}{.}\PY{n}{append}\PY{p}{(}\PY{n}{n}\PY{p}{)}
        \PY{n}{num\PYZus{}quadrant}\PY{o}{+}\PY{o}{=}\PY{l+m+mi}{1}
\PY{l+s+sd}{\PYZsq{}\PYZsq{}\PYZsq{}Second Quadrant\PYZsq{}\PYZsq{}\PYZsq{}}
\PY{n}{num\PYZus{}quadrant}\PY{o}{=}\PY{l+m+mi}{0}
\PY{k}{while} \PY{n}{num\PYZus{}quadrant}\PY{o}{\PYZlt{}}\PY{l+m+mi}{13}\PY{p}{:} 
    \PY{n}{nx}\PY{o}{=}\PY{n}{np}\PY{o}{.}\PY{n}{random}\PY{o}{.}\PY{n}{uniform}\PY{p}{(}\PY{n}{C}\PY{p}{[}\PY{l+m+mi}{0}\PY{p}{]}\PY{p}{,}\PY{n}{C}\PY{p}{[}\PY{l+m+mi}{0}\PY{p}{]}\PY{o}{+}\PY{n}{height\PYZus{}C}\PY{p}{,}\PY{l+m+mi}{1}\PY{p}{)}\PY{p}{[}\PY{l+m+mi}{0}\PY{p}{]}
    \PY{n}{ny}\PY{o}{=}\PY{n}{np}\PY{o}{.}\PY{n}{random}\PY{o}{.}\PY{n}{uniform}\PY{p}{(}\PY{n}{C}\PY{p}{[}\PY{l+m+mi}{1}\PY{p}{]}\PY{o}{\PYZhy{}}\PY{n}{width\PYZus{}C}\PY{p}{,}\PY{n}{C}\PY{p}{[}\PY{l+m+mi}{1}\PY{p}{]}\PY{p}{,}\PY{l+m+mi}{1}\PY{p}{)}\PY{p}{[}\PY{l+m+mi}{0}\PY{p}{]}
    \PY{k}{if} \PY{n}{nx}\PY{o}{\PYZgt{}}\PY{l+m+mi}{8} \PY{o+ow}{or} \PY{n}{ny}\PY{o}{\PYZlt{}}\PY{l+m+mi}{2}\PY{p}{:}
        \PY{n}{n}\PY{o}{=}\PY{p}{(}\PY{n}{nx}\PY{o}{+}\PY{n}{FAR\PYZus{}BORDER}\PY{p}{,}\PY{n}{ny}\PY{o}{\PYZhy{}}\PY{n}{FAR\PYZus{}BORDER}\PY{p}{)}
        \PY{n}{N}\PY{o}{.}\PY{n}{append}\PY{p}{(}\PY{n}{n}\PY{p}{)}
        \PY{n}{num\PYZus{}quadrant}\PY{o}{+}\PY{o}{=}\PY{l+m+mi}{1}
\PY{l+s+sd}{\PYZsq{}\PYZsq{}\PYZsq{}Third Quadrant\PYZsq{}\PYZsq{}\PYZsq{}}
\PY{n}{num\PYZus{}quadrant}\PY{o}{=}\PY{l+m+mi}{0}
\PY{k}{while} \PY{n}{num\PYZus{}quadrant}\PY{o}{\PYZlt{}}\PY{l+m+mi}{12}\PY{p}{:} 
    \PY{n}{nx}\PY{o}{=}\PY{n}{np}\PY{o}{.}\PY{n}{random}\PY{o}{.}\PY{n}{uniform}\PY{p}{(}\PY{n}{C}\PY{p}{[}\PY{l+m+mi}{0}\PY{p}{]}\PY{o}{\PYZhy{}}\PY{n}{height\PYZus{}C}\PY{p}{,}\PY{n}{C}\PY{p}{[}\PY{l+m+mi}{0}\PY{p}{]}\PY{p}{,}\PY{l+m+mi}{1}\PY{p}{)}\PY{p}{[}\PY{l+m+mi}{0}\PY{p}{]}
    \PY{n}{ny}\PY{o}{=}\PY{n}{np}\PY{o}{.}\PY{n}{random}\PY{o}{.}\PY{n}{uniform}\PY{p}{(}\PY{n}{C}\PY{p}{[}\PY{l+m+mi}{1}\PY{p}{]}\PY{o}{\PYZhy{}}\PY{n}{width\PYZus{}C}\PY{p}{,}\PY{n}{C}\PY{p}{[}\PY{l+m+mi}{1}\PY{p}{]}\PY{p}{,}\PY{l+m+mi}{1}\PY{p}{)}\PY{p}{[}\PY{l+m+mi}{0}\PY{p}{]}
    \PY{k}{if} \PY{n}{nx}\PY{o}{\PYZlt{}}\PY{l+m+mi}{2} \PY{o+ow}{or} \PY{n}{ny}\PY{o}{\PYZlt{}}\PY{l+m+mi}{2}\PY{p}{:}
        \PY{n}{n}\PY{o}{=}\PY{p}{(}\PY{n}{nx}\PY{o}{\PYZhy{}}\PY{n}{FAR\PYZus{}BORDER}\PY{p}{,}\PY{n}{ny}\PY{o}{\PYZhy{}}\PY{n}{FAR\PYZus{}BORDER}\PY{p}{)}
        \PY{n}{N}\PY{o}{.}\PY{n}{append}\PY{p}{(}\PY{n}{n}\PY{p}{)}
        \PY{n}{num\PYZus{}quadrant}\PY{o}{+}\PY{o}{=}\PY{l+m+mi}{1}
\PY{l+s+sd}{\PYZsq{}\PYZsq{}\PYZsq{}Fourth Quadrant\PYZsq{}\PYZsq{}\PYZsq{}}
\PY{n}{num\PYZus{}quadrant}\PY{o}{=}\PY{l+m+mi}{0}
\PY{k}{while} \PY{n}{num\PYZus{}quadrant}\PY{o}{\PYZlt{}}\PY{l+m+mi}{12}\PY{p}{:} 
    \PY{n}{nx}\PY{o}{=}\PY{n}{np}\PY{o}{.}\PY{n}{random}\PY{o}{.}\PY{n}{uniform}\PY{p}{(}\PY{n}{C}\PY{p}{[}\PY{l+m+mi}{0}\PY{p}{]}\PY{o}{\PYZhy{}}\PY{n}{height\PYZus{}C}\PY{p}{,}\PY{n}{C}\PY{p}{[}\PY{l+m+mi}{0}\PY{p}{]}\PY{p}{,}\PY{l+m+mi}{1}\PY{p}{)}\PY{p}{[}\PY{l+m+mi}{0}\PY{p}{]}
    \PY{n}{ny}\PY{o}{=}\PY{n}{np}\PY{o}{.}\PY{n}{random}\PY{o}{.}\PY{n}{uniform}\PY{p}{(}\PY{n}{C}\PY{p}{[}\PY{l+m+mi}{1}\PY{p}{]}\PY{p}{,}\PY{n}{C}\PY{p}{[}\PY{l+m+mi}{1}\PY{p}{]}\PY{o}{+}\PY{n}{width\PYZus{}C}\PY{p}{,}\PY{l+m+mi}{1}\PY{p}{)}\PY{p}{[}\PY{l+m+mi}{0}\PY{p}{]}
    \PY{k}{if} \PY{n}{nx}\PY{o}{\PYZlt{}}\PY{l+m+mi}{2} \PY{o+ow}{or} \PY{n}{ny}\PY{o}{\PYZgt{}}\PY{l+m+mi}{6}\PY{p}{:}
        \PY{n}{n}\PY{o}{=}\PY{p}{(}\PY{n}{nx}\PY{o}{\PYZhy{}}\PY{n}{FAR\PYZus{}BORDER}\PY{p}{,}\PY{n}{ny}\PY{o}{+}\PY{n}{FAR\PYZus{}BORDER}\PY{p}{)}
        \PY{n}{N}\PY{o}{.}\PY{n}{append}\PY{p}{(}\PY{n}{n}\PY{p}{)}
        \PY{n}{num\PYZus{}quadrant}\PY{o}{+}\PY{o}{=}\PY{l+m+mi}{1}  
\PY{n}{N}\PY{o}{=}\PY{n}{np}\PY{o}{.}\PY{n}{asarray}\PY{p}{(}\PY{n}{N}\PY{p}{)}      
\end{Verbatim}
\end{tcolorbox}

    Ahora visualizaremos los datos generados en la celda anterior. Marcamos
la frontera del rectángulo \(\mathcal{R}\) para identificar las regiones
de decisión que queremos encontrar.

    \begin{tcolorbox}[breakable, size=fbox, boxrule=1pt, pad at break*=1mm,colback=cellbackground, colframe=cellborder]
\prompt{In}{incolor}{2}{\boxspacing}
\begin{Verbatim}[commandchars=\\\{\}]
\PY{k+kn}{import} \PY{n+nn}{matplotlib}\PY{n+nn}{.}\PY{n+nn}{pyplot} \PY{k}{as} \PY{n+nn}{plt}
\PY{k+kn}{from} \PY{n+nn}{matplotlib} \PY{k+kn}{import} \PY{n}{collections} \PY{k}{as} \PY{n}{mc} 

\PY{n}{rectangle\PYZus{}sides}\PY{o}{=}\PY{p}{[}\PY{p}{[}\PY{p}{(}\PY{l+m+mi}{2}\PY{p}{,}\PY{l+m+mi}{2}\PY{p}{)}\PY{p}{,}\PY{p}{(}\PY{l+m+mi}{8}\PY{p}{,}\PY{l+m+mi}{2}\PY{p}{)}\PY{p}{]}\PY{p}{,}\PY{p}{[}\PY{p}{(}\PY{l+m+mi}{8}\PY{p}{,}\PY{l+m+mi}{2}\PY{p}{)}\PY{p}{,}\PY{p}{(}\PY{l+m+mi}{8}\PY{p}{,}\PY{l+m+mi}{6}\PY{p}{)}\PY{p}{]}\PY{p}{,}\PY{p}{[}\PY{p}{(}\PY{l+m+mi}{2}\PY{p}{,}\PY{l+m+mi}{6}\PY{p}{)}\PY{p}{,}\PY{p}{(}\PY{l+m+mi}{8}\PY{p}{,}\PY{l+m+mi}{6}\PY{p}{)}\PY{p}{]}\PY{p}{,}\PY{p}{[}\PY{p}{(}\PY{l+m+mi}{2}\PY{p}{,}\PY{l+m+mi}{2}\PY{p}{)}\PY{p}{,}\PY{p}{(}\PY{l+m+mi}{2}\PY{p}{,}\PY{l+m+mi}{6}\PY{p}{)}\PY{p}{]}\PY{p}{]}
\PY{n}{c}\PY{o}{=}\PY{n}{np}\PY{o}{.}\PY{n}{array}\PY{p}{(}\PY{p}{[}\PY{l+s+s1}{\PYZsq{}}\PY{l+s+s1}{\PYZsh{}131D63}\PY{l+s+s1}{\PYZsq{}}\PY{p}{,} \PY{l+s+s1}{\PYZsq{}}\PY{l+s+s1}{\PYZsh{}131D63}\PY{l+s+s1}{\PYZsq{}}\PY{p}{,} \PY{l+s+s1}{\PYZsq{}}\PY{l+s+s1}{\PYZsh{}131D63}\PY{l+s+s1}{\PYZsq{}}\PY{p}{,} \PY{l+s+s1}{\PYZsq{}}\PY{l+s+s1}{\PYZsh{}131D63}\PY{l+s+s1}{\PYZsq{}}\PY{p}{]}\PY{p}{)}

\PY{n}{sides}\PY{o}{=}\PY{n}{mc}\PY{o}{.}\PY{n}{LineCollection}\PY{p}{(}\PY{n}{rectangle\PYZus{}sides}\PY{p}{,}\PY{n}{colors}\PY{o}{=}\PY{n}{c}\PY{p}{,}\PY{n}{linewidths}\PY{o}{=}\PY{l+m+mi}{2}\PY{p}{)}
\PY{n}{fig}\PY{p}{,} \PY{n}{ax}\PY{o}{=}\PY{n}{plt}\PY{o}{.}\PY{n}{subplots}\PY{p}{(}\PY{p}{)}
\PY{n}{ax}\PY{o}{.}\PY{n}{scatter}\PY{p}{(}\PY{n}{P}\PY{p}{[}\PY{p}{:}\PY{p}{,}\PY{l+m+mi}{0}\PY{p}{]}\PY{p}{,}\PY{n}{P}\PY{p}{[}\PY{p}{:}\PY{p}{,}\PY{l+m+mi}{1}\PY{p}{]}\PY{p}{,}\PY{n}{marker}\PY{o}{=}\PY{l+s+s1}{\PYZsq{}}\PY{l+s+s1}{\PYZca{}}\PY{l+s+s1}{\PYZsq{}}\PY{p}{,}\PY{n}{alpha}\PY{o}{=}\PY{l+m+mf}{0.5}\PY{p}{,}\PY{n}{c}\PY{o}{=}\PY{l+s+s1}{\PYZsq{}}\PY{l+s+s1}{r}\PY{l+s+s1}{\PYZsq{}}\PY{p}{,}\PY{n}{label}\PY{o}{=}\PY{l+s+s1}{\PYZsq{}}\PY{l+s+s1}{positive}\PY{l+s+s1}{\PYZsq{}}\PY{p}{)}
\PY{n}{ax}\PY{o}{.}\PY{n}{scatter}\PY{p}{(}\PY{n}{N}\PY{p}{[}\PY{p}{:}\PY{p}{,}\PY{l+m+mi}{0}\PY{p}{]}\PY{p}{,}\PY{n}{N}\PY{p}{[}\PY{p}{:}\PY{p}{,}\PY{l+m+mi}{1}\PY{p}{]}\PY{p}{,}\PY{n}{marker}\PY{o}{=}\PY{l+s+s1}{\PYZsq{}}\PY{l+s+s1}{*}\PY{l+s+s1}{\PYZsq{}}\PY{p}{,}\PY{n}{alpha}\PY{o}{=}\PY{l+m+mf}{0.5}\PY{p}{,}\PY{n}{c}\PY{o}{=}\PY{l+s+s1}{\PYZsq{}}\PY{l+s+s1}{b}\PY{l+s+s1}{\PYZsq{}}\PY{p}{,}\PY{n}{label}\PY{o}{=}\PY{l+s+s1}{\PYZsq{}}\PY{l+s+s1}{negative}\PY{l+s+s1}{\PYZsq{}}\PY{p}{)}
\PY{n}{ax}\PY{o}{.}\PY{n}{add\PYZus{}collection}\PY{p}{(}\PY{n}{sides}\PY{p}{)}
\PY{n}{ax}\PY{o}{.}\PY{n}{autoscale}\PY{p}{(}\PY{p}{)}
\PY{n}{ax}\PY{o}{.}\PY{n}{margins}\PY{p}{(}\PY{l+m+mf}{0.1}\PY{p}{)}
\PY{n}{ax}\PY{o}{.}\PY{n}{legend}\PY{p}{(}\PY{p}{)}
\end{Verbatim}
\end{tcolorbox}

            \begin{tcolorbox}[breakable, size=fbox, boxrule=.5pt, pad at break*=1mm, opacityfill=0]
\prompt{Out}{outcolor}{2}{\boxspacing}
\begin{Verbatim}[commandchars=\\\{\}]
<matplotlib.legend.Legend at 0x7f83844f8f10>
\end{Verbatim}
\end{tcolorbox}
        
    \begin{center}
    \adjustimage{max size={0.9\linewidth}{0.9\paperheight}}{Tarea_4_IA_files/Tarea_4_IA_7_1.png}
    \end{center}
    { \hspace*{\fill} \\}
    
    A continuación obtendremos los conjuntos \(E\) y \(D\), cuidando que los
vértices del rectángulo anterior pertenezcan a \(E\) y tanto \(E\) como
\(D\) cumplan las propiedades que numeramos anteriormente.

    \begin{tcolorbox}[breakable, size=fbox, boxrule=1pt, pad at break*=1mm,colback=cellbackground, colframe=cellborder]
\prompt{In}{incolor}{3}{\boxspacing}
\begin{Verbatim}[commandchars=\\\{\}]
\PY{k+kn}{import} \PY{n+nn}{pandas} \PY{k}{as} \PY{n+nn}{pd}
\PY{k+kn}{from} \PY{n+nn}{sklearn}\PY{n+nn}{.}\PY{n+nn}{model\PYZus{}selection} \PY{k+kn}{import} \PY{n}{train\PYZus{}test\PYZus{}split}

\PY{c+c1}{\PYZsh{} Samples of positive and negative class of the form (X1,X2,Y)}
\PY{n}{P\PYZus{}random\PYZus{}class}\PY{o}{=}\PY{n}{np}\PY{o}{.}\PY{n}{column\PYZus{}stack}\PY{p}{(}\PY{p}{[}\PY{n}{P\PYZus{}random}\PY{p}{,}\PY{n}{np}\PY{o}{.}\PY{n}{ones}\PY{p}{(}\PY{n+nb}{len}\PY{p}{(}\PY{n}{P\PYZus{}random}\PY{p}{)}\PY{p}{)}\PY{p}{]}\PY{p}{)}
\PY{n}{N\PYZus{}class}\PY{o}{=}\PY{n}{np}\PY{o}{.}\PY{n}{column\PYZus{}stack}\PY{p}{(}\PY{p}{[}\PY{n}{N}\PY{p}{,}\PY{n}{np}\PY{o}{.}\PY{n}{zeros}\PY{p}{(}\PY{n+nb}{len}\PY{p}{(}\PY{n}{N}\PY{p}{[}\PY{p}{:}\PY{p}{,}\PY{l+m+mi}{0}\PY{p}{]}\PY{p}{)}\PY{p}{)}\PY{p}{]}\PY{p}{)}
\PY{n}{df\PYZus{}P\PYZus{}random}\PY{o}{=}\PY{n}{pd}\PY{o}{.}\PY{n}{DataFrame}\PY{p}{(}\PY{n}{P\PYZus{}random\PYZus{}class}\PY{p}{,}\PY{n}{columns}\PY{o}{=}\PY{p}{[}\PY{l+s+s1}{\PYZsq{}}\PY{l+s+s1}{X1}\PY{l+s+s1}{\PYZsq{}}\PY{p}{,}\PY{l+s+s1}{\PYZsq{}}\PY{l+s+s1}{X2}\PY{l+s+s1}{\PYZsq{}}\PY{p}{,}\PY{l+s+s1}{\PYZsq{}}\PY{l+s+s1}{Y}\PY{l+s+s1}{\PYZsq{}}\PY{p}{]}\PY{p}{)}
\PY{n}{df\PYZus{}N}\PY{o}{=}\PY{n}{pd}\PY{o}{.}\PY{n}{DataFrame}\PY{p}{(}\PY{n}{N\PYZus{}class}\PY{p}{,}\PY{n}{columns}\PY{o}{=}\PY{p}{[}\PY{l+s+s1}{\PYZsq{}}\PY{l+s+s1}{X1}\PY{l+s+s1}{\PYZsq{}}\PY{p}{,}\PY{l+s+s1}{\PYZsq{}}\PY{l+s+s1}{X2}\PY{l+s+s1}{\PYZsq{}}\PY{p}{,}\PY{l+s+s1}{\PYZsq{}}\PY{l+s+s1}{Y}\PY{l+s+s1}{\PYZsq{}}\PY{p}{]}\PY{p}{)}
\PY{c+c1}{\PYZsh{} All sample with shuffle}
\PY{n}{df\PYZus{}Sample}\PY{o}{=}\PY{n}{pd}\PY{o}{.}\PY{n}{concat}\PY{p}{(}\PY{p}{[}\PY{n}{df\PYZus{}P\PYZus{}random}\PY{p}{,}\PY{n}{df\PYZus{}N}\PY{p}{]}\PY{p}{)}
\PY{n}{df\PYZus{}Sample}\PY{o}{=}\PY{n}{df\PYZus{}Sample}\PY{o}{.}\PY{n}{sample}\PY{p}{(}\PY{n}{frac}\PY{o}{=}\PY{l+m+mi}{1}\PY{p}{,}\PY{n}{random\PYZus{}state}\PY{o}{=}\PY{n}{SEED}\PY{p}{)}
\PY{c+c1}{\PYZsh{} Get train and test sets}
\PY{n}{X\PYZus{}train}\PY{p}{,} \PY{n}{X\PYZus{}test}\PY{p}{,}\PY{n}{y\PYZus{}train}\PY{p}{,} \PY{n}{y\PYZus{}test}\PY{o}{=}\PY{n}{train\PYZus{}test\PYZus{}split}\PY{p}{(}\PY{n}{df\PYZus{}Sample}\PY{p}{[}\PY{p}{[}\PY{l+s+s1}{\PYZsq{}}\PY{l+s+s1}{X1}\PY{l+s+s1}{\PYZsq{}}\PY{p}{,}\PY{l+s+s1}{\PYZsq{}}\PY{l+s+s1}{X2}\PY{l+s+s1}{\PYZsq{}}\PY{p}{]}\PY{p}{]}\PY{o}{.}\PY{n}{to\PYZus{}numpy}\PY{p}{(}\PY{p}{)}\PY{p}{,}\PY{n}{df\PYZus{}Sample}\PY{p}{[}\PY{l+s+s1}{\PYZsq{}}\PY{l+s+s1}{Y}\PY{l+s+s1}{\PYZsq{}}\PY{p}{]}\PY{o}{.}\PY{n}{to\PYZus{}numpy}\PY{p}{(}\PY{p}{)}\PY{p}{,}\PY{n}{test\PYZus{}size}\PY{o}{=}\PY{l+m+mi}{5}\PY{o}{/}\PY{l+m+mi}{16}\PY{p}{,}\PY{n}{random\PYZus{}state}\PY{o}{=}\PY{n}{SEED}\PY{p}{)}
\PY{c+c1}{\PYZsh{} Add vertices}
\PY{n}{X\PYZus{}train}\PY{o}{=}\PY{n}{np}\PY{o}{.}\PY{n}{vstack}\PY{p}{(}\PY{p}{[}\PY{n}{X\PYZus{}train}\PY{p}{,}\PY{n}{V}\PY{p}{]}\PY{p}{)}
\PY{n}{y\PYZus{}train}\PY{o}{=}\PY{n}{np}\PY{o}{.}\PY{n}{append}\PY{p}{(}\PY{n}{y\PYZus{}train}\PY{p}{,}\PY{n}{np}\PY{o}{.}\PY{n}{ones}\PY{p}{(}\PY{n}{V}\PY{o}{.}\PY{n}{shape}\PY{p}{[}\PY{l+m+mi}{0}\PY{p}{]}\PY{p}{)}\PY{p}{)}
\PY{c+c1}{\PYZsh{} Save data}
\PY{n}{df\PYZus{}train}\PY{o}{=}\PY{n}{pd}\PY{o}{.}\PY{n}{DataFrame}\PY{p}{(}\PY{n}{np}\PY{o}{.}\PY{n}{column\PYZus{}stack}\PY{p}{(}\PY{p}{[}\PY{n}{X\PYZus{}train}\PY{p}{,}\PY{n}{y\PYZus{}train}\PY{p}{]}\PY{p}{)}\PY{p}{,}\PY{n}{columns}\PY{o}{=}\PY{p}{[}\PY{l+s+s1}{\PYZsq{}}\PY{l+s+s1}{X1}\PY{l+s+s1}{\PYZsq{}}\PY{p}{,}\PY{l+s+s1}{\PYZsq{}}\PY{l+s+s1}{X2}\PY{l+s+s1}{\PYZsq{}}\PY{p}{,}\PY{l+s+s1}{\PYZsq{}}\PY{l+s+s1}{Y}\PY{l+s+s1}{\PYZsq{}}\PY{p}{]}\PY{p}{)}
\PY{n}{df\PYZus{}train}\PY{o}{=}\PY{n}{df\PYZus{}train}\PY{o}{.}\PY{n}{sample}\PY{p}{(}\PY{n}{frac}\PY{o}{=}\PY{l+m+mi}{1}\PY{p}{,}\PY{n}{random\PYZus{}state}\PY{o}{=}\PY{n}{SEED}\PY{p}{)}
\PY{n}{df\PYZus{}train}\PY{o}{.}\PY{n}{to\PYZus{}csv}\PY{p}{(}\PY{l+s+s1}{\PYZsq{}}\PY{l+s+s1}{train.csv}\PY{l+s+s1}{\PYZsq{}}\PY{p}{)}
\PY{n}{df\PYZus{}test}\PY{o}{=}\PY{n}{pd}\PY{o}{.}\PY{n}{DataFrame}\PY{p}{(}\PY{n}{np}\PY{o}{.}\PY{n}{column\PYZus{}stack}\PY{p}{(}\PY{p}{[}\PY{n}{X\PYZus{}test}\PY{p}{,}\PY{n}{y\PYZus{}test}\PY{p}{]}\PY{p}{)}\PY{p}{,}\PY{n}{columns}\PY{o}{=}\PY{p}{[}\PY{l+s+s1}{\PYZsq{}}\PY{l+s+s1}{X1}\PY{l+s+s1}{\PYZsq{}}\PY{p}{,}\PY{l+s+s1}{\PYZsq{}}\PY{l+s+s1}{X2}\PY{l+s+s1}{\PYZsq{}}\PY{p}{,}\PY{l+s+s1}{\PYZsq{}}\PY{l+s+s1}{Y}\PY{l+s+s1}{\PYZsq{}}\PY{p}{]}\PY{p}{)}
\PY{n}{df\PYZus{}test}\PY{o}{.}\PY{n}{to\PYZus{}csv}\PY{p}{(}\PY{l+s+s1}{\PYZsq{}}\PY{l+s+s1}{test.csv}\PY{l+s+s1}{\PYZsq{}}\PY{p}{)}
\end{Verbatim}
\end{tcolorbox}

    \hypertarget{muxe9todo-1}{%
\subsection{Método 1}\label{muxe9todo-1}}

    A continuación realizaremos el entrenamiento y la validación del
perceptrón multicapas para hacer la clasificación binaria del conjunto
de datos anterior con nuestra propia implementación del algoritmo
\emph{backpropagation}. La implementación realizada se encuentra en el
archivo \emph{perceptron.py} en la clase \emph{perceptron} (Ver códigos
anexos a la tarea y README.md).

    \hypertarget{introducciuxf3n}{%
\subsubsection{Introducción}\label{introducciuxf3n}}

    Como las muestras son de la forma \((X_{1i},X_{2i},Y_i)\) entonces
tendremos dos neuronas en la capa de entrada y una en la capa de salida.

La estructura sugerida para realizar la clasificación es utilizar un
capa oculta con cuatro neuronas. Entonces tenemos 7 neuronas en total en
el perceptrón que las etiquetaremos de la siguiente manera
\(n_0,n_1,\dots,n_6\).

Sean \(l_0,l_1,l_2\) las capas que tiene este perceptrón, \(l_0\) es la
capa de entrada, \(l_1\) la capa oculta y \(l_2\) la de salida.
Enumeramos las neuronas de forma que \begin{align*}
    l_0=&\{n_0,n_1\}\\ 
    l_1=&\{n_2,n_3,n_4,n_5\}\\
    l_2=&\{n_6\}.
\end{align*}

    Además, consideramos los pesos asociados a las conexiones de las
neuronas como sigue \begin{equation*}
    w_{ij}\text{ es el peso asociado a la conexión de la neurona }n_i\text{ con la neurona }n_j.
\end{equation*}

Con esto en mente, los pesos de las conexiones entre \(l_0\) y \(l_1\)
los codificamos en la siguiente matriz
\begin{equation*}W_{l_0l_1}=\begin{pmatrix}w_{02}&w_{03}&w_{04}&w_{05}\\ w_{12} & w_{13} & w_{14} & w_{15} \end{pmatrix}
\end{equation*}

    De manera similar, los pesos de las conexiones entre \(l_1\) y \(l_2\)
se codificarán así
\begin{equation*}W_{l_1l_2}=\begin{pmatrix}w_{26}\\ w_{36} \\ w_{46} \\ w_{56}\end{pmatrix}\end{equation*}

    Recordamos que el término \emph{bias} solo existe para las neuronas de
la capa escondida y de salida. Si \(b_{i}\) el peso asociado al bias de
la neurona \(n_i\) entonces denotamos al vector que captura estos pesos
de la siguiente forma
\begin{equation*}BIAS=(b_2,b_3,b_4,b_5,b_6)\end{equation*}

    \hypertarget{resultados}{%
\subsubsection{Resultados}\label{resultados}}

    En primer lugar, realizamos el entrenamiento del perceptrón propuesto.
Utilizaremos el conjunto \(E\) construido en la sección anterior. \(E\)
consta de \(70\) observaciones e incluye los vértices del rectángulo
\(\mathcal{R}\).

    \begin{tcolorbox}[breakable, size=fbox, boxrule=1pt, pad at break*=1mm,colback=cellbackground, colframe=cellborder]
\prompt{In}{incolor}{4}{\boxspacing}
\begin{Verbatim}[commandchars=\\\{\}]
\PY{k+kn}{import} \PY{n+nn}{sys}
\PY{n}{sys}\PY{o}{.}\PY{n}{path}\PY{o}{.}\PY{n}{append}\PY{p}{(}\PY{l+s+s1}{\PYZsq{}}\PY{l+s+s1}{..}\PY{l+s+s1}{\PYZsq{}}\PY{p}{)}
\PY{k+kn}{from} \PY{n+nn}{perceptron} \PY{k+kn}{import} \PY{n}{perceptron}
\PY{c+c1}{\PYZsh{} Train data}
\PY{n}{data\PYZus{}train}\PY{o}{=}\PY{n}{pd}\PY{o}{.}\PY{n}{read\PYZus{}csv}\PY{p}{(}\PY{l+s+s1}{\PYZsq{}}\PY{l+s+s1}{train.csv}\PY{l+s+s1}{\PYZsq{}}\PY{p}{)}
\PY{n}{X\PYZus{}train}\PY{o}{=}\PY{n}{data\PYZus{}train}\PY{p}{[}\PY{p}{[}\PY{l+s+s1}{\PYZsq{}}\PY{l+s+s1}{X1}\PY{l+s+s1}{\PYZsq{}}\PY{p}{,}\PY{l+s+s1}{\PYZsq{}}\PY{l+s+s1}{X2}\PY{l+s+s1}{\PYZsq{}}\PY{p}{]}\PY{p}{]}\PY{o}{.}\PY{n}{to\PYZus{}numpy}\PY{p}{(}\PY{p}{)}
\PY{n}{y\PYZus{}train}\PY{o}{=}\PY{n}{data\PYZus{}train}\PY{p}{[}\PY{l+s+s1}{\PYZsq{}}\PY{l+s+s1}{Y}\PY{l+s+s1}{\PYZsq{}}\PY{p}{]}\PY{o}{.}\PY{n}{to\PYZus{}numpy}\PY{p}{(}\PY{p}{)}
\PY{n}{layer\PYZus{}lenghts}\PY{o}{=}\PY{p}{[}\PY{l+m+mi}{2}\PY{p}{,}\PY{l+m+mi}{4}\PY{p}{,}\PY{l+m+mi}{1}\PY{p}{]}  \PY{c+c1}{\PYZsh{} We proposed this structure in class}
\PY{c+c1}{\PYZsh{} Network training}
\PY{n}{network\PYZus{}method\PYZus{}1}\PY{o}{=}\PY{n}{perceptron}\PY{p}{(}\PY{n}{X\PYZus{}train}\PY{p}{,}\PY{n}{y\PYZus{}train}\PY{p}{,}\PY{n}{layer\PYZus{}lenghts}\PY{p}{,}\PY{n}{MAX\PYZus{}EPOCHS}\PY{p}{)}
\PY{n}{network\PYZus{}method\PYZus{}1}\PY{o}{.}\PY{n}{train}\PY{p}{(}\PY{p}{)}
\end{Verbatim}
\end{tcolorbox}

    \begin{Verbatim}[commandchars=\\\{\}]
Numero de iteraciones para el conjunto de entrenamiento: 343
MSE para el entrenamiento: 0.0499
Numero de muestras bien clasificadas (Entrenamiento): 67
Numero de muestras mal clasificadas  (Entrenamiento): 3
Numero de muestras en el conjunto de entrenamiento: 70
Precision del entrenamiento: 0.9571
    \end{Verbatim}

    El criterio de paro fue que el MSE fuera mayor a 0.05, o bien que el
número de épocas fuera mayor a 1000.

    Calculamos tanto el MSE como la precisión del modelo con el conjunto de
validación \(V\), obteniendo lo siguiente

    \begin{tcolorbox}[breakable, size=fbox, boxrule=1pt, pad at break*=1mm,colback=cellbackground, colframe=cellborder]
\prompt{In}{incolor}{5}{\boxspacing}
\begin{Verbatim}[commandchars=\\\{\}]
\PY{n}{data\PYZus{}test}\PY{o}{=}\PY{n}{pd}\PY{o}{.}\PY{n}{read\PYZus{}csv}\PY{p}{(}\PY{l+s+s1}{\PYZsq{}}\PY{l+s+s1}{test.csv}\PY{l+s+s1}{\PYZsq{}}\PY{p}{)}
\PY{n}{X\PYZus{}test}\PY{o}{=}\PY{n}{data\PYZus{}test}\PY{p}{[}\PY{p}{[}\PY{l+s+s1}{\PYZsq{}}\PY{l+s+s1}{X1}\PY{l+s+s1}{\PYZsq{}}\PY{p}{,}\PY{l+s+s1}{\PYZsq{}}\PY{l+s+s1}{X2}\PY{l+s+s1}{\PYZsq{}}\PY{p}{]}\PY{p}{]}\PY{o}{.}\PY{n}{to\PYZus{}numpy}\PY{p}{(}\PY{p}{)}
\PY{n}{y\PYZus{}test}\PY{o}{=}\PY{n}{data\PYZus{}test}\PY{o}{.}\PY{n}{loc}\PY{p}{[}\PY{p}{:}\PY{p}{,}\PY{l+s+s1}{\PYZsq{}}\PY{l+s+s1}{Y}\PY{l+s+s1}{\PYZsq{}}\PY{p}{]}\PY{o}{.}\PY{n}{to\PYZus{}numpy}\PY{p}{(}\PY{p}{)}
\PY{n}{network\PYZus{}method\PYZus{}1}\PY{o}{.}\PY{n}{test}\PY{p}{(}\PY{n}{X\PYZus{}test}\PY{p}{,}\PY{n}{y\PYZus{}test}\PY{p}{)}
\end{Verbatim}
\end{tcolorbox}

    \begin{Verbatim}[commandchars=\\\{\}]
MSE para la validacion: 0.0685
Numero de muestras bien clasificadas (Validacion): 27
Numero de muestras mal clasificadas (Validacion): 3
Numero de muestras en el conjunto de validacion: 30
Precision en la validacion: 0.9000
    \end{Verbatim}

    Posteriormente, reportamos los parámetros y el tamaño del perceptrón.

    \begin{tcolorbox}[breakable, size=fbox, boxrule=1pt, pad at break*=1mm,colback=cellbackground, colframe=cellborder]
\prompt{In}{incolor}{6}{\boxspacing}
\begin{Verbatim}[commandchars=\\\{\}]
\PY{n}{network\PYZus{}method\PYZus{}1}\PY{o}{.}\PY{n}{report\PYZus{}parameters}\PY{p}{(}\PY{p}{)}
\end{Verbatim}
\end{tcolorbox}

    \begin{Verbatim}[commandchars=\\\{\}]
Los pesos de la capa inicial a la capa oculta son:
[[ 2.03977342  2.69337469 -3.31474544  0.01984903]
 [ 0.97194516 -5.6928559  -1.50456415  1.1714856 ]]


Los pesos de la capa oculta a la capa de salida son:
[[ 5.14749429]
 [-4.42417026]
 [ 0.93947153]
 [-5.07569751]]


El coeficiente asociado al bias para cada neurona de la capa oculta es:
[4.56981282 1.17506376 0.29703146 7.63657354]


El coeficiente asociado al bias para cada neurona de la capa de salida es:
[2.34941783]


Numero de parametros contando el bias: 17
    \end{Verbatim}

    Observamos que tenemos una precisión aceptable tanto para el conjunto de
entrenamiento \(E\) como el de validación \(V\) en \(343\) épocas. En el
entrenamiento el MSE si fue menor 0.05 esto no fue así con el conjunto
\(V\), aún así tenemos un buen resultado con una precisión de \(95\%\)
para el entrenamiento y de \(90\%\) para la validación.

    \hypertarget{visualizaciuxf3n}{%
\subsubsection{Visualización}\label{visualizaciuxf3n}}

    Cada neurona de la capa escondida y la de salida se puede ver como un
función que va de \(\mathbb{R}^2\) a \([0,1]\). Sea
\(f_i: \mathbb{R}^2\to [0,1]\) la función correspondiente a la neurona
\(n_i\).

    Para visualizar el comportamiento de cada neurona graficaremos estas
funciones. Obtenedremos una gráfica 3D por neurona, así también
graficaremos los puntos
\((X_{1i},X_{2i},\mathbf{1}_{\mathcal{R}}(X_{1i},X_{2i}))\)

    \begin{tcolorbox}[breakable, size=fbox, boxrule=1pt, pad at break*=1mm,colback=cellbackground, colframe=cellborder]
\prompt{In}{incolor}{7}{\boxspacing}
\begin{Verbatim}[commandchars=\\\{\}]
\PY{n}{x1}\PY{o}{=}\PY{n}{np}\PY{o}{.}\PY{n}{linspace}\PY{p}{(}\PY{o}{\PYZhy{}}\PY{l+m+mi}{3}\PY{p}{,}\PY{l+m+mi}{12}\PY{p}{,}\PY{l+m+mi}{50}\PY{p}{)}
\PY{n}{x2}\PY{o}{=}\PY{n}{np}\PY{o}{.}\PY{n}{linspace}\PY{p}{(}\PY{o}{\PYZhy{}}\PY{l+m+mi}{3}\PY{p}{,}\PY{l+m+mi}{12}\PY{p}{,}\PY{l+m+mi}{50}\PY{p}{)}
\PY{n}{X1}\PY{p}{,}\PY{n}{X2}\PY{o}{=}\PY{n}{np}\PY{o}{.}\PY{n}{meshgrid}\PY{p}{(}\PY{n}{x1}\PY{p}{,}\PY{n}{x2}\PY{p}{,}\PY{n}{indexing}\PY{o}{=}\PY{l+s+s1}{\PYZsq{}}\PY{l+s+s1}{ij}\PY{l+s+s1}{\PYZsq{}}\PY{p}{)}

\PY{c+c1}{\PYZsh{} Neuron functions}
\PY{k}{def} \PY{n+nf}{f2}\PY{p}{(}\PY{n}{x}\PY{p}{)}\PY{p}{:}
    \PY{n}{network\PYZus{}method\PYZus{}1}\PY{o}{.}\PY{n}{forward}\PY{p}{(}\PY{n}{x}\PY{p}{)}
    \PY{k}{return} \PY{n}{network\PYZus{}method\PYZus{}1}\PY{o}{.}\PY{n}{output}\PY{p}{[}\PY{l+m+mi}{2}\PY{p}{]}
\PY{k}{def} \PY{n+nf}{f3}\PY{p}{(}\PY{n}{x}\PY{p}{)}\PY{p}{:}
    \PY{n}{network\PYZus{}method\PYZus{}1}\PY{o}{.}\PY{n}{forward}\PY{p}{(}\PY{n}{x}\PY{p}{)}
    \PY{k}{return} \PY{n}{network\PYZus{}method\PYZus{}1}\PY{o}{.}\PY{n}{output}\PY{p}{[}\PY{l+m+mi}{3}\PY{p}{]}
\PY{k}{def} \PY{n+nf}{f4}\PY{p}{(}\PY{n}{x}\PY{p}{)}\PY{p}{:}
    \PY{n}{network\PYZus{}method\PYZus{}1}\PY{o}{.}\PY{n}{forward}\PY{p}{(}\PY{n}{x}\PY{p}{)}
    \PY{k}{return} \PY{n}{network\PYZus{}method\PYZus{}1}\PY{o}{.}\PY{n}{output}\PY{p}{[}\PY{l+m+mi}{4}\PY{p}{]}
\PY{k}{def} \PY{n+nf}{f5}\PY{p}{(}\PY{n}{x}\PY{p}{)}\PY{p}{:}
    \PY{n}{network\PYZus{}method\PYZus{}1}\PY{o}{.}\PY{n}{forward}\PY{p}{(}\PY{n}{x}\PY{p}{)}
    \PY{k}{return} \PY{n}{network\PYZus{}method\PYZus{}1}\PY{o}{.}\PY{n}{output}\PY{p}{[}\PY{l+m+mi}{5}\PY{p}{]}
\PY{k}{def} \PY{n+nf}{f6}\PY{p}{(}\PY{n}{x}\PY{p}{)}\PY{p}{:}
    \PY{n}{network\PYZus{}method\PYZus{}1}\PY{o}{.}\PY{n}{forward}\PY{p}{(}\PY{n}{x}\PY{p}{)}
    \PY{k}{return} \PY{n}{network\PYZus{}method\PYZus{}1}\PY{o}{.}\PY{n}{output}\PY{p}{[}\PY{l+m+mi}{6}\PY{p}{]}
\PY{c+c1}{\PYZsh{} Output of functions  in the grid}
\PY{n}{Z2}\PY{p}{,}\PY{n}{Z3}\PY{p}{,}\PY{n}{Z4}\PY{p}{,}\PY{n}{Z5}\PY{p}{,}\PY{n}{Z6}\PY{o}{=}\PY{p}{[}\PY{p}{]}\PY{p}{,}\PY{p}{[}\PY{p}{]}\PY{p}{,}\PY{p}{[}\PY{p}{]}\PY{p}{,}\PY{p}{[}\PY{p}{]}\PY{p}{,}\PY{p}{[}\PY{p}{]}
\PY{k}{for} \PY{n}{i} \PY{o+ow}{in} \PY{n+nb}{range}\PY{p}{(}\PY{n+nb}{len}\PY{p}{(}\PY{n}{x1}\PY{p}{)}\PY{p}{)}\PY{p}{:}
    \PY{k}{for} \PY{n}{j} \PY{o+ow}{in} \PY{n+nb}{range}\PY{p}{(}\PY{n+nb}{len}\PY{p}{(}\PY{n}{x2}\PY{p}{)}\PY{p}{)}\PY{p}{:}
        \PY{n}{Z2}\PY{o}{.}\PY{n}{append}\PY{p}{(}\PY{n}{f2}\PY{p}{(}\PY{n}{np}\PY{o}{.}\PY{n}{array}\PY{p}{(}\PY{p}{[}\PY{n}{X1}\PY{p}{[}\PY{n}{i}\PY{p}{,}\PY{n}{j}\PY{p}{]}\PY{p}{,}\PY{n}{X2}\PY{p}{[}\PY{n}{i}\PY{p}{,}\PY{n}{j}\PY{p}{]}\PY{p}{]}\PY{p}{)}\PY{p}{)}\PY{p}{)}
        \PY{n}{Z3}\PY{o}{.}\PY{n}{append}\PY{p}{(}\PY{n}{f3}\PY{p}{(}\PY{n}{np}\PY{o}{.}\PY{n}{array}\PY{p}{(}\PY{p}{[}\PY{n}{X1}\PY{p}{[}\PY{n}{i}\PY{p}{,}\PY{n}{j}\PY{p}{]}\PY{p}{,}\PY{n}{X2}\PY{p}{[}\PY{n}{i}\PY{p}{,}\PY{n}{j}\PY{p}{]}\PY{p}{]}\PY{p}{)}\PY{p}{)}\PY{p}{)}
        \PY{n}{Z4}\PY{o}{.}\PY{n}{append}\PY{p}{(}\PY{n}{f4}\PY{p}{(}\PY{n}{np}\PY{o}{.}\PY{n}{array}\PY{p}{(}\PY{p}{[}\PY{n}{X1}\PY{p}{[}\PY{n}{i}\PY{p}{,}\PY{n}{j}\PY{p}{]}\PY{p}{,}\PY{n}{X2}\PY{p}{[}\PY{n}{i}\PY{p}{,}\PY{n}{j}\PY{p}{]}\PY{p}{]}\PY{p}{)}\PY{p}{)}\PY{p}{)}
        \PY{n}{Z5}\PY{o}{.}\PY{n}{append}\PY{p}{(}\PY{n}{f5}\PY{p}{(}\PY{n}{np}\PY{o}{.}\PY{n}{array}\PY{p}{(}\PY{p}{[}\PY{n}{X1}\PY{p}{[}\PY{n}{i}\PY{p}{,}\PY{n}{j}\PY{p}{]}\PY{p}{,}\PY{n}{X2}\PY{p}{[}\PY{n}{i}\PY{p}{,}\PY{n}{j}\PY{p}{]}\PY{p}{]}\PY{p}{)}\PY{p}{)}\PY{p}{)}
        \PY{n}{Z6}\PY{o}{.}\PY{n}{append}\PY{p}{(}\PY{n}{f6}\PY{p}{(}\PY{n}{np}\PY{o}{.}\PY{n}{array}\PY{p}{(}\PY{p}{[}\PY{n}{X1}\PY{p}{[}\PY{n}{i}\PY{p}{,}\PY{n}{j}\PY{p}{]}\PY{p}{,}\PY{n}{X2}\PY{p}{[}\PY{n}{i}\PY{p}{,}\PY{n}{j}\PY{p}{]}\PY{p}{]}\PY{p}{)}\PY{p}{)}\PY{p}{)}
\PY{n}{Z2}\PY{o}{=}\PY{n}{np}\PY{o}{.}\PY{n}{reshape}\PY{p}{(}\PY{n}{Z2}\PY{p}{,}\PY{p}{(}\PY{n+nb}{len}\PY{p}{(}\PY{n}{x1}\PY{p}{)}\PY{p}{,}\PY{n+nb}{len}\PY{p}{(}\PY{n}{x2}\PY{p}{)}\PY{p}{)}\PY{p}{)}
\PY{n}{Z3}\PY{o}{=}\PY{n}{np}\PY{o}{.}\PY{n}{reshape}\PY{p}{(}\PY{n}{Z3}\PY{p}{,}\PY{p}{(}\PY{n+nb}{len}\PY{p}{(}\PY{n}{x1}\PY{p}{)}\PY{p}{,}\PY{n+nb}{len}\PY{p}{(}\PY{n}{x2}\PY{p}{)}\PY{p}{)}\PY{p}{)}
\PY{n}{Z4}\PY{o}{=}\PY{n}{np}\PY{o}{.}\PY{n}{reshape}\PY{p}{(}\PY{n}{Z4}\PY{p}{,}\PY{p}{(}\PY{n+nb}{len}\PY{p}{(}\PY{n}{x1}\PY{p}{)}\PY{p}{,}\PY{n+nb}{len}\PY{p}{(}\PY{n}{x2}\PY{p}{)}\PY{p}{)}\PY{p}{)}
\PY{n}{Z5}\PY{o}{=}\PY{n}{np}\PY{o}{.}\PY{n}{reshape}\PY{p}{(}\PY{n}{Z5}\PY{p}{,}\PY{p}{(}\PY{n+nb}{len}\PY{p}{(}\PY{n}{x1}\PY{p}{)}\PY{p}{,}\PY{n+nb}{len}\PY{p}{(}\PY{n}{x2}\PY{p}{)}\PY{p}{)}\PY{p}{)}
\PY{n}{Z6}\PY{o}{=}\PY{n}{np}\PY{o}{.}\PY{n}{reshape}\PY{p}{(}\PY{n}{Z6}\PY{p}{,}\PY{p}{(}\PY{n+nb}{len}\PY{p}{(}\PY{n}{x1}\PY{p}{)}\PY{p}{,}\PY{n+nb}{len}\PY{p}{(}\PY{n}{x2}\PY{p}{)}\PY{p}{)}\PY{p}{)}
\end{Verbatim}
\end{tcolorbox}

    A continuación visualizaremos la gráfica de las neuronas en la capa
escondida, es decir, las neuronas \(n_2,n_3,n_4\) y \(n_5\).

    La siguiente es la gráfica de la función \(f_2\), podemos observar que
la función \(f_2\) solo pudo identificar una parte de la clase negativa.
Como discutimos en clase el perceptrón de Rosenblatt necesita que el
conjunto sea linealmente separable, luego la frontera de la región de
decisión de \(n_2\) en el plano \((X_1,X_2)\) es una recta, en este caso
correspondiente la recta que pasa por \((-2,0)\) y \((2,-3)\),
aproximadamente.

    \begin{tcolorbox}[breakable, size=fbox, boxrule=1pt, pad at break*=1mm,colback=cellbackground, colframe=cellborder]
\prompt{In}{incolor}{8}{\boxspacing}
\begin{Verbatim}[commandchars=\\\{\}]
\PY{c+c1}{\PYZsh{} Create figure }
\PY{n}{fig} \PY{o}{=} \PY{n}{plt}\PY{o}{.}\PY{n}{figure}\PY{p}{(}\PY{n}{figsize}\PY{o}{=}\PY{p}{(}\PY{l+m+mi}{8}\PY{p}{,}\PY{l+m+mi}{8}\PY{p}{)}\PY{p}{)}

\PY{l+s+sd}{\PYZsq{}\PYZsq{}\PYZsq{} }
\PY{l+s+sd}{Plot Neuron N2}
\PY{l+s+sd}{\PYZsq{}\PYZsq{}\PYZsq{}}
\PY{n}{ax\PYZus{}n2} \PY{o}{=} \PY{n}{fig}\PY{o}{.}\PY{n}{add\PYZus{}subplot}\PY{p}{(}\PY{l+m+mi}{111}\PY{p}{,} \PY{n}{projection}\PY{o}{=}\PY{l+s+s1}{\PYZsq{}}\PY{l+s+s1}{3d}\PY{l+s+s1}{\PYZsq{}}\PY{p}{)}
\PY{n}{plot\PYZus{}n2\PYZus{}surface}\PY{o}{=}\PY{n}{ax\PYZus{}n2}\PY{o}{.}\PY{n}{plot\PYZus{}surface}\PY{p}{(}\PY{n}{X1}\PY{p}{,}\PY{n}{X2}\PY{p}{,}\PY{n}{Z2}\PY{p}{,}\PY{n}{cmap}\PY{o}{=}\PY{l+s+s1}{\PYZsq{}}\PY{l+s+s1}{viridis}\PY{l+s+s1}{\PYZsq{}}\PY{p}{)}
\PY{n}{plot\PYZus{}positive}\PY{o}{=}\PY{n}{ax\PYZus{}n2}\PY{o}{.}\PY{n}{scatter3D}\PY{p}{(}\PY{n}{P}\PY{p}{[}\PY{p}{:}\PY{p}{,}\PY{l+m+mi}{0}\PY{p}{]}\PY{p}{,}\PY{n}{P}\PY{p}{[}\PY{p}{:}\PY{p}{,}\PY{l+m+mi}{1}\PY{p}{]}\PY{p}{,}\PY{n}{np}\PY{o}{.}\PY{n}{ones}\PY{p}{(}\PY{n+nb}{len}\PY{p}{(}\PY{n}{P}\PY{p}{[}\PY{p}{:}\PY{p}{,}\PY{l+m+mi}{0}\PY{p}{]}\PY{p}{)}\PY{p}{)}\PY{p}{,}\PY{n}{color}\PY{o}{=}\PY{l+s+s1}{\PYZsq{}}\PY{l+s+s1}{red}\PY{l+s+s1}{\PYZsq{}}\PY{p}{,}\PY{n}{label}\PY{o}{=}\PY{l+s+s1}{\PYZsq{}}\PY{l+s+s1}{positive}\PY{l+s+s1}{\PYZsq{}}\PY{p}{,}\PY{n}{marker}\PY{o}{=}\PY{l+s+s1}{\PYZsq{}}\PY{l+s+s1}{\PYZca{}}\PY{l+s+s1}{\PYZsq{}}\PY{p}{)}
\PY{n}{plot\PYZus{}negative}\PY{o}{=}\PY{n}{ax\PYZus{}n2}\PY{o}{.}\PY{n}{scatter3D}\PY{p}{(}\PY{n}{N}\PY{p}{[}\PY{p}{:}\PY{p}{,}\PY{l+m+mi}{0}\PY{p}{]}\PY{p}{,}\PY{n}{N}\PY{p}{[}\PY{p}{:}\PY{p}{,}\PY{l+m+mi}{1}\PY{p}{]}\PY{p}{,}\PY{n}{np}\PY{o}{.}\PY{n}{zeros}\PY{p}{(}\PY{n+nb}{len}\PY{p}{(}\PY{n}{N}\PY{p}{[}\PY{p}{:}\PY{p}{,}\PY{l+m+mi}{0}\PY{p}{]}\PY{p}{)}\PY{p}{)}\PY{p}{,}\PY{n}{color}\PY{o}{=}\PY{l+s+s1}{\PYZsq{}}\PY{l+s+s1}{b}\PY{l+s+s1}{\PYZsq{}}\PY{p}{,}\PY{n}{label}\PY{o}{=}\PY{l+s+s1}{\PYZsq{}}\PY{l+s+s1}{negative}\PY{l+s+s1}{\PYZsq{}}\PY{p}{,}\PY{n}{marker}\PY{o}{=}\PY{l+s+s1}{\PYZsq{}}\PY{l+s+s1}{*}\PY{l+s+s1}{\PYZsq{}}\PY{p}{)}
\PY{n}{ax\PYZus{}n2}\PY{o}{.}\PY{n}{view\PYZus{}init}\PY{p}{(}\PY{n}{elev}\PY{o}{=}\PY{l+m+mi}{20}\PY{p}{,} \PY{n}{azim}\PY{o}{=}\PY{l+m+mi}{100}\PY{p}{)}
\PY{n}{ax\PYZus{}n2}\PY{o}{.}\PY{n}{dist}\PY{o}{=}\PY{l+m+mi}{11}
\PY{n}{ax\PYZus{}n2}\PY{o}{.}\PY{n}{legend}\PY{p}{(}\PY{p}{)}
\PY{n}{ax\PYZus{}n2}\PY{o}{.}\PY{n}{set\PYZus{}title}\PY{p}{(}\PY{l+s+sa}{r}\PY{l+s+s1}{\PYZsq{}}\PY{l+s+s1}{Neuron \PYZdl{}n\PYZus{}2\PYZdl{}}\PY{l+s+s1}{\PYZsq{}}\PY{p}{)}
\PY{n}{ax\PYZus{}n2}\PY{o}{.}\PY{n}{set\PYZus{}xlabel}\PY{p}{(}\PY{l+s+s1}{\PYZsq{}}\PY{l+s+s1}{X1}\PY{l+s+s1}{\PYZsq{}}\PY{p}{)}
\PY{n}{ax\PYZus{}n2}\PY{o}{.}\PY{n}{set\PYZus{}ylabel}\PY{p}{(}\PY{l+s+s1}{\PYZsq{}}\PY{l+s+s1}{X2}\PY{l+s+s1}{\PYZsq{}}\PY{p}{)}
\PY{n}{ax\PYZus{}n2}\PY{o}{.}\PY{n}{set\PYZus{}zlabel}\PY{p}{(}\PY{l+s+s1}{\PYZsq{}}\PY{l+s+s1}{Z}\PY{l+s+s1}{\PYZsq{}}\PY{p}{)}
\PY{c+c1}{\PYZsh{} Add colorbar}
\PY{n}{cbar} \PY{o}{=} \PY{n}{fig}\PY{o}{.}\PY{n}{colorbar}\PY{p}{(}\PY{n}{plot\PYZus{}n2\PYZus{}surface}\PY{p}{,}\PY{n}{ax}\PY{o}{=}\PY{n}{ax\PYZus{}n2}\PY{p}{,} \PY{n}{shrink}\PY{o}{=}\PY{l+m+mf}{0.6}\PY{p}{)}
\PY{n}{cbar}\PY{o}{.}\PY{n}{set\PYZus{}ticks}\PY{p}{(}\PY{p}{[}\PY{l+m+mi}{0}\PY{p}{,} \PY{l+m+mf}{0.25}\PY{p}{,} \PY{l+m+mf}{0.5}\PY{p}{,} \PY{l+m+mf}{0.75}\PY{p}{,} \PY{l+m+mi}{1}\PY{p}{]}\PY{p}{)}
\PY{n}{cbar}\PY{o}{.}\PY{n}{set\PYZus{}ticklabels}\PY{p}{(}\PY{p}{[}\PY{l+s+s1}{\PYZsq{}}\PY{l+s+s1}{0}\PY{l+s+s1}{\PYZsq{}}\PY{p}{,} \PY{l+s+s1}{\PYZsq{}}\PY{l+s+s1}{0.25}\PY{l+s+s1}{\PYZsq{}}\PY{p}{,} \PY{l+s+s1}{\PYZsq{}}\PY{l+s+s1}{0.5}\PY{l+s+s1}{\PYZsq{}}\PY{p}{,} \PY{l+s+s1}{\PYZsq{}}\PY{l+s+s1}{0.75}\PY{l+s+s1}{\PYZsq{}}\PY{p}{,} \PY{l+s+s1}{\PYZsq{}}\PY{l+s+s1}{1}\PY{l+s+s1}{\PYZsq{}}\PY{p}{]}\PY{p}{)}
\end{Verbatim}
\end{tcolorbox}

    \begin{center}
    \adjustimage{max size={0.9\linewidth}{0.9\paperheight}}{Tarea_4_IA_files/Tarea_4_IA_32_0.png}
    \end{center}
    { \hspace*{\fill} \\}
    
    Hacemos lo propio con la neurona \(n_3\). Aquí observamos que la región
de decisión en el plano \((X_1,X_2)\) es aproximadamente la recta que
pasa por \((1,0)\) y \((12,6)\).

    \begin{tcolorbox}[breakable, size=fbox, boxrule=1pt, pad at break*=1mm,colback=cellbackground, colframe=cellborder]
\prompt{In}{incolor}{9}{\boxspacing}
\begin{Verbatim}[commandchars=\\\{\}]
\PY{c+c1}{\PYZsh{} Create figure }
\PY{n}{fig} \PY{o}{=} \PY{n}{plt}\PY{o}{.}\PY{n}{figure}\PY{p}{(}\PY{n}{figsize}\PY{o}{=}\PY{p}{(}\PY{l+m+mi}{8}\PY{p}{,}\PY{l+m+mi}{8}\PY{p}{)}\PY{p}{)}

\PY{l+s+sd}{\PYZsq{}\PYZsq{}\PYZsq{} }
\PY{l+s+sd}{Plot Neuron n3}
\PY{l+s+sd}{\PYZsq{}\PYZsq{}\PYZsq{}}
\PY{n}{ax\PYZus{}n3} \PY{o}{=} \PY{n}{fig}\PY{o}{.}\PY{n}{add\PYZus{}subplot}\PY{p}{(}\PY{l+m+mi}{111}\PY{p}{,} \PY{n}{projection}\PY{o}{=}\PY{l+s+s1}{\PYZsq{}}\PY{l+s+s1}{3d}\PY{l+s+s1}{\PYZsq{}}\PY{p}{)}
\PY{n}{plot\PYZus{}n3\PYZus{}surface}\PY{o}{=}\PY{n}{ax\PYZus{}n3}\PY{o}{.}\PY{n}{plot\PYZus{}surface}\PY{p}{(}\PY{n}{X1}\PY{p}{,}\PY{n}{X2}\PY{p}{,}\PY{n}{Z3}\PY{p}{,}\PY{n}{cmap}\PY{o}{=}\PY{l+s+s1}{\PYZsq{}}\PY{l+s+s1}{viridis}\PY{l+s+s1}{\PYZsq{}}\PY{p}{)}
\PY{n}{plot\PYZus{}positive}\PY{o}{=}\PY{n}{ax\PYZus{}n3}\PY{o}{.}\PY{n}{scatter3D}\PY{p}{(}\PY{n}{P}\PY{p}{[}\PY{p}{:}\PY{p}{,}\PY{l+m+mi}{0}\PY{p}{]}\PY{p}{,}\PY{n}{P}\PY{p}{[}\PY{p}{:}\PY{p}{,}\PY{l+m+mi}{1}\PY{p}{]}\PY{p}{,}\PY{n}{np}\PY{o}{.}\PY{n}{ones}\PY{p}{(}\PY{n+nb}{len}\PY{p}{(}\PY{n}{P}\PY{p}{[}\PY{p}{:}\PY{p}{,}\PY{l+m+mi}{0}\PY{p}{]}\PY{p}{)}\PY{p}{)}\PY{p}{,}\PY{n}{color}\PY{o}{=}\PY{l+s+s1}{\PYZsq{}}\PY{l+s+s1}{red}\PY{l+s+s1}{\PYZsq{}}\PY{p}{,}\PY{n}{label}\PY{o}{=}\PY{l+s+s1}{\PYZsq{}}\PY{l+s+s1}{positive}\PY{l+s+s1}{\PYZsq{}}\PY{p}{,}\PY{n}{marker}\PY{o}{=}\PY{l+s+s1}{\PYZsq{}}\PY{l+s+s1}{\PYZca{}}\PY{l+s+s1}{\PYZsq{}}\PY{p}{)}
\PY{n}{plot\PYZus{}negative}\PY{o}{=}\PY{n}{ax\PYZus{}n3}\PY{o}{.}\PY{n}{scatter3D}\PY{p}{(}\PY{n}{N}\PY{p}{[}\PY{p}{:}\PY{p}{,}\PY{l+m+mi}{0}\PY{p}{]}\PY{p}{,}\PY{n}{N}\PY{p}{[}\PY{p}{:}\PY{p}{,}\PY{l+m+mi}{1}\PY{p}{]}\PY{p}{,}\PY{n}{np}\PY{o}{.}\PY{n}{zeros}\PY{p}{(}\PY{n+nb}{len}\PY{p}{(}\PY{n}{N}\PY{p}{[}\PY{p}{:}\PY{p}{,}\PY{l+m+mi}{0}\PY{p}{]}\PY{p}{)}\PY{p}{)}\PY{p}{,}\PY{n}{color}\PY{o}{=}\PY{l+s+s1}{\PYZsq{}}\PY{l+s+s1}{b}\PY{l+s+s1}{\PYZsq{}}\PY{p}{,}\PY{n}{label}\PY{o}{=}\PY{l+s+s1}{\PYZsq{}}\PY{l+s+s1}{negative}\PY{l+s+s1}{\PYZsq{}}\PY{p}{,}\PY{n}{marker}\PY{o}{=}\PY{l+s+s1}{\PYZsq{}}\PY{l+s+s1}{*}\PY{l+s+s1}{\PYZsq{}}\PY{p}{)}
\PY{n}{ax\PYZus{}n3}\PY{o}{.}\PY{n}{view\PYZus{}init}\PY{p}{(}\PY{n}{elev}\PY{o}{=}\PY{l+m+mi}{20}\PY{p}{,} \PY{n}{azim}\PY{o}{=}\PY{l+m+mi}{200}\PY{p}{)}
\PY{n}{ax\PYZus{}n3}\PY{o}{.}\PY{n}{dist}\PY{o}{=}\PY{l+m+mi}{11}
\PY{n}{ax\PYZus{}n3}\PY{o}{.}\PY{n}{legend}\PY{p}{(}\PY{p}{)}
\PY{n}{ax\PYZus{}n3}\PY{o}{.}\PY{n}{set\PYZus{}title}\PY{p}{(}\PY{l+s+sa}{r}\PY{l+s+s1}{\PYZsq{}}\PY{l+s+s1}{Neuron \PYZdl{}n\PYZus{}3\PYZdl{}}\PY{l+s+s1}{\PYZsq{}}\PY{p}{)}
\PY{n}{ax\PYZus{}n3}\PY{o}{.}\PY{n}{set\PYZus{}xlabel}\PY{p}{(}\PY{l+s+s1}{\PYZsq{}}\PY{l+s+s1}{X1}\PY{l+s+s1}{\PYZsq{}}\PY{p}{)}
\PY{n}{ax\PYZus{}n3}\PY{o}{.}\PY{n}{set\PYZus{}ylabel}\PY{p}{(}\PY{l+s+s1}{\PYZsq{}}\PY{l+s+s1}{X2}\PY{l+s+s1}{\PYZsq{}}\PY{p}{)}
\PY{n}{ax\PYZus{}n3}\PY{o}{.}\PY{n}{set\PYZus{}zlabel}\PY{p}{(}\PY{l+s+s1}{\PYZsq{}}\PY{l+s+s1}{Z}\PY{l+s+s1}{\PYZsq{}}\PY{p}{)}
\PY{c+c1}{\PYZsh{} Add colorbar}
\PY{n}{cbar} \PY{o}{=} \PY{n}{fig}\PY{o}{.}\PY{n}{colorbar}\PY{p}{(}\PY{n}{plot\PYZus{}n3\PYZus{}surface}\PY{p}{,}\PY{n}{ax}\PY{o}{=}\PY{n}{ax\PYZus{}n3}\PY{p}{,} \PY{n}{shrink}\PY{o}{=}\PY{l+m+mf}{0.6}\PY{p}{)}
\PY{n}{cbar}\PY{o}{.}\PY{n}{set\PYZus{}ticks}\PY{p}{(}\PY{p}{[}\PY{l+m+mi}{0}\PY{p}{,} \PY{l+m+mf}{0.25}\PY{p}{,} \PY{l+m+mf}{0.5}\PY{p}{,} \PY{l+m+mf}{0.75}\PY{p}{,} \PY{l+m+mi}{1}\PY{p}{]}\PY{p}{)}
\PY{n}{cbar}\PY{o}{.}\PY{n}{set\PYZus{}ticklabels}\PY{p}{(}\PY{p}{[}\PY{l+s+s1}{\PYZsq{}}\PY{l+s+s1}{0}\PY{l+s+s1}{\PYZsq{}}\PY{p}{,} \PY{l+s+s1}{\PYZsq{}}\PY{l+s+s1}{0.25}\PY{l+s+s1}{\PYZsq{}}\PY{p}{,} \PY{l+s+s1}{\PYZsq{}}\PY{l+s+s1}{0.5}\PY{l+s+s1}{\PYZsq{}}\PY{p}{,} \PY{l+s+s1}{\PYZsq{}}\PY{l+s+s1}{0.75}\PY{l+s+s1}{\PYZsq{}}\PY{p}{,} \PY{l+s+s1}{\PYZsq{}}\PY{l+s+s1}{1}\PY{l+s+s1}{\PYZsq{}}\PY{p}{]}\PY{p}{)}
\end{Verbatim}
\end{tcolorbox}

    \begin{center}
    \adjustimage{max size={0.9\linewidth}{0.9\paperheight}}{Tarea_4_IA_files/Tarea_4_IA_34_0.png}
    \end{center}
    { \hspace*{\fill} \\}
    
    Ahora hacemos lo mismo con la gŕafica de \(f_4\). En esta ocasión
confunde por completo la clase postiva de la negativa, la recta
correspondiente a la frontera de las regiones de decisión en el plano
\((X_1,X_2)\) es aproximadamente la recta que pasa por \((6,-3)\) y
\((2,6)\).

    \begin{tcolorbox}[breakable, size=fbox, boxrule=1pt, pad at break*=1mm,colback=cellbackground, colframe=cellborder]
\prompt{In}{incolor}{10}{\boxspacing}
\begin{Verbatim}[commandchars=\\\{\}]
\PY{c+c1}{\PYZsh{} Create figure }
\PY{n}{fig} \PY{o}{=} \PY{n}{plt}\PY{o}{.}\PY{n}{figure}\PY{p}{(}\PY{n}{figsize}\PY{o}{=}\PY{p}{(}\PY{l+m+mi}{8}\PY{p}{,}\PY{l+m+mi}{8}\PY{p}{)}\PY{p}{)}

\PY{l+s+sd}{\PYZsq{}\PYZsq{}\PYZsq{} }
\PY{l+s+sd}{Plot Neuron n4}
\PY{l+s+sd}{\PYZsq{}\PYZsq{}\PYZsq{}}
\PY{n}{ax\PYZus{}n4} \PY{o}{=} \PY{n}{fig}\PY{o}{.}\PY{n}{add\PYZus{}subplot}\PY{p}{(}\PY{l+m+mi}{111}\PY{p}{,} \PY{n}{projection}\PY{o}{=}\PY{l+s+s1}{\PYZsq{}}\PY{l+s+s1}{3d}\PY{l+s+s1}{\PYZsq{}}\PY{p}{)}
\PY{n}{plot\PYZus{}n4\PYZus{}surface}\PY{o}{=}\PY{n}{ax\PYZus{}n4}\PY{o}{.}\PY{n}{plot\PYZus{}surface}\PY{p}{(}\PY{n}{X1}\PY{p}{,}\PY{n}{X2}\PY{p}{,}\PY{n}{Z4}\PY{p}{,}\PY{n}{cmap}\PY{o}{=}\PY{l+s+s1}{\PYZsq{}}\PY{l+s+s1}{viridis}\PY{l+s+s1}{\PYZsq{}}\PY{p}{)}
\PY{n}{plot\PYZus{}positive}\PY{o}{=}\PY{n}{ax\PYZus{}n4}\PY{o}{.}\PY{n}{scatter3D}\PY{p}{(}\PY{n}{P}\PY{p}{[}\PY{p}{:}\PY{p}{,}\PY{l+m+mi}{0}\PY{p}{]}\PY{p}{,}\PY{n}{P}\PY{p}{[}\PY{p}{:}\PY{p}{,}\PY{l+m+mi}{1}\PY{p}{]}\PY{p}{,}\PY{n}{np}\PY{o}{.}\PY{n}{ones}\PY{p}{(}\PY{n+nb}{len}\PY{p}{(}\PY{n}{P}\PY{p}{[}\PY{p}{:}\PY{p}{,}\PY{l+m+mi}{0}\PY{p}{]}\PY{p}{)}\PY{p}{)}\PY{p}{,}\PY{n}{color}\PY{o}{=}\PY{l+s+s1}{\PYZsq{}}\PY{l+s+s1}{red}\PY{l+s+s1}{\PYZsq{}}\PY{p}{,}\PY{n}{label}\PY{o}{=}\PY{l+s+s1}{\PYZsq{}}\PY{l+s+s1}{positive}\PY{l+s+s1}{\PYZsq{}}\PY{p}{,}\PY{n}{marker}\PY{o}{=}\PY{l+s+s1}{\PYZsq{}}\PY{l+s+s1}{\PYZca{}}\PY{l+s+s1}{\PYZsq{}}\PY{p}{)}
\PY{n}{plot\PYZus{}negative}\PY{o}{=}\PY{n}{ax\PYZus{}n4}\PY{o}{.}\PY{n}{scatter3D}\PY{p}{(}\PY{n}{N}\PY{p}{[}\PY{p}{:}\PY{p}{,}\PY{l+m+mi}{0}\PY{p}{]}\PY{p}{,}\PY{n}{N}\PY{p}{[}\PY{p}{:}\PY{p}{,}\PY{l+m+mi}{1}\PY{p}{]}\PY{p}{,}\PY{n}{np}\PY{o}{.}\PY{n}{zeros}\PY{p}{(}\PY{n+nb}{len}\PY{p}{(}\PY{n}{N}\PY{p}{[}\PY{p}{:}\PY{p}{,}\PY{l+m+mi}{0}\PY{p}{]}\PY{p}{)}\PY{p}{)}\PY{p}{,}\PY{n}{color}\PY{o}{=}\PY{l+s+s1}{\PYZsq{}}\PY{l+s+s1}{b}\PY{l+s+s1}{\PYZsq{}}\PY{p}{,}\PY{n}{label}\PY{o}{=}\PY{l+s+s1}{\PYZsq{}}\PY{l+s+s1}{negative}\PY{l+s+s1}{\PYZsq{}}\PY{p}{,}\PY{n}{marker}\PY{o}{=}\PY{l+s+s1}{\PYZsq{}}\PY{l+s+s1}{*}\PY{l+s+s1}{\PYZsq{}}\PY{p}{)}
\PY{n}{ax\PYZus{}n4}\PY{o}{.}\PY{n}{view\PYZus{}init}\PY{p}{(}\PY{n}{elev}\PY{o}{=}\PY{l+m+mi}{20}\PY{p}{,} \PY{n}{azim}\PY{o}{=}\PY{l+m+mi}{150}\PY{p}{)}
\PY{n}{ax\PYZus{}n4}\PY{o}{.}\PY{n}{dist}\PY{o}{=}\PY{l+m+mi}{11}
\PY{n}{ax\PYZus{}n4}\PY{o}{.}\PY{n}{legend}\PY{p}{(}\PY{p}{)}
\PY{n}{ax\PYZus{}n4}\PY{o}{.}\PY{n}{set\PYZus{}title}\PY{p}{(}\PY{l+s+sa}{r}\PY{l+s+s1}{\PYZsq{}}\PY{l+s+s1}{Neuron \PYZdl{}n\PYZus{}4\PYZdl{}}\PY{l+s+s1}{\PYZsq{}}\PY{p}{)}
\PY{n}{ax\PYZus{}n4}\PY{o}{.}\PY{n}{set\PYZus{}xlabel}\PY{p}{(}\PY{l+s+s1}{\PYZsq{}}\PY{l+s+s1}{X1}\PY{l+s+s1}{\PYZsq{}}\PY{p}{)}
\PY{n}{ax\PYZus{}n4}\PY{o}{.}\PY{n}{set\PYZus{}ylabel}\PY{p}{(}\PY{l+s+s1}{\PYZsq{}}\PY{l+s+s1}{X2}\PY{l+s+s1}{\PYZsq{}}\PY{p}{)}
\PY{n}{ax\PYZus{}n4}\PY{o}{.}\PY{n}{set\PYZus{}zlabel}\PY{p}{(}\PY{l+s+s1}{\PYZsq{}}\PY{l+s+s1}{Z}\PY{l+s+s1}{\PYZsq{}}\PY{p}{)}
\PY{c+c1}{\PYZsh{} Add colorbar}
\PY{n}{cbar} \PY{o}{=} \PY{n}{fig}\PY{o}{.}\PY{n}{colorbar}\PY{p}{(}\PY{n}{plot\PYZus{}n4\PYZus{}surface}\PY{p}{,}\PY{n}{ax}\PY{o}{=}\PY{n}{ax\PYZus{}n4}\PY{p}{,} \PY{n}{shrink}\PY{o}{=}\PY{l+m+mf}{0.6}\PY{p}{)}
\PY{n}{cbar}\PY{o}{.}\PY{n}{set\PYZus{}ticks}\PY{p}{(}\PY{p}{[}\PY{l+m+mi}{0}\PY{p}{,} \PY{l+m+mf}{0.25}\PY{p}{,} \PY{l+m+mf}{0.5}\PY{p}{,} \PY{l+m+mf}{0.75}\PY{p}{,} \PY{l+m+mi}{1}\PY{p}{]}\PY{p}{)}
\PY{n}{cbar}\PY{o}{.}\PY{n}{set\PYZus{}ticklabels}\PY{p}{(}\PY{p}{[}\PY{l+s+s1}{\PYZsq{}}\PY{l+s+s1}{0}\PY{l+s+s1}{\PYZsq{}}\PY{p}{,} \PY{l+s+s1}{\PYZsq{}}\PY{l+s+s1}{0.25}\PY{l+s+s1}{\PYZsq{}}\PY{p}{,} \PY{l+s+s1}{\PYZsq{}}\PY{l+s+s1}{0.5}\PY{l+s+s1}{\PYZsq{}}\PY{p}{,} \PY{l+s+s1}{\PYZsq{}}\PY{l+s+s1}{0.75}\PY{l+s+s1}{\PYZsq{}}\PY{p}{,} \PY{l+s+s1}{\PYZsq{}}\PY{l+s+s1}{1}\PY{l+s+s1}{\PYZsq{}}\PY{p}{]}\PY{p}{)}
\end{Verbatim}
\end{tcolorbox}

    \begin{center}
    \adjustimage{max size={0.9\linewidth}{0.9\paperheight}}{Tarea_4_IA_files/Tarea_4_IA_36_0.png}
    \end{center}
    { \hspace*{\fill} \\}
    
    Finalmente, con la neurona \(n_5\) tenemos que aquí la frontera de las
regiones de decisión justo parece ser uno de los lados del rectángulo
\(\mathcal{R}\), el correspondiente a los vértices \((2,2)\) y
\((8,2)\).

    \begin{tcolorbox}[breakable, size=fbox, boxrule=1pt, pad at break*=1mm,colback=cellbackground, colframe=cellborder]
\prompt{In}{incolor}{11}{\boxspacing}
\begin{Verbatim}[commandchars=\\\{\}]
\PY{c+c1}{\PYZsh{} Create figure }
\PY{n}{fig} \PY{o}{=} \PY{n}{plt}\PY{o}{.}\PY{n}{figure}\PY{p}{(}\PY{n}{figsize}\PY{o}{=}\PY{p}{(}\PY{l+m+mi}{8}\PY{p}{,}\PY{l+m+mi}{8}\PY{p}{)}\PY{p}{)}

\PY{l+s+sd}{\PYZsq{}\PYZsq{}\PYZsq{} }
\PY{l+s+sd}{Plot Neuron n5}
\PY{l+s+sd}{\PYZsq{}\PYZsq{}\PYZsq{}}
\PY{n}{ax\PYZus{}n5} \PY{o}{=} \PY{n}{fig}\PY{o}{.}\PY{n}{add\PYZus{}subplot}\PY{p}{(}\PY{l+m+mi}{111}\PY{p}{,} \PY{n}{projection}\PY{o}{=}\PY{l+s+s1}{\PYZsq{}}\PY{l+s+s1}{3d}\PY{l+s+s1}{\PYZsq{}}\PY{p}{)}
\PY{n}{plot\PYZus{}n5\PYZus{}surface}\PY{o}{=}\PY{n}{ax\PYZus{}n5}\PY{o}{.}\PY{n}{plot\PYZus{}surface}\PY{p}{(}\PY{n}{X1}\PY{p}{,}\PY{n}{X2}\PY{p}{,}\PY{n}{Z5}\PY{p}{,}\PY{n}{cmap}\PY{o}{=}\PY{l+s+s1}{\PYZsq{}}\PY{l+s+s1}{viridis}\PY{l+s+s1}{\PYZsq{}}\PY{p}{)}
\PY{n}{plot\PYZus{}positive}\PY{o}{=}\PY{n}{ax\PYZus{}n5}\PY{o}{.}\PY{n}{scatter3D}\PY{p}{(}\PY{n}{P}\PY{p}{[}\PY{p}{:}\PY{p}{,}\PY{l+m+mi}{0}\PY{p}{]}\PY{p}{,}\PY{n}{P}\PY{p}{[}\PY{p}{:}\PY{p}{,}\PY{l+m+mi}{1}\PY{p}{]}\PY{p}{,}\PY{n}{np}\PY{o}{.}\PY{n}{ones}\PY{p}{(}\PY{n+nb}{len}\PY{p}{(}\PY{n}{P}\PY{p}{[}\PY{p}{:}\PY{p}{,}\PY{l+m+mi}{0}\PY{p}{]}\PY{p}{)}\PY{p}{)}\PY{p}{,}\PY{n}{color}\PY{o}{=}\PY{l+s+s1}{\PYZsq{}}\PY{l+s+s1}{red}\PY{l+s+s1}{\PYZsq{}}\PY{p}{,}\PY{n}{label}\PY{o}{=}\PY{l+s+s1}{\PYZsq{}}\PY{l+s+s1}{positive}\PY{l+s+s1}{\PYZsq{}}\PY{p}{,}\PY{n}{marker}\PY{o}{=}\PY{l+s+s1}{\PYZsq{}}\PY{l+s+s1}{\PYZca{}}\PY{l+s+s1}{\PYZsq{}}\PY{p}{)}
\PY{n}{plot\PYZus{}negative}\PY{o}{=}\PY{n}{ax\PYZus{}n5}\PY{o}{.}\PY{n}{scatter3D}\PY{p}{(}\PY{n}{N}\PY{p}{[}\PY{p}{:}\PY{p}{,}\PY{l+m+mi}{0}\PY{p}{]}\PY{p}{,}\PY{n}{N}\PY{p}{[}\PY{p}{:}\PY{p}{,}\PY{l+m+mi}{1}\PY{p}{]}\PY{p}{,}\PY{n}{np}\PY{o}{.}\PY{n}{zeros}\PY{p}{(}\PY{n+nb}{len}\PY{p}{(}\PY{n}{N}\PY{p}{[}\PY{p}{:}\PY{p}{,}\PY{l+m+mi}{0}\PY{p}{]}\PY{p}{)}\PY{p}{)}\PY{p}{,}\PY{n}{color}\PY{o}{=}\PY{l+s+s1}{\PYZsq{}}\PY{l+s+s1}{b}\PY{l+s+s1}{\PYZsq{}}\PY{p}{,}\PY{n}{label}\PY{o}{=}\PY{l+s+s1}{\PYZsq{}}\PY{l+s+s1}{negative}\PY{l+s+s1}{\PYZsq{}}\PY{p}{,}\PY{n}{marker}\PY{o}{=}\PY{l+s+s1}{\PYZsq{}}\PY{l+s+s1}{*}\PY{l+s+s1}{\PYZsq{}}\PY{p}{)}
\PY{n}{ax\PYZus{}n5}\PY{o}{.}\PY{n}{view\PYZus{}init}\PY{p}{(}\PY{n}{elev}\PY{o}{=}\PY{l+m+mi}{20}\PY{p}{,} \PY{n}{azim}\PY{o}{=}\PY{l+m+mi}{200}\PY{p}{)}
\PY{n}{ax\PYZus{}n5}\PY{o}{.}\PY{n}{dist}\PY{o}{=}\PY{l+m+mi}{11}
\PY{n}{ax\PYZus{}n5}\PY{o}{.}\PY{n}{legend}\PY{p}{(}\PY{p}{)}
\PY{n}{ax\PYZus{}n5}\PY{o}{.}\PY{n}{set\PYZus{}title}\PY{p}{(}\PY{l+s+sa}{r}\PY{l+s+s1}{\PYZsq{}}\PY{l+s+s1}{Neurona \PYZdl{}n\PYZus{}5\PYZdl{}}\PY{l+s+s1}{\PYZsq{}}\PY{p}{)}
\PY{n}{ax\PYZus{}n5}\PY{o}{.}\PY{n}{set\PYZus{}xlabel}\PY{p}{(}\PY{l+s+s1}{\PYZsq{}}\PY{l+s+s1}{X1}\PY{l+s+s1}{\PYZsq{}}\PY{p}{)}
\PY{n}{ax\PYZus{}n5}\PY{o}{.}\PY{n}{set\PYZus{}ylabel}\PY{p}{(}\PY{l+s+s1}{\PYZsq{}}\PY{l+s+s1}{X2}\PY{l+s+s1}{\PYZsq{}}\PY{p}{)}
\PY{n}{ax\PYZus{}n5}\PY{o}{.}\PY{n}{set\PYZus{}zlabel}\PY{p}{(}\PY{l+s+s1}{\PYZsq{}}\PY{l+s+s1}{Z}\PY{l+s+s1}{\PYZsq{}}\PY{p}{)}
\PY{c+c1}{\PYZsh{} Add colorbar}
\PY{n}{cbar} \PY{o}{=} \PY{n}{fig}\PY{o}{.}\PY{n}{colorbar}\PY{p}{(}\PY{n}{plot\PYZus{}n5\PYZus{}surface}\PY{p}{,}\PY{n}{ax}\PY{o}{=}\PY{n}{ax\PYZus{}n5}\PY{p}{,} \PY{n}{shrink}\PY{o}{=}\PY{l+m+mf}{0.6}\PY{p}{)}
\PY{n}{cbar}\PY{o}{.}\PY{n}{set\PYZus{}ticks}\PY{p}{(}\PY{p}{[}\PY{l+m+mi}{0}\PY{p}{,} \PY{l+m+mf}{0.25}\PY{p}{,} \PY{l+m+mf}{0.5}\PY{p}{,} \PY{l+m+mf}{0.75}\PY{p}{,} \PY{l+m+mi}{1}\PY{p}{]}\PY{p}{)}
\PY{n}{cbar}\PY{o}{.}\PY{n}{set\PYZus{}ticklabels}\PY{p}{(}\PY{p}{[}\PY{l+s+s1}{\PYZsq{}}\PY{l+s+s1}{0}\PY{l+s+s1}{\PYZsq{}}\PY{p}{,} \PY{l+s+s1}{\PYZsq{}}\PY{l+s+s1}{0.25}\PY{l+s+s1}{\PYZsq{}}\PY{p}{,} \PY{l+s+s1}{\PYZsq{}}\PY{l+s+s1}{0.5}\PY{l+s+s1}{\PYZsq{}}\PY{p}{,} \PY{l+s+s1}{\PYZsq{}}\PY{l+s+s1}{0.75}\PY{l+s+s1}{\PYZsq{}}\PY{p}{,} \PY{l+s+s1}{\PYZsq{}}\PY{l+s+s1}{1}\PY{l+s+s1}{\PYZsq{}}\PY{p}{]}\PY{p}{)}
\end{Verbatim}
\end{tcolorbox}

    \begin{center}
    \adjustimage{max size={0.9\linewidth}{0.9\paperheight}}{Tarea_4_IA_files/Tarea_4_IA_38_0.png}
    \end{center}
    { \hspace*{\fill} \\}
    
    Finalmente veremos cual es la superficie asociada a la neurona \(n_6\).
Vemos que es la frontera de la región de decisión es el contorno de
nivel de \(0.5\) cuando la salida de la red no ve a la muestra con la
misma probabilidad de ser negativo o positivo. y justo los valores
cercanos a \(1\) de \(f_6\) coinciden con el interior del rectángulo.

    \begin{tcolorbox}[breakable, size=fbox, boxrule=1pt, pad at break*=1mm,colback=cellbackground, colframe=cellborder]
\prompt{In}{incolor}{12}{\boxspacing}
\begin{Verbatim}[commandchars=\\\{\}]
\PY{c+c1}{\PYZsh{} Create figure }
\PY{n}{fig} \PY{o}{=} \PY{n}{plt}\PY{o}{.}\PY{n}{figure}\PY{p}{(}\PY{n}{figsize}\PY{o}{=}\PY{p}{(}\PY{l+m+mi}{8}\PY{p}{,}\PY{l+m+mi}{8}\PY{p}{)}\PY{p}{)}

\PY{l+s+sd}{\PYZsq{}\PYZsq{}\PYZsq{} }
\PY{l+s+sd}{Plot Neuron n6}
\PY{l+s+sd}{\PYZsq{}\PYZsq{}\PYZsq{}}
\PY{n}{ax\PYZus{}n6} \PY{o}{=} \PY{n}{fig}\PY{o}{.}\PY{n}{add\PYZus{}subplot}\PY{p}{(}\PY{l+m+mi}{111}\PY{p}{,} \PY{n}{projection}\PY{o}{=}\PY{l+s+s1}{\PYZsq{}}\PY{l+s+s1}{3d}\PY{l+s+s1}{\PYZsq{}}\PY{p}{)}
\PY{n}{plot\PYZus{}n6\PYZus{}surface}\PY{o}{=}\PY{n}{ax\PYZus{}n6}\PY{o}{.}\PY{n}{plot\PYZus{}surface}\PY{p}{(}\PY{n}{X1}\PY{p}{,}\PY{n}{X2}\PY{p}{,}\PY{n}{Z6}\PY{p}{,}\PY{n}{cmap}\PY{o}{=}\PY{l+s+s1}{\PYZsq{}}\PY{l+s+s1}{viridis}\PY{l+s+s1}{\PYZsq{}}\PY{p}{)}
\PY{n}{plot\PYZus{}positive}\PY{o}{=}\PY{n}{ax\PYZus{}n6}\PY{o}{.}\PY{n}{scatter3D}\PY{p}{(}\PY{n}{P}\PY{p}{[}\PY{p}{:}\PY{p}{,}\PY{l+m+mi}{0}\PY{p}{]}\PY{p}{,}\PY{n}{P}\PY{p}{[}\PY{p}{:}\PY{p}{,}\PY{l+m+mi}{1}\PY{p}{]}\PY{p}{,}\PY{n}{np}\PY{o}{.}\PY{n}{ones}\PY{p}{(}\PY{n+nb}{len}\PY{p}{(}\PY{n}{P}\PY{p}{[}\PY{p}{:}\PY{p}{,}\PY{l+m+mi}{0}\PY{p}{]}\PY{p}{)}\PY{p}{)}\PY{p}{,}\PY{n}{color}\PY{o}{=}\PY{l+s+s1}{\PYZsq{}}\PY{l+s+s1}{red}\PY{l+s+s1}{\PYZsq{}}\PY{p}{,}\PY{n}{label}\PY{o}{=}\PY{l+s+s1}{\PYZsq{}}\PY{l+s+s1}{positive}\PY{l+s+s1}{\PYZsq{}}\PY{p}{,}\PY{n}{marker}\PY{o}{=}\PY{l+s+s1}{\PYZsq{}}\PY{l+s+s1}{\PYZca{}}\PY{l+s+s1}{\PYZsq{}}\PY{p}{)}
\PY{n}{plot\PYZus{}negative}\PY{o}{=}\PY{n}{ax\PYZus{}n6}\PY{o}{.}\PY{n}{scatter3D}\PY{p}{(}\PY{n}{N}\PY{p}{[}\PY{p}{:}\PY{p}{,}\PY{l+m+mi}{0}\PY{p}{]}\PY{p}{,}\PY{n}{N}\PY{p}{[}\PY{p}{:}\PY{p}{,}\PY{l+m+mi}{1}\PY{p}{]}\PY{p}{,}\PY{n}{np}\PY{o}{.}\PY{n}{zeros}\PY{p}{(}\PY{n+nb}{len}\PY{p}{(}\PY{n}{N}\PY{p}{[}\PY{p}{:}\PY{p}{,}\PY{l+m+mi}{0}\PY{p}{]}\PY{p}{)}\PY{p}{)}\PY{p}{,}\PY{n}{color}\PY{o}{=}\PY{l+s+s1}{\PYZsq{}}\PY{l+s+s1}{b}\PY{l+s+s1}{\PYZsq{}}\PY{p}{,}\PY{n}{label}\PY{o}{=}\PY{l+s+s1}{\PYZsq{}}\PY{l+s+s1}{negative}\PY{l+s+s1}{\PYZsq{}}\PY{p}{,}\PY{n}{marker}\PY{o}{=}\PY{l+s+s1}{\PYZsq{}}\PY{l+s+s1}{*}\PY{l+s+s1}{\PYZsq{}}\PY{p}{)}
\PY{n}{ax\PYZus{}n6}\PY{o}{.}\PY{n}{view\PYZus{}init}\PY{p}{(}\PY{n}{elev}\PY{o}{=}\PY{l+m+mi}{20}\PY{p}{,} \PY{n}{azim}\PY{o}{=}\PY{l+m+mi}{10}\PY{p}{)}
\PY{n}{ax\PYZus{}n6}\PY{o}{.}\PY{n}{dist}\PY{o}{=}\PY{l+m+mi}{11}
\PY{n}{ax\PYZus{}n6}\PY{o}{.}\PY{n}{legend}\PY{p}{(}\PY{p}{)}
\PY{n}{ax\PYZus{}n6}\PY{o}{.}\PY{n}{set\PYZus{}title}\PY{p}{(}\PY{l+s+sa}{r}\PY{l+s+s1}{\PYZsq{}}\PY{l+s+s1}{Neurona \PYZdl{}n\PYZus{}6\PYZdl{}}\PY{l+s+s1}{\PYZsq{}}\PY{p}{)}
\PY{n}{ax\PYZus{}n6}\PY{o}{.}\PY{n}{set\PYZus{}xlabel}\PY{p}{(}\PY{l+s+s1}{\PYZsq{}}\PY{l+s+s1}{X1}\PY{l+s+s1}{\PYZsq{}}\PY{p}{)}
\PY{n}{ax\PYZus{}n6}\PY{o}{.}\PY{n}{set\PYZus{}ylabel}\PY{p}{(}\PY{l+s+s1}{\PYZsq{}}\PY{l+s+s1}{X2}\PY{l+s+s1}{\PYZsq{}}\PY{p}{)}
\PY{n}{ax\PYZus{}n6}\PY{o}{.}\PY{n}{set\PYZus{}zlabel}\PY{p}{(}\PY{l+s+s1}{\PYZsq{}}\PY{l+s+s1}{Z}\PY{l+s+s1}{\PYZsq{}}\PY{p}{)}
\PY{c+c1}{\PYZsh{} Add colorbar}
\PY{n}{cbar} \PY{o}{=} \PY{n}{fig}\PY{o}{.}\PY{n}{colorbar}\PY{p}{(}\PY{n}{plot\PYZus{}n6\PYZus{}surface}\PY{p}{,}\PY{n}{ax}\PY{o}{=}\PY{n}{ax\PYZus{}n6}\PY{p}{,} \PY{n}{shrink}\PY{o}{=}\PY{l+m+mf}{0.6}\PY{p}{)}
\PY{n}{cbar}\PY{o}{.}\PY{n}{set\PYZus{}ticks}\PY{p}{(}\PY{p}{[}\PY{l+m+mi}{0}\PY{p}{,} \PY{l+m+mf}{0.25}\PY{p}{,} \PY{l+m+mf}{0.5}\PY{p}{,} \PY{l+m+mf}{0.75}\PY{p}{,} \PY{l+m+mi}{1}\PY{p}{]}\PY{p}{)}
\PY{n}{cbar}\PY{o}{.}\PY{n}{set\PYZus{}ticklabels}\PY{p}{(}\PY{p}{[}\PY{l+s+s1}{\PYZsq{}}\PY{l+s+s1}{0}\PY{l+s+s1}{\PYZsq{}}\PY{p}{,} \PY{l+s+s1}{\PYZsq{}}\PY{l+s+s1}{0.25}\PY{l+s+s1}{\PYZsq{}}\PY{p}{,} \PY{l+s+s1}{\PYZsq{}}\PY{l+s+s1}{0.5}\PY{l+s+s1}{\PYZsq{}}\PY{p}{,} \PY{l+s+s1}{\PYZsq{}}\PY{l+s+s1}{0.75}\PY{l+s+s1}{\PYZsq{}}\PY{p}{,} \PY{l+s+s1}{\PYZsq{}}\PY{l+s+s1}{1}\PY{l+s+s1}{\PYZsq{}}\PY{p}{]}\PY{p}{)}
\end{Verbatim}
\end{tcolorbox}

    \begin{center}
    \adjustimage{max size={0.9\linewidth}{0.9\paperheight}}{Tarea_4_IA_files/Tarea_4_IA_40_0.png}
    \end{center}
    { \hspace*{\fill} \\}
    
    \hypertarget{muxe9todo-2}{%
\subsection{Método 2}\label{muxe9todo-2}}

    Haremos lo mismo utilizando el perceptron multicapas de la librería
\emph{sklearn}.

    En primer lugar realizamos el entrenamiento, considernado igual un
factor de aprendizaje de \(\eta=0.1\), la función de activación sigmoide
y la misma estructura que en nuestra implementación del backpropagation,
aunque aquí el tipo de gradiente es estocástico.

    Para obtener aproximadamente los mismos resultados utilizamos la función
\emph{MLPRegressor} de la librería \emph{sklearn}, esto porque aquí se
utiliza al MSE como función de pérdida al igual que nuestra propia
implementación, además debemos considerar las mismas condiciones
iniciales eso lo hacemos ajustando \emph{warm\_state=True} y luego
modificando los parámetros de la red con las condiciones iniciales que
obtuvimos a partir del Método \(1\).

Cabe señalar que el término de \emph{bias} en \emph{MLPRegressor} se
multiplica por \(+1\) mientras que en la implementación vista en clase
consideramos multiplicar por \(-1\), por eso en el vector de \emph{bias}
del Método 2 es aproximadamente el negativo del vector \emph{bias} del
Método 1.

    \begin{tcolorbox}[breakable, size=fbox, boxrule=1pt, pad at break*=1mm,colback=cellbackground, colframe=cellborder]
\prompt{In}{incolor}{13}{\boxspacing}
\begin{Verbatim}[commandchars=\\\{\}]
\PY{k+kn}{from} \PY{n+nn}{sklearn}\PY{n+nn}{.}\PY{n+nn}{neural\PYZus{}network} \PY{k+kn}{import} \PY{n}{MLPRegressor}
\PY{k+kn}{from} \PY{n+nn}{sklearn}\PY{n+nn}{.}\PY{n+nn}{metrics} \PY{k+kn}{import} \PY{n}{mean\PYZus{}squared\PYZus{}error}
\PY{k+kn}{from} \PY{n+nn}{sklearn}\PY{n+nn}{.}\PY{n+nn}{metrics} \PY{k+kn}{import} \PY{n}{accuracy\PYZus{}score}
\PY{k+kn}{from} \PY{n+nn}{perceptron} \PY{k+kn}{import} \PY{n}{sigmoid}

\PY{k+kn}{import} \PY{n+nn}{warnings}
\PY{n}{warnings}\PY{o}{.}\PY{n}{filterwarnings}\PY{p}{(}\PY{l+s+s2}{\PYZdq{}}\PY{l+s+s2}{ignore}\PY{l+s+s2}{\PYZdq{}}\PY{p}{)}


\PY{c+c1}{\PYZsh{} Define neural network}
\PY{n}{network\PYZus{}method\PYZus{}2}\PY{o}{=}\PY{n}{MLPRegressor}\PY{p}{(}\PY{n}{solver}\PY{o}{=}\PY{l+s+s1}{\PYZsq{}}\PY{l+s+s1}{sgd}\PY{l+s+s1}{\PYZsq{}}\PY{p}{,}\PY{n}{activation}\PY{o}{=}\PY{l+s+s1}{\PYZsq{}}\PY{l+s+s1}{logistic}\PY{l+s+s1}{\PYZsq{}}\PY{p}{,}\PY{n}{hidden\PYZus{}layer\PYZus{}sizes}\PY{o}{=}\PY{p}{(}\PY{l+m+mi}{4}\PY{p}{,}\PY{p}{)}\PY{p}{,}\PY{n}{learning\PYZus{}rate}\PY{o}{=}\PY{l+s+s1}{\PYZsq{}}\PY{l+s+s1}{constant}\PY{l+s+s1}{\PYZsq{}}\PY{p}{,}\PY{n}{learning\PYZus{}rate\PYZus{}init}\PY{o}{=}\PY{l+m+mf}{0.1}\PY{p}{,}\PY{n}{random\PYZus{}state}\PY{o}{=}\PY{n}{SEED}\PY{p}{,}\PY{n}{tol}\PY{o}{=}\PY{l+m+mf}{1e\PYZhy{}4}\PY{p}{,}\PY{n}{max\PYZus{}iter}\PY{o}{=}\PY{n}{MAX\PYZus{}EPOCHS}\PY{p}{,}\PY{n}{n\PYZus{}iter\PYZus{}no\PYZus{}change}\PY{o}{=}\PY{l+m+mi}{100}\PY{p}{,}\PY{n}{shuffle}\PY{o}{=}\PY{k+kc}{False}\PY{p}{,}\PY{n}{warm\PYZus{}start}\PY{o}{=}\PY{k+kc}{True}\PY{p}{,}\PY{n}{alpha}\PY{o}{=}\PY{l+m+mf}{0.0}\PY{p}{,}\PY{n}{momentum}\PY{o}{=}\PY{l+m+mf}{0.0}\PY{p}{)}

\PY{c+c1}{\PYZsh{} Fit model}
\PY{n}{network\PYZus{}method\PYZus{}2}\PY{o}{.}\PY{n}{fit}\PY{p}{(}\PY{n}{X\PYZus{}train}\PY{p}{,}\PY{n}{y\PYZus{}train}\PY{o}{.}\PY{n}{astype}\PY{p}{(}\PY{n+nb}{int}\PY{p}{)}\PY{p}{)}


\PY{n}{network\PYZus{}method\PYZus{}2}\PY{o}{.}\PY{n}{coefs\PYZus{}}\PY{p}{[}\PY{l+m+mi}{0}\PY{p}{]}\PY{o}{=}\PY{n}{network\PYZus{}method\PYZus{}1}\PY{o}{.}\PY{n}{initial\PYZus{}weights}\PY{p}{[}\PY{n}{np}\PY{o}{.}\PY{n}{ix\PYZus{}}\PY{p}{(}\PY{n}{network\PYZus{}method\PYZus{}1}\PY{o}{.}\PY{n}{index\PYZus{}l0}\PY{p}{,}\PY{n}{network\PYZus{}method\PYZus{}1}\PY{o}{.}\PY{n}{index\PYZus{}l1}\PY{p}{)}\PY{p}{]}
\PY{n}{network\PYZus{}method\PYZus{}2}\PY{o}{.}\PY{n}{coefs\PYZus{}}\PY{p}{[}\PY{l+m+mi}{1}\PY{p}{]}\PY{o}{=}\PY{n}{network\PYZus{}method\PYZus{}1}\PY{o}{.}\PY{n}{initial\PYZus{}weights}\PY{p}{[}\PY{n}{np}\PY{o}{.}\PY{n}{ix\PYZus{}}\PY{p}{(}\PY{n}{network\PYZus{}method\PYZus{}1}\PY{o}{.}\PY{n}{index\PYZus{}l1}\PY{p}{,}\PY{n}{network\PYZus{}method\PYZus{}1}\PY{o}{.}\PY{n}{index\PYZus{}l2}\PY{p}{)}\PY{p}{]}
\PY{n}{network\PYZus{}method\PYZus{}2}\PY{o}{.}\PY{n}{intercepts\PYZus{}}\PY{p}{[}\PY{l+m+mi}{0}\PY{p}{]}\PY{o}{=}\PY{o}{\PYZhy{}}\PY{n}{network\PYZus{}method\PYZus{}1}\PY{o}{.}\PY{n}{initial\PYZus{}bias}\PY{p}{[}\PY{n}{network\PYZus{}method\PYZus{}1}\PY{o}{.}\PY{n}{index\PYZus{}l1}\PY{p}{]}
\PY{n}{network\PYZus{}method\PYZus{}2}\PY{o}{.}\PY{n}{intercepts\PYZus{}}\PY{p}{[}\PY{l+m+mi}{1}\PY{p}{]}\PY{o}{=}\PY{o}{\PYZhy{}}\PY{n}{network\PYZus{}method\PYZus{}1}\PY{o}{.}\PY{n}{initial\PYZus{}bias}\PY{p}{[}\PY{n}{network\PYZus{}method\PYZus{}1}\PY{o}{.}\PY{n}{index\PYZus{}l2}\PY{p}{]}

\PY{c+c1}{\PYZsh{} Train again with the same initial conditions of Method 1}
\PY{n}{network\PYZus{}method\PYZus{}2}\PY{o}{.}\PY{n}{fit}\PY{p}{(}\PY{n}{X\PYZus{}train}\PY{p}{,}\PY{n}{y\PYZus{}train}\PY{p}{)}

\PY{c+c1}{\PYZsh{} Measures Train Data}
\PY{n}{y\PYZus{}pred\PYZus{}train\PYZus{}prob}\PY{o}{=}\PY{n}{sigmoid}\PY{p}{(}\PY{n}{network\PYZus{}method\PYZus{}2}\PY{o}{.}\PY{n}{predict}\PY{p}{(}\PY{n}{X\PYZus{}train}\PY{p}{)}\PY{o}{.}\PY{n}{reshape}\PY{p}{(}\PY{n}{X\PYZus{}train}\PY{o}{.}\PY{n}{shape}\PY{p}{[}\PY{l+m+mi}{0}\PY{p}{]}\PY{p}{)}\PY{p}{)}
\PY{n}{y\PYZus{}pred\PYZus{}train}\PY{o}{=}\PY{n}{np}\PY{o}{.}\PY{n}{array}\PY{p}{(}\PY{p}{[}\PY{l+m+mf}{1.0} \PY{k}{if} \PY{n}{prob}\PY{o}{\PYZgt{}}\PY{o}{=}\PY{l+m+mf}{0.5} \PY{k}{else} \PY{l+m+mf}{0.0} \PY{k}{for} \PY{n}{prob} \PY{o+ow}{in} \PY{n}{y\PYZus{}pred\PYZus{}train\PYZus{}prob}\PY{p}{]}\PY{p}{)}
\PY{n}{train\PYZus{}accuracy\PYZus{}score}\PY{o}{=}\PY{n}{accuracy\PYZus{}score}\PY{p}{(}\PY{n}{y\PYZus{}train}\PY{p}{,}\PY{n}{y\PYZus{}pred\PYZus{}train}\PY{p}{)}
\PY{n}{train\PYZus{}accuracy}\PY{o}{=}\PY{n}{accuracy\PYZus{}score}\PY{p}{(}\PY{n}{y\PYZus{}train}\PY{p}{,}\PY{n}{y\PYZus{}pred\PYZus{}train}\PY{p}{,}\PY{n}{normalize}\PY{o}{=}\PY{k+kc}{False}\PY{p}{)}
\PY{n}{train\PYZus{}MSE}\PY{o}{=}\PY{n}{mean\PYZus{}squared\PYZus{}error}\PY{p}{(}\PY{n}{y\PYZus{}train}\PY{p}{,}\PY{n}{y\PYZus{}pred\PYZus{}train\PYZus{}prob}\PY{p}{)}

\PY{c+c1}{\PYZsh{} Printing measures}
\PY{n+nb}{print}\PY{p}{(}\PY{l+s+s1}{\PYZsq{}}\PY{l+s+s1}{Numero de iteraciones para el conjunto de entrenamiento: }\PY{l+s+si}{\PYZpc{}d}\PY{l+s+s1}{\PYZsq{}} \PY{o}{\PYZpc{}}\PY{p}{(}\PY{n}{network\PYZus{}method\PYZus{}2}\PY{o}{.}\PY{n}{n\PYZus{}iter\PYZus{}}\PY{p}{)}\PY{p}{)}
\PY{n+nb}{print}\PY{p}{(}\PY{l+s+s1}{\PYZsq{}}\PY{l+s+s1}{MSE para el entrenamiento: }\PY{l+s+si}{\PYZpc{}.4f}\PY{l+s+s1}{\PYZsq{}} \PY{o}{\PYZpc{}}\PY{p}{(}\PY{n}{train\PYZus{}MSE}\PY{p}{)}\PY{p}{)}
\PY{n+nb}{print}\PY{p}{(}\PY{l+s+s1}{\PYZsq{}}\PY{l+s+s1}{Numero de muestras bien clasificadas (entrenamiento): }\PY{l+s+si}{\PYZpc{}d}\PY{l+s+s1}{\PYZsq{}} \PY{o}{\PYZpc{}}\PY{p}{(}\PY{n}{train\PYZus{}accuracy}\PY{p}{)}\PY{p}{)}
\PY{n+nb}{print}\PY{p}{(}\PY{l+s+s1}{\PYZsq{}}\PY{l+s+s1}{Numero de muestras mal clasificadas (entrenamiento): }\PY{l+s+si}{\PYZpc{}d}\PY{l+s+s1}{\PYZsq{}} \PY{o}{\PYZpc{}}\PY{p}{(}\PY{n}{X\PYZus{}train}\PY{o}{.}\PY{n}{shape}\PY{p}{[}\PY{l+m+mi}{0}\PY{p}{]}\PY{o}{\PYZhy{}}\PY{n}{train\PYZus{}accuracy}\PY{p}{)}\PY{p}{)}
\PY{n+nb}{print}\PY{p}{(}\PY{l+s+s1}{\PYZsq{}}\PY{l+s+s1}{Numero de muestras en el conjunto de entrenamiento: }\PY{l+s+si}{\PYZpc{}d}\PY{l+s+s1}{\PYZsq{}} \PY{o}{\PYZpc{}}\PY{p}{(}\PY{n}{X\PYZus{}train}\PY{o}{.}\PY{n}{shape}\PY{p}{[}\PY{l+m+mi}{0}\PY{p}{]}\PY{p}{)}\PY{p}{)}
\PY{n+nb}{print}\PY{p}{(}\PY{l+s+s1}{\PYZsq{}}\PY{l+s+s1}{Precision en la entrenamiento: }\PY{l+s+si}{\PYZpc{}.4f}\PY{l+s+s1}{\PYZsq{}} \PY{o}{\PYZpc{}}\PY{p}{(}\PY{n}{train\PYZus{}accuracy\PYZus{}score}\PY{p}{)}\PY{p}{)}
\end{Verbatim}
\end{tcolorbox}

    \begin{Verbatim}[commandchars=\\\{\}]
Numero de iteraciones para el conjunto de entrenamiento: 463
MSE para el entrenamiento: 0.0771
Numero de muestras bien clasificadas (entrenamiento): 63
Numero de muestras mal clasificadas (entrenamiento): 7
Numero de muestras en el conjunto de entrenamiento: 70
Precision en la entrenamiento: 0.9000
    \end{Verbatim}

    De igual manera, calculamos tanto la precisión como el MSE del conjunto
de validación.

    \begin{tcolorbox}[breakable, size=fbox, boxrule=1pt, pad at break*=1mm,colback=cellbackground, colframe=cellborder]
\prompt{In}{incolor}{14}{\boxspacing}
\begin{Verbatim}[commandchars=\\\{\}]
\PY{c+c1}{\PYZsh{} Measure Test Data}
\PY{n}{y\PYZus{}pred\PYZus{}test\PYZus{}prob}\PY{o}{=}\PY{n}{sigmoid}\PY{p}{(}\PY{n}{network\PYZus{}method\PYZus{}2}\PY{o}{.}\PY{n}{\PYZus{}predict}\PY{p}{(}\PY{n}{X\PYZus{}test}\PY{p}{)}\PY{o}{.}\PY{n}{reshape}\PY{p}{(}\PY{n}{X\PYZus{}test}\PY{o}{.}\PY{n}{shape}\PY{p}{[}\PY{l+m+mi}{0}\PY{p}{]}\PY{p}{)}\PY{p}{)}
\PY{n}{y\PYZus{}pred\PYZus{}test}\PY{o}{=}\PY{n}{np}\PY{o}{.}\PY{n}{array}\PY{p}{(}\PY{p}{[}\PY{l+m+mf}{1.0} \PY{k}{if} \PY{n}{prob}\PY{o}{\PYZgt{}}\PY{o}{=}\PY{l+m+mf}{0.5} \PY{k}{else} \PY{l+m+mf}{0.0} \PY{k}{for} \PY{n}{prob} \PY{o+ow}{in} \PY{n}{y\PYZus{}pred\PYZus{}test\PYZus{}prob}\PY{p}{]}\PY{p}{)}
\PY{n}{test\PYZus{}accuracy\PYZus{}score}\PY{o}{=}\PY{n}{accuracy\PYZus{}score}\PY{p}{(}\PY{n}{y\PYZus{}test}\PY{p}{,}\PY{n}{y\PYZus{}pred\PYZus{}test}\PY{p}{)}
\PY{n}{test\PYZus{}accuracy}\PY{o}{=}\PY{n}{accuracy\PYZus{}score}\PY{p}{(}\PY{n}{y\PYZus{}test}\PY{p}{,}\PY{n}{y\PYZus{}pred\PYZus{}test}\PY{p}{,}\PY{n}{normalize}\PY{o}{=}\PY{k+kc}{False}\PY{p}{)}
\PY{n}{test\PYZus{}MSE}\PY{o}{=}\PY{n}{mean\PYZus{}squared\PYZus{}error}\PY{p}{(}\PY{n}{y\PYZus{}test}\PY{p}{,}\PY{n}{y\PYZus{}pred\PYZus{}test\PYZus{}prob}\PY{p}{)}

\PY{c+c1}{\PYZsh{} Printing measures}
\PY{n+nb}{print}\PY{p}{(}\PY{l+s+s1}{\PYZsq{}}\PY{l+s+s1}{MSE para la validacion: }\PY{l+s+si}{\PYZpc{}.4f}\PY{l+s+s1}{\PYZsq{}} \PY{o}{\PYZpc{}}\PY{p}{(}\PY{n}{test\PYZus{}MSE}\PY{p}{)}\PY{p}{)}
\PY{n+nb}{print}\PY{p}{(}\PY{l+s+s1}{\PYZsq{}}\PY{l+s+s1}{Numero de muestras bien clasificadas (validacion): }\PY{l+s+si}{\PYZpc{}d}\PY{l+s+s1}{\PYZsq{}} \PY{o}{\PYZpc{}}\PY{p}{(}\PY{n}{test\PYZus{}accuracy}\PY{p}{)}\PY{p}{)}
\PY{n+nb}{print}\PY{p}{(}\PY{l+s+s1}{\PYZsq{}}\PY{l+s+s1}{Numero de muestras mal clasificadas (validacion): }\PY{l+s+si}{\PYZpc{}d}\PY{l+s+s1}{\PYZsq{}} \PY{o}{\PYZpc{}}\PY{p}{(}\PY{n}{X\PYZus{}test}\PY{o}{.}\PY{n}{shape}\PY{p}{[}\PY{l+m+mi}{0}\PY{p}{]}\PY{o}{\PYZhy{}}\PY{n}{test\PYZus{}accuracy}\PY{p}{)}\PY{p}{)}
\PY{n+nb}{print}\PY{p}{(}\PY{l+s+s1}{\PYZsq{}}\PY{l+s+s1}{Numero de muestras en el conjunto de validacion: }\PY{l+s+si}{\PYZpc{}d}\PY{l+s+s1}{\PYZsq{}} \PY{o}{\PYZpc{}}\PY{p}{(}\PY{n}{X\PYZus{}test}\PY{o}{.}\PY{n}{shape}\PY{p}{[}\PY{l+m+mi}{0}\PY{p}{]}\PY{p}{)}\PY{p}{)}
\PY{n+nb}{print}\PY{p}{(}\PY{l+s+s1}{\PYZsq{}}\PY{l+s+s1}{Precision en la validacion: }\PY{l+s+si}{\PYZpc{}.4f}\PY{l+s+s1}{\PYZsq{}} \PY{o}{\PYZpc{}}\PY{p}{(}\PY{n}{test\PYZus{}accuracy\PYZus{}score}\PY{p}{)}\PY{p}{)}
\end{Verbatim}
\end{tcolorbox}

    \begin{Verbatim}[commandchars=\\\{\}]
MSE para la validacion: 0.1081
Numero de muestras bien clasificadas (validacion): 25
Numero de muestras mal clasificadas (validacion): 5
Numero de muestras en el conjunto de validacion: 30
Precision en la validacion: 0.8333
    \end{Verbatim}

    Finalmente reportamos los parámetros que se obtienen con esta librería y
la complejidad del perceptrón multicapas.

    \begin{tcolorbox}[breakable, size=fbox, boxrule=1pt, pad at break*=1mm,colback=cellbackground, colframe=cellborder]
\prompt{In}{incolor}{15}{\boxspacing}
\begin{Verbatim}[commandchars=\\\{\}]
\PY{n+nb}{print}\PY{p}{(}\PY{l+s+s1}{\PYZsq{}}\PY{l+s+s1}{Los pesos de la capa inicial a la capa oculta son:}\PY{l+s+s1}{\PYZsq{}}\PY{p}{)}
\PY{n+nb}{print}\PY{p}{(}\PY{n}{network\PYZus{}method\PYZus{}2}\PY{o}{.}\PY{n}{coefs\PYZus{}}\PY{p}{[}\PY{l+m+mi}{0}\PY{p}{]}\PY{p}{)}
\PY{n+nb}{print}\PY{p}{(}\PY{l+s+s1}{\PYZsq{}}\PY{l+s+se}{\PYZbs{}n}\PY{l+s+s1}{\PYZsq{}}\PY{p}{)}
\PY{n+nb}{print}\PY{p}{(}\PY{l+s+s1}{\PYZsq{}}\PY{l+s+s1}{Los pesos de la capa oculta a la capa de salida son:}\PY{l+s+s1}{\PYZsq{}}\PY{p}{)}
\PY{n+nb}{print}\PY{p}{(}\PY{n}{network\PYZus{}method\PYZus{}2}\PY{o}{.}\PY{n}{coefs\PYZus{}}\PY{p}{[}\PY{l+m+mi}{1}\PY{p}{]}\PY{p}{)}
\PY{n+nb}{print}\PY{p}{(}\PY{l+s+s1}{\PYZsq{}}\PY{l+s+se}{\PYZbs{}n}\PY{l+s+s1}{\PYZsq{}}\PY{p}{)}
\PY{n+nb}{print}\PY{p}{(}\PY{l+s+s1}{\PYZsq{}}\PY{l+s+s1}{El coeficiente asociado al bias para cada neurona de la capa oculta es: }\PY{l+s+s1}{\PYZsq{}}\PY{p}{)}
\PY{n+nb}{print}\PY{p}{(}\PY{n}{network\PYZus{}method\PYZus{}2}\PY{o}{.}\PY{n}{intercepts\PYZus{}}\PY{p}{[}\PY{l+m+mi}{0}\PY{p}{]}\PY{p}{)}
\PY{n+nb}{print}\PY{p}{(}\PY{l+s+s1}{\PYZsq{}}\PY{l+s+se}{\PYZbs{}n}\PY{l+s+s1}{\PYZsq{}}\PY{p}{)}
\PY{n+nb}{print}\PY{p}{(}\PY{l+s+s1}{\PYZsq{}}\PY{l+s+s1}{El coeficiente asociado al bias para cada neurona de la capa de salida es: }\PY{l+s+s1}{\PYZsq{}}\PY{p}{)}
\PY{n+nb}{print}\PY{p}{(}\PY{n}{network\PYZus{}method\PYZus{}2}\PY{o}{.}\PY{n}{intercepts\PYZus{}}\PY{p}{[}\PY{l+m+mi}{1}\PY{p}{]}\PY{p}{)}
\PY{n+nb}{print}\PY{p}{(}\PY{l+s+s1}{\PYZsq{}}\PY{l+s+se}{\PYZbs{}n}\PY{l+s+s1}{\PYZsq{}}\PY{p}{)}
\PY{n+nb}{print}\PY{p}{(}\PY{l+s+s1}{\PYZsq{}}\PY{l+s+s1}{Numero de parametros contando el bias: }\PY{l+s+si}{\PYZpc{}d}\PY{l+s+s1}{\PYZsq{}} \PY{o}{\PYZpc{}}\PY{p}{(}\PY{n}{network\PYZus{}method\PYZus{}1}\PY{o}{.}\PY{n}{num\PYZus{}parameters}\PY{p}{)}\PY{p}{)}
\PY{n+nb}{print}\PY{p}{(}\PY{l+s+s1}{\PYZsq{}}\PY{l+s+se}{\PYZbs{}n}\PY{l+s+s1}{\PYZsq{}}\PY{p}{)}
\PY{n+nb}{print}\PY{p}{(}\PY{l+s+s1}{\PYZsq{}}\PY{l+s+s1}{Numero de capas: }\PY{l+s+si}{\PYZpc{}d}\PY{l+s+s1}{\PYZsq{}} \PY{o}{\PYZpc{}}\PY{p}{(}\PY{n}{network\PYZus{}method\PYZus{}2}\PY{o}{.}\PY{n}{n\PYZus{}layers\PYZus{}}\PY{p}{)}\PY{p}{)}
\end{Verbatim}
\end{tcolorbox}

    \begin{Verbatim}[commandchars=\\\{\}]
Los pesos de la capa inicial a la capa oculta son:
[[ 2.04205389e+00  2.72524293e+00 -3.31726946e+00 -4.59568867e-03]
 [ 9.93870433e-01 -5.67495947e+00 -1.49872621e+00  1.04858521e+00]]


Los pesos de la capa oculta a la capa de salida son:
[[ 5.15590786]
 [-4.38901865]
 [ 0.9534802 ]
 [-5.04814739]]


El coeficiente asociado al bias para cada neurona de la capa oculta es:
[-4.56187236 -1.18070359 -0.29569413 -7.6493035 ]


El coeficiente asociado al bias para cada neurona de la capa de salida es:
[-2.31609806]


Numero de parametros contando el bias: 17


Numero de capas: 3
    \end{Verbatim}

    Con esta paquetería lo que intentamos fue reproducir de la manera más
aproximada el backpropagation que programamos por nuestra cuenta para
validar los resultados de nuestra implementación. Pude notar que otras
herramientas como \emph{Tensorflow} y \emph{Pytorch} e incluso la
función \emph{MLPClassifier} de \emph{scikitlearn} no utilizan el MSE y
utilizan otras formas de optimización de la función de pérdida, por lo
que los resultados con estas herramientas no son comparables con nuestra
implementación.

Con este método los resultados fueron similares al Método 1 pero lo hizo
en \(463\) épocas, el MSE fue mayor tanto en la validación como en el
entrenamiento, respectivamente aunque la precisión es aceptable teniendo
\(90\%\) en el entrenamiento y \(83\%\) en la validación.

    \hypertarget{visualizaciuxf3n}{%
\subsubsection{Visualización}\label{visualizaciuxf3n}}

    A continuación visualizaremos lo que hace cada neurona graficando las
funciones \(f_2,f_3,f_4,f_5\) correspondientes a las neuronas de la capa
escondida y \(f_6\) asociada a la neurona de salida.

    \begin{tcolorbox}[breakable, size=fbox, boxrule=1pt, pad at break*=1mm,colback=cellbackground, colframe=cellborder]
\prompt{In}{incolor}{16}{\boxspacing}
\begin{Verbatim}[commandchars=\\\{\}]
\PY{c+c1}{\PYZsh{} Output of functions  in the grid}

\PY{c+c1}{\PYZsh{} Use de weights of the method 2}
\PY{n}{network\PYZus{}method\PYZus{}1}\PY{o}{.}\PY{n}{weights}\PY{p}{[}\PY{n}{np}\PY{o}{.}\PY{n}{ix\PYZus{}}\PY{p}{(}\PY{n}{network\PYZus{}method\PYZus{}1}\PY{o}{.}\PY{n}{index\PYZus{}l0}\PY{p}{,}\PY{n}{network\PYZus{}method\PYZus{}1}\PY{o}{.}\PY{n}{index\PYZus{}l1}\PY{p}{)}\PY{p}{]}\PY{o}{=}\PY{n}{network\PYZus{}method\PYZus{}2}\PY{o}{.}\PY{n}{coefs\PYZus{}}\PY{p}{[}\PY{l+m+mi}{0}\PY{p}{]}
\PY{n}{network\PYZus{}method\PYZus{}1}\PY{o}{.}\PY{n}{weights}\PY{p}{[}\PY{n}{np}\PY{o}{.}\PY{n}{ix\PYZus{}}\PY{p}{(}\PY{n}{network\PYZus{}method\PYZus{}1}\PY{o}{.}\PY{n}{index\PYZus{}l1}\PY{p}{,}\PY{n}{network\PYZus{}method\PYZus{}1}\PY{o}{.}\PY{n}{index\PYZus{}l2}\PY{p}{)}\PY{p}{]}\PY{o}{=}\PY{n}{network\PYZus{}method\PYZus{}2}\PY{o}{.}\PY{n}{coefs\PYZus{}}\PY{p}{[}\PY{l+m+mi}{1}\PY{p}{]}
\PY{n}{network\PYZus{}method\PYZus{}1}\PY{o}{.}\PY{n}{bias}\PY{p}{[}\PY{n}{np}\PY{o}{.}\PY{n}{ix\PYZus{}}\PY{p}{(}\PY{n}{network\PYZus{}method\PYZus{}1}\PY{o}{.}\PY{n}{index\PYZus{}l1}\PY{p}{)}\PY{p}{]}\PY{o}{=}\PY{o}{\PYZhy{}}\PY{n}{network\PYZus{}method\PYZus{}2}\PY{o}{.}\PY{n}{intercepts\PYZus{}}\PY{p}{[}\PY{l+m+mi}{0}\PY{p}{]}
\PY{n}{network\PYZus{}method\PYZus{}1}\PY{o}{.}\PY{n}{bias}\PY{p}{[}\PY{n}{np}\PY{o}{.}\PY{n}{ix\PYZus{}}\PY{p}{(}\PY{n}{network\PYZus{}method\PYZus{}1}\PY{o}{.}\PY{n}{index\PYZus{}l2}\PY{p}{)}\PY{p}{]}\PY{o}{=}\PY{o}{\PYZhy{}}\PY{n}{network\PYZus{}method\PYZus{}2}\PY{o}{.}\PY{n}{intercepts\PYZus{}}\PY{p}{[}\PY{l+m+mi}{1}\PY{p}{]}

\PY{c+c1}{\PYZsh{} Get the values of functions}
\PY{n}{Z2}\PY{p}{,}\PY{n}{Z3}\PY{p}{,}\PY{n}{Z4}\PY{p}{,}\PY{n}{Z5}\PY{p}{,}\PY{n}{Z6}\PY{o}{=}\PY{p}{[}\PY{p}{]}\PY{p}{,}\PY{p}{[}\PY{p}{]}\PY{p}{,}\PY{p}{[}\PY{p}{]}\PY{p}{,}\PY{p}{[}\PY{p}{]}\PY{p}{,}\PY{p}{[}\PY{p}{]}
\PY{k}{for} \PY{n}{i} \PY{o+ow}{in} \PY{n+nb}{range}\PY{p}{(}\PY{n+nb}{len}\PY{p}{(}\PY{n}{x1}\PY{p}{)}\PY{p}{)}\PY{p}{:}
    \PY{k}{for} \PY{n}{j} \PY{o+ow}{in} \PY{n+nb}{range}\PY{p}{(}\PY{n+nb}{len}\PY{p}{(}\PY{n}{x2}\PY{p}{)}\PY{p}{)}\PY{p}{:}
        \PY{n}{Z2}\PY{o}{.}\PY{n}{append}\PY{p}{(}\PY{n}{f2}\PY{p}{(}\PY{n}{np}\PY{o}{.}\PY{n}{array}\PY{p}{(}\PY{p}{[}\PY{n}{X1}\PY{p}{[}\PY{n}{i}\PY{p}{,}\PY{n}{j}\PY{p}{]}\PY{p}{,}\PY{n}{X2}\PY{p}{[}\PY{n}{i}\PY{p}{,}\PY{n}{j}\PY{p}{]}\PY{p}{]}\PY{p}{)}\PY{p}{)}\PY{p}{)}
        \PY{n}{Z3}\PY{o}{.}\PY{n}{append}\PY{p}{(}\PY{n}{f3}\PY{p}{(}\PY{n}{np}\PY{o}{.}\PY{n}{array}\PY{p}{(}\PY{p}{[}\PY{n}{X1}\PY{p}{[}\PY{n}{i}\PY{p}{,}\PY{n}{j}\PY{p}{]}\PY{p}{,}\PY{n}{X2}\PY{p}{[}\PY{n}{i}\PY{p}{,}\PY{n}{j}\PY{p}{]}\PY{p}{]}\PY{p}{)}\PY{p}{)}\PY{p}{)}
        \PY{n}{Z4}\PY{o}{.}\PY{n}{append}\PY{p}{(}\PY{n}{f4}\PY{p}{(}\PY{n}{np}\PY{o}{.}\PY{n}{array}\PY{p}{(}\PY{p}{[}\PY{n}{X1}\PY{p}{[}\PY{n}{i}\PY{p}{,}\PY{n}{j}\PY{p}{]}\PY{p}{,}\PY{n}{X2}\PY{p}{[}\PY{n}{i}\PY{p}{,}\PY{n}{j}\PY{p}{]}\PY{p}{]}\PY{p}{)}\PY{p}{)}\PY{p}{)}
        \PY{n}{Z5}\PY{o}{.}\PY{n}{append}\PY{p}{(}\PY{n}{f5}\PY{p}{(}\PY{n}{np}\PY{o}{.}\PY{n}{array}\PY{p}{(}\PY{p}{[}\PY{n}{X1}\PY{p}{[}\PY{n}{i}\PY{p}{,}\PY{n}{j}\PY{p}{]}\PY{p}{,}\PY{n}{X2}\PY{p}{[}\PY{n}{i}\PY{p}{,}\PY{n}{j}\PY{p}{]}\PY{p}{]}\PY{p}{)}\PY{p}{)}\PY{p}{)}
        \PY{n}{Z6}\PY{o}{.}\PY{n}{append}\PY{p}{(}\PY{n}{f6}\PY{p}{(}\PY{n}{np}\PY{o}{.}\PY{n}{array}\PY{p}{(}\PY{p}{[}\PY{n}{X1}\PY{p}{[}\PY{n}{i}\PY{p}{,}\PY{n}{j}\PY{p}{]}\PY{p}{,}\PY{n}{X2}\PY{p}{[}\PY{n}{i}\PY{p}{,}\PY{n}{j}\PY{p}{]}\PY{p}{]}\PY{p}{)}\PY{p}{)}\PY{p}{)}
\PY{n}{Z2}\PY{o}{=}\PY{n}{np}\PY{o}{.}\PY{n}{reshape}\PY{p}{(}\PY{n}{Z2}\PY{p}{,}\PY{p}{(}\PY{n+nb}{len}\PY{p}{(}\PY{n}{x1}\PY{p}{)}\PY{p}{,}\PY{n+nb}{len}\PY{p}{(}\PY{n}{x2}\PY{p}{)}\PY{p}{)}\PY{p}{)}
\PY{n}{Z3}\PY{o}{=}\PY{n}{np}\PY{o}{.}\PY{n}{reshape}\PY{p}{(}\PY{n}{Z3}\PY{p}{,}\PY{p}{(}\PY{n+nb}{len}\PY{p}{(}\PY{n}{x1}\PY{p}{)}\PY{p}{,}\PY{n+nb}{len}\PY{p}{(}\PY{n}{x2}\PY{p}{)}\PY{p}{)}\PY{p}{)}
\PY{n}{Z4}\PY{o}{=}\PY{n}{np}\PY{o}{.}\PY{n}{reshape}\PY{p}{(}\PY{n}{Z4}\PY{p}{,}\PY{p}{(}\PY{n+nb}{len}\PY{p}{(}\PY{n}{x1}\PY{p}{)}\PY{p}{,}\PY{n+nb}{len}\PY{p}{(}\PY{n}{x2}\PY{p}{)}\PY{p}{)}\PY{p}{)}
\PY{n}{Z5}\PY{o}{=}\PY{n}{np}\PY{o}{.}\PY{n}{reshape}\PY{p}{(}\PY{n}{Z5}\PY{p}{,}\PY{p}{(}\PY{n+nb}{len}\PY{p}{(}\PY{n}{x1}\PY{p}{)}\PY{p}{,}\PY{n+nb}{len}\PY{p}{(}\PY{n}{x2}\PY{p}{)}\PY{p}{)}\PY{p}{)}
\PY{n}{Z6}\PY{o}{=}\PY{n}{np}\PY{o}{.}\PY{n}{reshape}\PY{p}{(}\PY{n}{Z6}\PY{p}{,}\PY{p}{(}\PY{n+nb}{len}\PY{p}{(}\PY{n}{x1}\PY{p}{)}\PY{p}{,}\PY{n+nb}{len}\PY{p}{(}\PY{n}{x2}\PY{p}{)}\PY{p}{)}\PY{p}{)}
\end{Verbatim}
\end{tcolorbox}

    Con los pesos calculados con la librería, la gráfica de \(f_2\) es la
siguiente, un resultado similar al método 1.

    \begin{tcolorbox}[breakable, size=fbox, boxrule=1pt, pad at break*=1mm,colback=cellbackground, colframe=cellborder]
\prompt{In}{incolor}{17}{\boxspacing}
\begin{Verbatim}[commandchars=\\\{\}]
\PY{c+c1}{\PYZsh{} Create figure }
\PY{n}{fig} \PY{o}{=} \PY{n}{plt}\PY{o}{.}\PY{n}{figure}\PY{p}{(}\PY{n}{figsize}\PY{o}{=}\PY{p}{(}\PY{l+m+mi}{8}\PY{p}{,}\PY{l+m+mi}{8}\PY{p}{)}\PY{p}{)}

\PY{l+s+sd}{\PYZsq{}\PYZsq{}\PYZsq{} }
\PY{l+s+sd}{Plot Neuron N2}
\PY{l+s+sd}{\PYZsq{}\PYZsq{}\PYZsq{}}
\PY{n}{ax\PYZus{}n2} \PY{o}{=} \PY{n}{fig}\PY{o}{.}\PY{n}{add\PYZus{}subplot}\PY{p}{(}\PY{l+m+mi}{111}\PY{p}{,} \PY{n}{projection}\PY{o}{=}\PY{l+s+s1}{\PYZsq{}}\PY{l+s+s1}{3d}\PY{l+s+s1}{\PYZsq{}}\PY{p}{)}
\PY{n}{plot\PYZus{}n2\PYZus{}surface}\PY{o}{=}\PY{n}{ax\PYZus{}n2}\PY{o}{.}\PY{n}{plot\PYZus{}surface}\PY{p}{(}\PY{n}{X1}\PY{p}{,}\PY{n}{X2}\PY{p}{,}\PY{n}{Z2}\PY{p}{,}\PY{n}{cmap}\PY{o}{=}\PY{l+s+s1}{\PYZsq{}}\PY{l+s+s1}{viridis}\PY{l+s+s1}{\PYZsq{}}\PY{p}{)}
\PY{n}{plot\PYZus{}positive}\PY{o}{=}\PY{n}{ax\PYZus{}n2}\PY{o}{.}\PY{n}{scatter3D}\PY{p}{(}\PY{n}{P}\PY{p}{[}\PY{p}{:}\PY{p}{,}\PY{l+m+mi}{0}\PY{p}{]}\PY{p}{,}\PY{n}{P}\PY{p}{[}\PY{p}{:}\PY{p}{,}\PY{l+m+mi}{1}\PY{p}{]}\PY{p}{,}\PY{n}{np}\PY{o}{.}\PY{n}{ones}\PY{p}{(}\PY{n+nb}{len}\PY{p}{(}\PY{n}{P}\PY{p}{[}\PY{p}{:}\PY{p}{,}\PY{l+m+mi}{0}\PY{p}{]}\PY{p}{)}\PY{p}{)}\PY{p}{,}\PY{n}{color}\PY{o}{=}\PY{l+s+s1}{\PYZsq{}}\PY{l+s+s1}{red}\PY{l+s+s1}{\PYZsq{}}\PY{p}{,}\PY{n}{label}\PY{o}{=}\PY{l+s+s1}{\PYZsq{}}\PY{l+s+s1}{positive}\PY{l+s+s1}{\PYZsq{}}\PY{p}{,}\PY{n}{marker}\PY{o}{=}\PY{l+s+s1}{\PYZsq{}}\PY{l+s+s1}{\PYZca{}}\PY{l+s+s1}{\PYZsq{}}\PY{p}{)}
\PY{n}{plot\PYZus{}negative}\PY{o}{=}\PY{n}{ax\PYZus{}n2}\PY{o}{.}\PY{n}{scatter3D}\PY{p}{(}\PY{n}{N}\PY{p}{[}\PY{p}{:}\PY{p}{,}\PY{l+m+mi}{0}\PY{p}{]}\PY{p}{,}\PY{n}{N}\PY{p}{[}\PY{p}{:}\PY{p}{,}\PY{l+m+mi}{1}\PY{p}{]}\PY{p}{,}\PY{n}{np}\PY{o}{.}\PY{n}{zeros}\PY{p}{(}\PY{n+nb}{len}\PY{p}{(}\PY{n}{N}\PY{p}{[}\PY{p}{:}\PY{p}{,}\PY{l+m+mi}{0}\PY{p}{]}\PY{p}{)}\PY{p}{)}\PY{p}{,}\PY{n}{color}\PY{o}{=}\PY{l+s+s1}{\PYZsq{}}\PY{l+s+s1}{b}\PY{l+s+s1}{\PYZsq{}}\PY{p}{,}\PY{n}{label}\PY{o}{=}\PY{l+s+s1}{\PYZsq{}}\PY{l+s+s1}{negative}\PY{l+s+s1}{\PYZsq{}}\PY{p}{,}\PY{n}{marker}\PY{o}{=}\PY{l+s+s1}{\PYZsq{}}\PY{l+s+s1}{*}\PY{l+s+s1}{\PYZsq{}}\PY{p}{)}
\PY{n}{ax\PYZus{}n2}\PY{o}{.}\PY{n}{view\PYZus{}init}\PY{p}{(}\PY{n}{elev}\PY{o}{=}\PY{l+m+mi}{20}\PY{p}{,} \PY{n}{azim}\PY{o}{=}\PY{l+m+mi}{100}\PY{p}{)}
\PY{n}{ax\PYZus{}n2}\PY{o}{.}\PY{n}{dist}\PY{o}{=}\PY{l+m+mi}{11}
\PY{n}{ax\PYZus{}n2}\PY{o}{.}\PY{n}{legend}\PY{p}{(}\PY{p}{)}
\PY{n}{ax\PYZus{}n2}\PY{o}{.}\PY{n}{set\PYZus{}title}\PY{p}{(}\PY{l+s+sa}{r}\PY{l+s+s1}{\PYZsq{}}\PY{l+s+s1}{Neuron \PYZdl{}n\PYZus{}2\PYZdl{}}\PY{l+s+s1}{\PYZsq{}}\PY{p}{)}
\PY{n}{ax\PYZus{}n2}\PY{o}{.}\PY{n}{set\PYZus{}xlabel}\PY{p}{(}\PY{l+s+s1}{\PYZsq{}}\PY{l+s+s1}{X1}\PY{l+s+s1}{\PYZsq{}}\PY{p}{)}
\PY{n}{ax\PYZus{}n2}\PY{o}{.}\PY{n}{set\PYZus{}ylabel}\PY{p}{(}\PY{l+s+s1}{\PYZsq{}}\PY{l+s+s1}{X2}\PY{l+s+s1}{\PYZsq{}}\PY{p}{)}
\PY{n}{ax\PYZus{}n2}\PY{o}{.}\PY{n}{set\PYZus{}zlabel}\PY{p}{(}\PY{l+s+s1}{\PYZsq{}}\PY{l+s+s1}{Z}\PY{l+s+s1}{\PYZsq{}}\PY{p}{)}
\PY{c+c1}{\PYZsh{} Add colorbar}
\PY{n}{cbar} \PY{o}{=} \PY{n}{fig}\PY{o}{.}\PY{n}{colorbar}\PY{p}{(}\PY{n}{plot\PYZus{}n2\PYZus{}surface}\PY{p}{,}\PY{n}{ax}\PY{o}{=}\PY{n}{ax\PYZus{}n2}\PY{p}{,} \PY{n}{shrink}\PY{o}{=}\PY{l+m+mf}{0.6}\PY{p}{)}
\PY{n}{cbar}\PY{o}{.}\PY{n}{set\PYZus{}ticks}\PY{p}{(}\PY{p}{[}\PY{l+m+mi}{0}\PY{p}{,} \PY{l+m+mf}{0.25}\PY{p}{,} \PY{l+m+mf}{0.5}\PY{p}{,} \PY{l+m+mf}{0.75}\PY{p}{,} \PY{l+m+mi}{1}\PY{p}{]}\PY{p}{)}
\PY{n}{cbar}\PY{o}{.}\PY{n}{set\PYZus{}ticklabels}\PY{p}{(}\PY{p}{[}\PY{l+s+s1}{\PYZsq{}}\PY{l+s+s1}{0}\PY{l+s+s1}{\PYZsq{}}\PY{p}{,} \PY{l+s+s1}{\PYZsq{}}\PY{l+s+s1}{0.25}\PY{l+s+s1}{\PYZsq{}}\PY{p}{,} \PY{l+s+s1}{\PYZsq{}}\PY{l+s+s1}{0.5}\PY{l+s+s1}{\PYZsq{}}\PY{p}{,} \PY{l+s+s1}{\PYZsq{}}\PY{l+s+s1}{0.75}\PY{l+s+s1}{\PYZsq{}}\PY{p}{,} \PY{l+s+s1}{\PYZsq{}}\PY{l+s+s1}{1}\PY{l+s+s1}{\PYZsq{}}\PY{p}{]}\PY{p}{)}
\end{Verbatim}
\end{tcolorbox}

    \begin{center}
    \adjustimage{max size={0.9\linewidth}{0.9\paperheight}}{Tarea_4_IA_files/Tarea_4_IA_55_0.png}
    \end{center}
    { \hspace*{\fill} \\}
    
    Para la neurona \(n_3\) tenemos la siguiente visualización, muy similar
también a la superficie de esa neurona en el método 1.

    \begin{tcolorbox}[breakable, size=fbox, boxrule=1pt, pad at break*=1mm,colback=cellbackground, colframe=cellborder]
\prompt{In}{incolor}{18}{\boxspacing}
\begin{Verbatim}[commandchars=\\\{\}]
\PY{c+c1}{\PYZsh{} Create figure }
\PY{n}{fig} \PY{o}{=} \PY{n}{plt}\PY{o}{.}\PY{n}{figure}\PY{p}{(}\PY{n}{figsize}\PY{o}{=}\PY{p}{(}\PY{l+m+mi}{8}\PY{p}{,}\PY{l+m+mi}{8}\PY{p}{)}\PY{p}{)}

\PY{l+s+sd}{\PYZsq{}\PYZsq{}\PYZsq{} }
\PY{l+s+sd}{Plot Neuron n3}
\PY{l+s+sd}{\PYZsq{}\PYZsq{}\PYZsq{}}
\PY{n}{ax\PYZus{}n3} \PY{o}{=} \PY{n}{fig}\PY{o}{.}\PY{n}{add\PYZus{}subplot}\PY{p}{(}\PY{l+m+mi}{111}\PY{p}{,} \PY{n}{projection}\PY{o}{=}\PY{l+s+s1}{\PYZsq{}}\PY{l+s+s1}{3d}\PY{l+s+s1}{\PYZsq{}}\PY{p}{)}
\PY{n}{plot\PYZus{}n3\PYZus{}surface}\PY{o}{=}\PY{n}{ax\PYZus{}n3}\PY{o}{.}\PY{n}{plot\PYZus{}surface}\PY{p}{(}\PY{n}{X1}\PY{p}{,}\PY{n}{X2}\PY{p}{,}\PY{n}{Z3}\PY{p}{,}\PY{n}{cmap}\PY{o}{=}\PY{l+s+s1}{\PYZsq{}}\PY{l+s+s1}{viridis}\PY{l+s+s1}{\PYZsq{}}\PY{p}{)}
\PY{n}{plot\PYZus{}positive}\PY{o}{=}\PY{n}{ax\PYZus{}n3}\PY{o}{.}\PY{n}{scatter3D}\PY{p}{(}\PY{n}{P}\PY{p}{[}\PY{p}{:}\PY{p}{,}\PY{l+m+mi}{0}\PY{p}{]}\PY{p}{,}\PY{n}{P}\PY{p}{[}\PY{p}{:}\PY{p}{,}\PY{l+m+mi}{1}\PY{p}{]}\PY{p}{,}\PY{n}{np}\PY{o}{.}\PY{n}{ones}\PY{p}{(}\PY{n+nb}{len}\PY{p}{(}\PY{n}{P}\PY{p}{[}\PY{p}{:}\PY{p}{,}\PY{l+m+mi}{0}\PY{p}{]}\PY{p}{)}\PY{p}{)}\PY{p}{,}\PY{n}{color}\PY{o}{=}\PY{l+s+s1}{\PYZsq{}}\PY{l+s+s1}{red}\PY{l+s+s1}{\PYZsq{}}\PY{p}{,}\PY{n}{label}\PY{o}{=}\PY{l+s+s1}{\PYZsq{}}\PY{l+s+s1}{positive}\PY{l+s+s1}{\PYZsq{}}\PY{p}{,}\PY{n}{marker}\PY{o}{=}\PY{l+s+s1}{\PYZsq{}}\PY{l+s+s1}{\PYZca{}}\PY{l+s+s1}{\PYZsq{}}\PY{p}{)}
\PY{n}{plot\PYZus{}negative}\PY{o}{=}\PY{n}{ax\PYZus{}n3}\PY{o}{.}\PY{n}{scatter3D}\PY{p}{(}\PY{n}{N}\PY{p}{[}\PY{p}{:}\PY{p}{,}\PY{l+m+mi}{0}\PY{p}{]}\PY{p}{,}\PY{n}{N}\PY{p}{[}\PY{p}{:}\PY{p}{,}\PY{l+m+mi}{1}\PY{p}{]}\PY{p}{,}\PY{n}{np}\PY{o}{.}\PY{n}{zeros}\PY{p}{(}\PY{n+nb}{len}\PY{p}{(}\PY{n}{N}\PY{p}{[}\PY{p}{:}\PY{p}{,}\PY{l+m+mi}{0}\PY{p}{]}\PY{p}{)}\PY{p}{)}\PY{p}{,}\PY{n}{color}\PY{o}{=}\PY{l+s+s1}{\PYZsq{}}\PY{l+s+s1}{b}\PY{l+s+s1}{\PYZsq{}}\PY{p}{,}\PY{n}{label}\PY{o}{=}\PY{l+s+s1}{\PYZsq{}}\PY{l+s+s1}{negative}\PY{l+s+s1}{\PYZsq{}}\PY{p}{,}\PY{n}{marker}\PY{o}{=}\PY{l+s+s1}{\PYZsq{}}\PY{l+s+s1}{*}\PY{l+s+s1}{\PYZsq{}}\PY{p}{)}
\PY{n}{ax\PYZus{}n3}\PY{o}{.}\PY{n}{view\PYZus{}init}\PY{p}{(}\PY{n}{elev}\PY{o}{=}\PY{l+m+mi}{20}\PY{p}{,} \PY{n}{azim}\PY{o}{=}\PY{l+m+mi}{200}\PY{p}{)}
\PY{n}{ax\PYZus{}n3}\PY{o}{.}\PY{n}{dist}\PY{o}{=}\PY{l+m+mi}{11}
\PY{n}{ax\PYZus{}n3}\PY{o}{.}\PY{n}{legend}\PY{p}{(}\PY{p}{)}
\PY{n}{ax\PYZus{}n3}\PY{o}{.}\PY{n}{set\PYZus{}title}\PY{p}{(}\PY{l+s+sa}{r}\PY{l+s+s1}{\PYZsq{}}\PY{l+s+s1}{Neuron \PYZdl{}n\PYZus{}3\PYZdl{}}\PY{l+s+s1}{\PYZsq{}}\PY{p}{)}
\PY{n}{ax\PYZus{}n3}\PY{o}{.}\PY{n}{set\PYZus{}xlabel}\PY{p}{(}\PY{l+s+s1}{\PYZsq{}}\PY{l+s+s1}{X1}\PY{l+s+s1}{\PYZsq{}}\PY{p}{)}
\PY{n}{ax\PYZus{}n3}\PY{o}{.}\PY{n}{set\PYZus{}ylabel}\PY{p}{(}\PY{l+s+s1}{\PYZsq{}}\PY{l+s+s1}{X2}\PY{l+s+s1}{\PYZsq{}}\PY{p}{)}
\PY{n}{ax\PYZus{}n3}\PY{o}{.}\PY{n}{set\PYZus{}zlabel}\PY{p}{(}\PY{l+s+s1}{\PYZsq{}}\PY{l+s+s1}{Z}\PY{l+s+s1}{\PYZsq{}}\PY{p}{)}
\PY{c+c1}{\PYZsh{} Add colorbar}
\PY{n}{cbar} \PY{o}{=} \PY{n}{fig}\PY{o}{.}\PY{n}{colorbar}\PY{p}{(}\PY{n}{plot\PYZus{}n3\PYZus{}surface}\PY{p}{,}\PY{n}{ax}\PY{o}{=}\PY{n}{ax\PYZus{}n3}\PY{p}{,} \PY{n}{shrink}\PY{o}{=}\PY{l+m+mf}{0.6}\PY{p}{)}
\PY{n}{cbar}\PY{o}{.}\PY{n}{set\PYZus{}ticks}\PY{p}{(}\PY{p}{[}\PY{l+m+mi}{0}\PY{p}{,} \PY{l+m+mf}{0.25}\PY{p}{,} \PY{l+m+mf}{0.5}\PY{p}{,} \PY{l+m+mf}{0.75}\PY{p}{,} \PY{l+m+mi}{1}\PY{p}{]}\PY{p}{)}
\PY{n}{cbar}\PY{o}{.}\PY{n}{set\PYZus{}ticklabels}\PY{p}{(}\PY{p}{[}\PY{l+s+s1}{\PYZsq{}}\PY{l+s+s1}{0}\PY{l+s+s1}{\PYZsq{}}\PY{p}{,} \PY{l+s+s1}{\PYZsq{}}\PY{l+s+s1}{0.25}\PY{l+s+s1}{\PYZsq{}}\PY{p}{,} \PY{l+s+s1}{\PYZsq{}}\PY{l+s+s1}{0.5}\PY{l+s+s1}{\PYZsq{}}\PY{p}{,} \PY{l+s+s1}{\PYZsq{}}\PY{l+s+s1}{0.75}\PY{l+s+s1}{\PYZsq{}}\PY{p}{,} \PY{l+s+s1}{\PYZsq{}}\PY{l+s+s1}{1}\PY{l+s+s1}{\PYZsq{}}\PY{p}{]}\PY{p}{)}
\end{Verbatim}
\end{tcolorbox}

    \begin{center}
    \adjustimage{max size={0.9\linewidth}{0.9\paperheight}}{Tarea_4_IA_files/Tarea_4_IA_57_0.png}
    \end{center}
    { \hspace*{\fill} \\}
    
    Ahora, para la neurona \(n_4\), al igual que en el método 1 confunde por
completo la clase positiva de la negativa, seguramente el peso asociado
con esa neurona y la capa de salida será el más pequeño en valor
absoluto, y de hecho en efecto lo es.

    \begin{tcolorbox}[breakable, size=fbox, boxrule=1pt, pad at break*=1mm,colback=cellbackground, colframe=cellborder]
\prompt{In}{incolor}{19}{\boxspacing}
\begin{Verbatim}[commandchars=\\\{\}]
\PY{c+c1}{\PYZsh{} Create figure }
\PY{n}{fig} \PY{o}{=} \PY{n}{plt}\PY{o}{.}\PY{n}{figure}\PY{p}{(}\PY{n}{figsize}\PY{o}{=}\PY{p}{(}\PY{l+m+mi}{8}\PY{p}{,}\PY{l+m+mi}{8}\PY{p}{)}\PY{p}{)}

\PY{l+s+sd}{\PYZsq{}\PYZsq{}\PYZsq{} }
\PY{l+s+sd}{Plot Neuron n4}
\PY{l+s+sd}{\PYZsq{}\PYZsq{}\PYZsq{}}
\PY{n}{ax\PYZus{}n4} \PY{o}{=} \PY{n}{fig}\PY{o}{.}\PY{n}{add\PYZus{}subplot}\PY{p}{(}\PY{l+m+mi}{111}\PY{p}{,} \PY{n}{projection}\PY{o}{=}\PY{l+s+s1}{\PYZsq{}}\PY{l+s+s1}{3d}\PY{l+s+s1}{\PYZsq{}}\PY{p}{)}
\PY{n}{plot\PYZus{}n4\PYZus{}surface}\PY{o}{=}\PY{n}{ax\PYZus{}n4}\PY{o}{.}\PY{n}{plot\PYZus{}surface}\PY{p}{(}\PY{n}{X1}\PY{p}{,}\PY{n}{X2}\PY{p}{,}\PY{n}{Z4}\PY{p}{,}\PY{n}{cmap}\PY{o}{=}\PY{l+s+s1}{\PYZsq{}}\PY{l+s+s1}{viridis}\PY{l+s+s1}{\PYZsq{}}\PY{p}{)}
\PY{n}{plot\PYZus{}positive}\PY{o}{=}\PY{n}{ax\PYZus{}n4}\PY{o}{.}\PY{n}{scatter3D}\PY{p}{(}\PY{n}{P}\PY{p}{[}\PY{p}{:}\PY{p}{,}\PY{l+m+mi}{0}\PY{p}{]}\PY{p}{,}\PY{n}{P}\PY{p}{[}\PY{p}{:}\PY{p}{,}\PY{l+m+mi}{1}\PY{p}{]}\PY{p}{,}\PY{n}{np}\PY{o}{.}\PY{n}{ones}\PY{p}{(}\PY{n+nb}{len}\PY{p}{(}\PY{n}{P}\PY{p}{[}\PY{p}{:}\PY{p}{,}\PY{l+m+mi}{0}\PY{p}{]}\PY{p}{)}\PY{p}{)}\PY{p}{,}\PY{n}{color}\PY{o}{=}\PY{l+s+s1}{\PYZsq{}}\PY{l+s+s1}{red}\PY{l+s+s1}{\PYZsq{}}\PY{p}{,}\PY{n}{label}\PY{o}{=}\PY{l+s+s1}{\PYZsq{}}\PY{l+s+s1}{positive}\PY{l+s+s1}{\PYZsq{}}\PY{p}{,}\PY{n}{marker}\PY{o}{=}\PY{l+s+s1}{\PYZsq{}}\PY{l+s+s1}{\PYZca{}}\PY{l+s+s1}{\PYZsq{}}\PY{p}{)}
\PY{n}{plot\PYZus{}negative}\PY{o}{=}\PY{n}{ax\PYZus{}n4}\PY{o}{.}\PY{n}{scatter3D}\PY{p}{(}\PY{n}{N}\PY{p}{[}\PY{p}{:}\PY{p}{,}\PY{l+m+mi}{0}\PY{p}{]}\PY{p}{,}\PY{n}{N}\PY{p}{[}\PY{p}{:}\PY{p}{,}\PY{l+m+mi}{1}\PY{p}{]}\PY{p}{,}\PY{n}{np}\PY{o}{.}\PY{n}{zeros}\PY{p}{(}\PY{n+nb}{len}\PY{p}{(}\PY{n}{N}\PY{p}{[}\PY{p}{:}\PY{p}{,}\PY{l+m+mi}{0}\PY{p}{]}\PY{p}{)}\PY{p}{)}\PY{p}{,}\PY{n}{color}\PY{o}{=}\PY{l+s+s1}{\PYZsq{}}\PY{l+s+s1}{b}\PY{l+s+s1}{\PYZsq{}}\PY{p}{,}\PY{n}{label}\PY{o}{=}\PY{l+s+s1}{\PYZsq{}}\PY{l+s+s1}{negative}\PY{l+s+s1}{\PYZsq{}}\PY{p}{,}\PY{n}{marker}\PY{o}{=}\PY{l+s+s1}{\PYZsq{}}\PY{l+s+s1}{*}\PY{l+s+s1}{\PYZsq{}}\PY{p}{)}
\PY{n}{ax\PYZus{}n4}\PY{o}{.}\PY{n}{view\PYZus{}init}\PY{p}{(}\PY{n}{elev}\PY{o}{=}\PY{l+m+mi}{20}\PY{p}{,} \PY{n}{azim}\PY{o}{=}\PY{l+m+mi}{150}\PY{p}{)}
\PY{n}{ax\PYZus{}n4}\PY{o}{.}\PY{n}{dist}\PY{o}{=}\PY{l+m+mi}{11}
\PY{n}{ax\PYZus{}n4}\PY{o}{.}\PY{n}{legend}\PY{p}{(}\PY{p}{)}
\PY{n}{ax\PYZus{}n4}\PY{o}{.}\PY{n}{set\PYZus{}title}\PY{p}{(}\PY{l+s+sa}{r}\PY{l+s+s1}{\PYZsq{}}\PY{l+s+s1}{Neuron \PYZdl{}n\PYZus{}4\PYZdl{}}\PY{l+s+s1}{\PYZsq{}}\PY{p}{)}
\PY{n}{ax\PYZus{}n4}\PY{o}{.}\PY{n}{set\PYZus{}xlabel}\PY{p}{(}\PY{l+s+s1}{\PYZsq{}}\PY{l+s+s1}{X1}\PY{l+s+s1}{\PYZsq{}}\PY{p}{)}
\PY{n}{ax\PYZus{}n4}\PY{o}{.}\PY{n}{set\PYZus{}ylabel}\PY{p}{(}\PY{l+s+s1}{\PYZsq{}}\PY{l+s+s1}{X2}\PY{l+s+s1}{\PYZsq{}}\PY{p}{)}
\PY{n}{ax\PYZus{}n4}\PY{o}{.}\PY{n}{set\PYZus{}zlabel}\PY{p}{(}\PY{l+s+s1}{\PYZsq{}}\PY{l+s+s1}{Z}\PY{l+s+s1}{\PYZsq{}}\PY{p}{)}
\PY{c+c1}{\PYZsh{} Add colorbar}
\PY{n}{cbar} \PY{o}{=} \PY{n}{fig}\PY{o}{.}\PY{n}{colorbar}\PY{p}{(}\PY{n}{plot\PYZus{}n4\PYZus{}surface}\PY{p}{,}\PY{n}{ax}\PY{o}{=}\PY{n}{ax\PYZus{}n4}\PY{p}{,} \PY{n}{shrink}\PY{o}{=}\PY{l+m+mf}{0.6}\PY{p}{)}
\PY{n}{cbar}\PY{o}{.}\PY{n}{set\PYZus{}ticks}\PY{p}{(}\PY{p}{[}\PY{l+m+mi}{0}\PY{p}{,} \PY{l+m+mf}{0.25}\PY{p}{,} \PY{l+m+mf}{0.5}\PY{p}{,} \PY{l+m+mf}{0.75}\PY{p}{,} \PY{l+m+mi}{1}\PY{p}{]}\PY{p}{)}
\PY{n}{cbar}\PY{o}{.}\PY{n}{set\PYZus{}ticklabels}\PY{p}{(}\PY{p}{[}\PY{l+s+s1}{\PYZsq{}}\PY{l+s+s1}{0}\PY{l+s+s1}{\PYZsq{}}\PY{p}{,} \PY{l+s+s1}{\PYZsq{}}\PY{l+s+s1}{0.25}\PY{l+s+s1}{\PYZsq{}}\PY{p}{,} \PY{l+s+s1}{\PYZsq{}}\PY{l+s+s1}{0.5}\PY{l+s+s1}{\PYZsq{}}\PY{p}{,} \PY{l+s+s1}{\PYZsq{}}\PY{l+s+s1}{0.75}\PY{l+s+s1}{\PYZsq{}}\PY{p}{,} \PY{l+s+s1}{\PYZsq{}}\PY{l+s+s1}{1}\PY{l+s+s1}{\PYZsq{}}\PY{p}{]}\PY{p}{)}
\end{Verbatim}
\end{tcolorbox}

    \begin{center}
    \adjustimage{max size={0.9\linewidth}{0.9\paperheight}}{Tarea_4_IA_files/Tarea_4_IA_59_0.png}
    \end{center}
    { \hspace*{\fill} \\}
    
    Para la última neurona de la capa oculta, la \(n_5\) tenemos el
siguiente gráfico. Esta se ve un poco más desplazada hacia la izquierda
en comparación con la del método \(1\) e igual parece estar la frontera
de la región de decisión a uno de los lados del rectángulo, el
correspondiente a los vértices \((2,2)\) y \((8,2)\)

    \begin{tcolorbox}[breakable, size=fbox, boxrule=1pt, pad at break*=1mm,colback=cellbackground, colframe=cellborder]
\prompt{In}{incolor}{20}{\boxspacing}
\begin{Verbatim}[commandchars=\\\{\}]
\PY{c+c1}{\PYZsh{} Create figure }
\PY{n}{fig} \PY{o}{=} \PY{n}{plt}\PY{o}{.}\PY{n}{figure}\PY{p}{(}\PY{n}{figsize}\PY{o}{=}\PY{p}{(}\PY{l+m+mi}{8}\PY{p}{,}\PY{l+m+mi}{8}\PY{p}{)}\PY{p}{)}

\PY{l+s+sd}{\PYZsq{}\PYZsq{}\PYZsq{} }
\PY{l+s+sd}{Plot Neuron n5}
\PY{l+s+sd}{\PYZsq{}\PYZsq{}\PYZsq{}}
\PY{n}{ax\PYZus{}n5} \PY{o}{=} \PY{n}{fig}\PY{o}{.}\PY{n}{add\PYZus{}subplot}\PY{p}{(}\PY{l+m+mi}{111}\PY{p}{,} \PY{n}{projection}\PY{o}{=}\PY{l+s+s1}{\PYZsq{}}\PY{l+s+s1}{3d}\PY{l+s+s1}{\PYZsq{}}\PY{p}{)}
\PY{n}{plot\PYZus{}n5\PYZus{}surface}\PY{o}{=}\PY{n}{ax\PYZus{}n5}\PY{o}{.}\PY{n}{plot\PYZus{}surface}\PY{p}{(}\PY{n}{X1}\PY{p}{,}\PY{n}{X2}\PY{p}{,}\PY{n}{Z5}\PY{p}{,}\PY{n}{cmap}\PY{o}{=}\PY{l+s+s1}{\PYZsq{}}\PY{l+s+s1}{viridis}\PY{l+s+s1}{\PYZsq{}}\PY{p}{)}
\PY{n}{plot\PYZus{}positive}\PY{o}{=}\PY{n}{ax\PYZus{}n5}\PY{o}{.}\PY{n}{scatter3D}\PY{p}{(}\PY{n}{P}\PY{p}{[}\PY{p}{:}\PY{p}{,}\PY{l+m+mi}{0}\PY{p}{]}\PY{p}{,}\PY{n}{P}\PY{p}{[}\PY{p}{:}\PY{p}{,}\PY{l+m+mi}{1}\PY{p}{]}\PY{p}{,}\PY{n}{np}\PY{o}{.}\PY{n}{ones}\PY{p}{(}\PY{n+nb}{len}\PY{p}{(}\PY{n}{P}\PY{p}{[}\PY{p}{:}\PY{p}{,}\PY{l+m+mi}{0}\PY{p}{]}\PY{p}{)}\PY{p}{)}\PY{p}{,}\PY{n}{color}\PY{o}{=}\PY{l+s+s1}{\PYZsq{}}\PY{l+s+s1}{red}\PY{l+s+s1}{\PYZsq{}}\PY{p}{,}\PY{n}{label}\PY{o}{=}\PY{l+s+s1}{\PYZsq{}}\PY{l+s+s1}{positive}\PY{l+s+s1}{\PYZsq{}}\PY{p}{,}\PY{n}{marker}\PY{o}{=}\PY{l+s+s1}{\PYZsq{}}\PY{l+s+s1}{\PYZca{}}\PY{l+s+s1}{\PYZsq{}}\PY{p}{)}
\PY{n}{plot\PYZus{}negative}\PY{o}{=}\PY{n}{ax\PYZus{}n5}\PY{o}{.}\PY{n}{scatter3D}\PY{p}{(}\PY{n}{N}\PY{p}{[}\PY{p}{:}\PY{p}{,}\PY{l+m+mi}{0}\PY{p}{]}\PY{p}{,}\PY{n}{N}\PY{p}{[}\PY{p}{:}\PY{p}{,}\PY{l+m+mi}{1}\PY{p}{]}\PY{p}{,}\PY{n}{np}\PY{o}{.}\PY{n}{zeros}\PY{p}{(}\PY{n+nb}{len}\PY{p}{(}\PY{n}{N}\PY{p}{[}\PY{p}{:}\PY{p}{,}\PY{l+m+mi}{0}\PY{p}{]}\PY{p}{)}\PY{p}{)}\PY{p}{,}\PY{n}{color}\PY{o}{=}\PY{l+s+s1}{\PYZsq{}}\PY{l+s+s1}{b}\PY{l+s+s1}{\PYZsq{}}\PY{p}{,}\PY{n}{label}\PY{o}{=}\PY{l+s+s1}{\PYZsq{}}\PY{l+s+s1}{negative}\PY{l+s+s1}{\PYZsq{}}\PY{p}{,}\PY{n}{marker}\PY{o}{=}\PY{l+s+s1}{\PYZsq{}}\PY{l+s+s1}{*}\PY{l+s+s1}{\PYZsq{}}\PY{p}{)}
\PY{n}{ax\PYZus{}n5}\PY{o}{.}\PY{n}{view\PYZus{}init}\PY{p}{(}\PY{n}{elev}\PY{o}{=}\PY{l+m+mi}{20}\PY{p}{,} \PY{n}{azim}\PY{o}{=}\PY{l+m+mi}{200}\PY{p}{)}
\PY{n}{ax\PYZus{}n5}\PY{o}{.}\PY{n}{dist}\PY{o}{=}\PY{l+m+mi}{11}
\PY{n}{ax\PYZus{}n5}\PY{o}{.}\PY{n}{legend}\PY{p}{(}\PY{p}{)}
\PY{n}{ax\PYZus{}n5}\PY{o}{.}\PY{n}{set\PYZus{}title}\PY{p}{(}\PY{l+s+sa}{r}\PY{l+s+s1}{\PYZsq{}}\PY{l+s+s1}{Neurona \PYZdl{}n\PYZus{}5\PYZdl{}}\PY{l+s+s1}{\PYZsq{}}\PY{p}{)}
\PY{n}{ax\PYZus{}n5}\PY{o}{.}\PY{n}{set\PYZus{}xlabel}\PY{p}{(}\PY{l+s+s1}{\PYZsq{}}\PY{l+s+s1}{X1}\PY{l+s+s1}{\PYZsq{}}\PY{p}{)}
\PY{n}{ax\PYZus{}n5}\PY{o}{.}\PY{n}{set\PYZus{}ylabel}\PY{p}{(}\PY{l+s+s1}{\PYZsq{}}\PY{l+s+s1}{X2}\PY{l+s+s1}{\PYZsq{}}\PY{p}{)}
\PY{n}{ax\PYZus{}n5}\PY{o}{.}\PY{n}{set\PYZus{}zlabel}\PY{p}{(}\PY{l+s+s1}{\PYZsq{}}\PY{l+s+s1}{Z}\PY{l+s+s1}{\PYZsq{}}\PY{p}{)}
\PY{c+c1}{\PYZsh{} Add colorbar}
\PY{n}{cbar} \PY{o}{=} \PY{n}{fig}\PY{o}{.}\PY{n}{colorbar}\PY{p}{(}\PY{n}{plot\PYZus{}n5\PYZus{}surface}\PY{p}{,}\PY{n}{ax}\PY{o}{=}\PY{n}{ax\PYZus{}n5}\PY{p}{,} \PY{n}{shrink}\PY{o}{=}\PY{l+m+mf}{0.6}\PY{p}{)}
\PY{n}{cbar}\PY{o}{.}\PY{n}{set\PYZus{}ticks}\PY{p}{(}\PY{p}{[}\PY{l+m+mi}{0}\PY{p}{,} \PY{l+m+mf}{0.25}\PY{p}{,} \PY{l+m+mf}{0.5}\PY{p}{,} \PY{l+m+mf}{0.75}\PY{p}{,} \PY{l+m+mi}{1}\PY{p}{]}\PY{p}{)}
\PY{n}{cbar}\PY{o}{.}\PY{n}{set\PYZus{}ticklabels}\PY{p}{(}\PY{p}{[}\PY{l+s+s1}{\PYZsq{}}\PY{l+s+s1}{0}\PY{l+s+s1}{\PYZsq{}}\PY{p}{,} \PY{l+s+s1}{\PYZsq{}}\PY{l+s+s1}{0.25}\PY{l+s+s1}{\PYZsq{}}\PY{p}{,} \PY{l+s+s1}{\PYZsq{}}\PY{l+s+s1}{0.5}\PY{l+s+s1}{\PYZsq{}}\PY{p}{,} \PY{l+s+s1}{\PYZsq{}}\PY{l+s+s1}{0.75}\PY{l+s+s1}{\PYZsq{}}\PY{p}{,} \PY{l+s+s1}{\PYZsq{}}\PY{l+s+s1}{1}\PY{l+s+s1}{\PYZsq{}}\PY{p}{]}\PY{p}{)}
\end{Verbatim}
\end{tcolorbox}

    \begin{center}
    \adjustimage{max size={0.9\linewidth}{0.9\paperheight}}{Tarea_4_IA_files/Tarea_4_IA_61_0.png}
    \end{center}
    { \hspace*{\fill} \\}
    
    Finalmente, la superficie asociada a la neurona de la capa de salida
\(n_6\) es la siguiente. Comparando las superficies del método \(1\) con
esta que corresponde al método \(2\) podemos ver que el tubo del método
\(1\) es más estrecho y concentrado en el centro del rectángulo que
contiene los puntos de la clase positiva, mientras que el del método
\(2\) tiene un tubo más ancho y en consecuencia es mas susceptible a
fallar en la clasificación pues estaría haciendo el contorno de nive
\(0.5\) más ancho de lo que realmente debe ser por eso visualmente hay
menor precisión en el método \(2\) que en el método \(1\) al menos con
la elección de los conjuntos \(E\) y \(V\) que tenemos.

    \begin{tcolorbox}[breakable, size=fbox, boxrule=1pt, pad at break*=1mm,colback=cellbackground, colframe=cellborder]
\prompt{In}{incolor}{21}{\boxspacing}
\begin{Verbatim}[commandchars=\\\{\}]
\PY{c+c1}{\PYZsh{} Create figure }
\PY{n}{fig} \PY{o}{=} \PY{n}{plt}\PY{o}{.}\PY{n}{figure}\PY{p}{(}\PY{n}{figsize}\PY{o}{=}\PY{p}{(}\PY{l+m+mi}{8}\PY{p}{,}\PY{l+m+mi}{8}\PY{p}{)}\PY{p}{)}

\PY{l+s+sd}{\PYZsq{}\PYZsq{}\PYZsq{} }
\PY{l+s+sd}{Plot Neuron n6}
\PY{l+s+sd}{\PYZsq{}\PYZsq{}\PYZsq{}}
\PY{n}{ax\PYZus{}n6} \PY{o}{=} \PY{n}{fig}\PY{o}{.}\PY{n}{add\PYZus{}subplot}\PY{p}{(}\PY{l+m+mi}{111}\PY{p}{,} \PY{n}{projection}\PY{o}{=}\PY{l+s+s1}{\PYZsq{}}\PY{l+s+s1}{3d}\PY{l+s+s1}{\PYZsq{}}\PY{p}{)}
\PY{n}{plot\PYZus{}n6\PYZus{}surface}\PY{o}{=}\PY{n}{ax\PYZus{}n6}\PY{o}{.}\PY{n}{plot\PYZus{}surface}\PY{p}{(}\PY{n}{X1}\PY{p}{,}\PY{n}{X2}\PY{p}{,}\PY{n}{Z6}\PY{p}{,}\PY{n}{cmap}\PY{o}{=}\PY{l+s+s1}{\PYZsq{}}\PY{l+s+s1}{viridis}\PY{l+s+s1}{\PYZsq{}}\PY{p}{)}
\PY{n}{plot\PYZus{}positive}\PY{o}{=}\PY{n}{ax\PYZus{}n6}\PY{o}{.}\PY{n}{scatter3D}\PY{p}{(}\PY{n}{P}\PY{p}{[}\PY{p}{:}\PY{p}{,}\PY{l+m+mi}{0}\PY{p}{]}\PY{p}{,}\PY{n}{P}\PY{p}{[}\PY{p}{:}\PY{p}{,}\PY{l+m+mi}{1}\PY{p}{]}\PY{p}{,}\PY{n}{np}\PY{o}{.}\PY{n}{ones}\PY{p}{(}\PY{n+nb}{len}\PY{p}{(}\PY{n}{P}\PY{p}{[}\PY{p}{:}\PY{p}{,}\PY{l+m+mi}{0}\PY{p}{]}\PY{p}{)}\PY{p}{)}\PY{p}{,}\PY{n}{color}\PY{o}{=}\PY{l+s+s1}{\PYZsq{}}\PY{l+s+s1}{red}\PY{l+s+s1}{\PYZsq{}}\PY{p}{,}\PY{n}{label}\PY{o}{=}\PY{l+s+s1}{\PYZsq{}}\PY{l+s+s1}{positive}\PY{l+s+s1}{\PYZsq{}}\PY{p}{,}\PY{n}{marker}\PY{o}{=}\PY{l+s+s1}{\PYZsq{}}\PY{l+s+s1}{\PYZca{}}\PY{l+s+s1}{\PYZsq{}}\PY{p}{)}
\PY{n}{plot\PYZus{}negative}\PY{o}{=}\PY{n}{ax\PYZus{}n6}\PY{o}{.}\PY{n}{scatter3D}\PY{p}{(}\PY{n}{N}\PY{p}{[}\PY{p}{:}\PY{p}{,}\PY{l+m+mi}{0}\PY{p}{]}\PY{p}{,}\PY{n}{N}\PY{p}{[}\PY{p}{:}\PY{p}{,}\PY{l+m+mi}{1}\PY{p}{]}\PY{p}{,}\PY{n}{np}\PY{o}{.}\PY{n}{zeros}\PY{p}{(}\PY{n+nb}{len}\PY{p}{(}\PY{n}{N}\PY{p}{[}\PY{p}{:}\PY{p}{,}\PY{l+m+mi}{0}\PY{p}{]}\PY{p}{)}\PY{p}{)}\PY{p}{,}\PY{n}{color}\PY{o}{=}\PY{l+s+s1}{\PYZsq{}}\PY{l+s+s1}{b}\PY{l+s+s1}{\PYZsq{}}\PY{p}{,}\PY{n}{label}\PY{o}{=}\PY{l+s+s1}{\PYZsq{}}\PY{l+s+s1}{negative}\PY{l+s+s1}{\PYZsq{}}\PY{p}{,}\PY{n}{marker}\PY{o}{=}\PY{l+s+s1}{\PYZsq{}}\PY{l+s+s1}{*}\PY{l+s+s1}{\PYZsq{}}\PY{p}{)}
\PY{n}{ax\PYZus{}n6}\PY{o}{.}\PY{n}{view\PYZus{}init}\PY{p}{(}\PY{n}{elev}\PY{o}{=}\PY{l+m+mi}{20}\PY{p}{,} \PY{n}{azim}\PY{o}{=}\PY{l+m+mi}{10}\PY{p}{)}
\PY{n}{ax\PYZus{}n6}\PY{o}{.}\PY{n}{dist}\PY{o}{=}\PY{l+m+mi}{11}
\PY{n}{ax\PYZus{}n6}\PY{o}{.}\PY{n}{legend}\PY{p}{(}\PY{p}{)}
\PY{n}{ax\PYZus{}n6}\PY{o}{.}\PY{n}{set\PYZus{}title}\PY{p}{(}\PY{l+s+sa}{r}\PY{l+s+s1}{\PYZsq{}}\PY{l+s+s1}{Neurona \PYZdl{}n\PYZus{}6\PYZdl{}}\PY{l+s+s1}{\PYZsq{}}\PY{p}{)}
\PY{n}{ax\PYZus{}n6}\PY{o}{.}\PY{n}{set\PYZus{}xlabel}\PY{p}{(}\PY{l+s+s1}{\PYZsq{}}\PY{l+s+s1}{X1}\PY{l+s+s1}{\PYZsq{}}\PY{p}{)}
\PY{n}{ax\PYZus{}n6}\PY{o}{.}\PY{n}{set\PYZus{}ylabel}\PY{p}{(}\PY{l+s+s1}{\PYZsq{}}\PY{l+s+s1}{X2}\PY{l+s+s1}{\PYZsq{}}\PY{p}{)}
\PY{n}{ax\PYZus{}n6}\PY{o}{.}\PY{n}{set\PYZus{}zlabel}\PY{p}{(}\PY{l+s+s1}{\PYZsq{}}\PY{l+s+s1}{Z}\PY{l+s+s1}{\PYZsq{}}\PY{p}{)}
\PY{c+c1}{\PYZsh{} Add colorbar}
\PY{n}{cbar} \PY{o}{=} \PY{n}{fig}\PY{o}{.}\PY{n}{colorbar}\PY{p}{(}\PY{n}{plot\PYZus{}n6\PYZus{}surface}\PY{p}{,}\PY{n}{ax}\PY{o}{=}\PY{n}{ax\PYZus{}n6}\PY{p}{,} \PY{n}{shrink}\PY{o}{=}\PY{l+m+mf}{0.6}\PY{p}{)}
\PY{n}{cbar}\PY{o}{.}\PY{n}{set\PYZus{}ticks}\PY{p}{(}\PY{p}{[}\PY{l+m+mi}{0}\PY{p}{,} \PY{l+m+mf}{0.25}\PY{p}{,} \PY{l+m+mf}{0.5}\PY{p}{,} \PY{l+m+mf}{0.75}\PY{p}{,} \PY{l+m+mi}{1}\PY{p}{]}\PY{p}{)}
\PY{n}{cbar}\PY{o}{.}\PY{n}{set\PYZus{}ticklabels}\PY{p}{(}\PY{p}{[}\PY{l+s+s1}{\PYZsq{}}\PY{l+s+s1}{0}\PY{l+s+s1}{\PYZsq{}}\PY{p}{,} \PY{l+s+s1}{\PYZsq{}}\PY{l+s+s1}{0.25}\PY{l+s+s1}{\PYZsq{}}\PY{p}{,} \PY{l+s+s1}{\PYZsq{}}\PY{l+s+s1}{0.5}\PY{l+s+s1}{\PYZsq{}}\PY{p}{,} \PY{l+s+s1}{\PYZsq{}}\PY{l+s+s1}{0.75}\PY{l+s+s1}{\PYZsq{}}\PY{p}{,} \PY{l+s+s1}{\PYZsq{}}\PY{l+s+s1}{1}\PY{l+s+s1}{\PYZsq{}}\PY{p}{]}\PY{p}{)}
\end{Verbatim}
\end{tcolorbox}

    \begin{center}
    \adjustimage{max size={0.9\linewidth}{0.9\paperheight}}{Tarea_4_IA_files/Tarea_4_IA_63_0.png}
    \end{center}
    { \hspace*{\fill} \\}
    

    % Add a bibliography block to the postdoc
    
    
    
\end{document}
