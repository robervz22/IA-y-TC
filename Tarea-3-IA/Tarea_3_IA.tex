\documentclass[11pt]{article}
    \usepackage{array}
    \usepackage[breakable]{tcolorbox}
    \usepackage{parskip} % Stop auto-indenting (to mimic markdown behaviour)
    
    \usepackage{iftex}
    \ifPDFTeX
    	\usepackage[T1]{fontenc}
    	\usepackage{mathpazo}
    \else
    	\usepackage{fontspec}
    \fi

    % Basic figure setup, for now with no caption control since it's done
    % automatically by Pandoc (which extracts ![](path) syntax from Markdown).
    \usepackage{graphicx}
    % Maintain compatibility with old templates. Remove in nbconvert 6.0
    \let\Oldincludegraphics\includegraphics
    % Ensure that by default, figures have no caption (until we provide a
    % proper Figure object with a Caption API and a way to capture that
    % in the conversion process - todo).
    \usepackage{caption}
    \DeclareCaptionFormat{nocaption}{}
    \captionsetup{format=nocaption,aboveskip=0pt,belowskip=0pt}

    \usepackage[Export]{adjustbox} % Used to constrain images to a maximum size
    \adjustboxset{max size={0.9\linewidth}{0.9\paperheight}}
    \usepackage{float}
    \floatplacement{figure}{H} % forces figures to be placed at the correct location
    \usepackage{xcolor} % Allow colors to be defined
    \usepackage{enumerate} % Needed for markdown enumerations to work
    \usepackage{geometry} % Used to adjust the document margins
    \usepackage{amsmath} % Equations
    \usepackage{amssymb} % Equations
    \usepackage{textcomp} % defines textquotesingle
    % Hack from http,//tex.stackexchange.com/a/47451/13684,
   \AtBeginDocument{%
        \def\PYZsq{\textquotesingle}% Upright quotes in Pygmentized code
    }
    \usepackage{upquote} % Upright quotes for verbatim code
    \usepackage{eurosym} % defines \euro
    %\usepackage[mathletters]{ucs} % Extended unicode (utf-8) support
    \usepackage{fancyvrb} % verbatim replacement that allows latex
    \usepackage{grffile} % extends the file name processing of package graphics 
                         % to support a larger range
    \makeatletter % fix for grffile with XeLaTeX
    \def\Gread@@xetex#1{%
      \IfFileExists{"\Gin@base".bb}%
      {\Gread@eps{\Gin@base.bb}}%
      {\Gread@@xetex@aux#1}%
    }
    \makeatother

    % The hyperref package gives us a pdf with properly built
    % internal navigation ('pdf bookmarks' for the table of contents,
    % internal cross-reference links, web links for URLs, etc.)
    \usepackage{hyperref}
    % The default LaTeX title has an obnoxious amount of whitespace. By default,
    % titling removes some of it. It also provides customization options.
    \usepackage{titling}
    \usepackage{longtable} % longtable support required by pandoc >1.10
    \usepackage{booktabs}  % table support for pandoc > 1.12.2
    \usepackage[inline]{enumitem} % IRkernel/repr support (it uses the enumerate* environment)
    \usepackage[normalem]{ulem} % ulem is needed to support strikethroughs (\sout)
                                % normalem makes italics be italics, not underlines
    \usepackage{mathrsfs}
    \usepackage[spanish]{babel}

    
    % Colors for the hyperref package
    \definecolor{urlcolor}{rgb}{0,.145,.698}
    \definecolor{linkcolor}{rgb}{.71,0.21,0.01}
    \definecolor{citecolor}{rgb}{.12,.54,.11}

    % ANSI colors
    \definecolor{ansi-black}{HTML}{3E424D}
    \definecolor{ansi-black-intense}{HTML}{282C36}
    \definecolor{ansi-red}{HTML}{E75C58}
    \definecolor{ansi-red-intense}{HTML}{B22B31}
    \definecolor{ansi-green}{HTML}{00A250}
    \definecolor{ansi-green-intense}{HTML}{007427}
    \definecolor{ansi-yellow}{HTML}{DDB62B}
    \definecolor{ansi-yellow-intense}{HTML}{B27D12}
    \definecolor{ansi-blue}{HTML}{208FFB}
    \definecolor{ansi-blue-intense}{HTML}{0065CA}
    \definecolor{ansi-magenta}{HTML}{D160C4}
    \definecolor{ansi-magenta-intense}{HTML}{A03196}
    \definecolor{ansi-cyan}{HTML}{60C6C8}
    \definecolor{ansi-cyan-intense}{HTML}{258F8F}
    \definecolor{ansi-white}{HTML}{C5C1B4}
    \definecolor{ansi-white-intense}{HTML}{A1A6B2}
    \definecolor{ansi-default-inverse-fg}{HTML}{FFFFFF}
    \definecolor{ansi-default-inverse-bg}{HTML}{000000}

    % commands and environments needed by pandoc snippets
    % extracted from the output of `pandoc -s`
    \providecommand{\tightlist}{%
      \setlength{\itemsep}{0pt}\setlength{\parskip}{0pt}}
    \DefineVerbatimEnvironment{Highlighting}{Verbatim}{commandchars=\\\{\}}
    % Add ',fontsize=\small' for more characters per line
    \newenvironment{Shaded}{}{}
    \newcommand{\KeywordTok}[1]{\textcolor[rgb]{0.00,0.44,0.13}{\textbf{{#1}}}}
    \newcommand{\DataTypeTok}[1]{\textcolor[rgb]{0.56,0.13,0.00}{{#1}}}
    \newcommand{\DecValTok}[1]{\textcolor[rgb]{0.25,0.63,0.44}{{#1}}}
    \newcommand{\BaseNTok}[1]{\textcolor[rgb]{0.25,0.63,0.44}{{#1}}}
    \newcommand{\FloatTok}[1]{\textcolor[rgb]{0.25,0.63,0.44}{{#1}}}
    \newcommand{\CharTok}[1]{\textcolor[rgb]{0.25,0.44,0.63}{{#1}}}
    \newcommand{\StringTok}[1]{\textcolor[rgb]{0.25,0.44,0.63}{{#1}}}
    \newcommand{\CommentTok}[1]{\textcolor[rgb]{0.38,0.63,0.69}{\textit{{#1}}}}
    \newcommand{\OtherTok}[1]{\textcolor[rgb]{0.00,0.44,0.13}{{#1}}}
    \newcommand{\AlertTok}[1]{\textcolor[rgb]{1.00,0.00,0.00}{\textbf{{#1}}}}
    \newcommand{\FunctionTok}[1]{\textcolor[rgb]{0.02,0.16,0.49}{{#1}}}
    \newcommand{\RegionMarkerTok}[1]{{#1}}
    \newcommand{\ErrorTok}[1]{\textcolor[rgb]{1.00,0.00,0.00}{\textbf{{#1}}}}
    \newcommand{\NormalTok}[1]{{#1}}
    
    % Additional commands for more recent versions of Pandoc
    \newcommand{\ConstantTok}[1]{\textcolor[rgb]{0.53,0.00,0.00}{{#1}}}
    \newcommand{\SpecialCharTok}[1]{\textcolor[rgb]{0.25,0.44,0.63}{{#1}}}
    \newcommand{\VerbatimStringTok}[1]{\textcolor[rgb]{0.25,0.44,0.63}{{#1}}}
    \newcommand{\SpecialStringTok}[1]{\textcolor[rgb]{0.73,0.40,0.53}{{#1}}}
    \newcommand{\ImportTok}[1]{{#1}}
    \newcommand{\DocumentationTok}[1]{\textcolor[rgb]{0.73,0.13,0.13}{\textit{{#1}}}}
    \newcommand{\AnnotationTok}[1]{\textcolor[rgb]{0.38,0.63,0.69}{\textbf{\textit{{#1}}}}}
    \newcommand{\CommentVarTok}[1]{\textcolor[rgb]{0.38,0.63,0.69}{\textbf{\textit{{#1}}}}}
    \newcommand{\VariableTok}[1]{\textcolor[rgb]{0.10,0.09,0.49}{{#1}}}
    \newcommand{\ControlFlowTok}[1]{\textcolor[rgb]{0.00,0.44,0.13}{\textbf{{#1}}}}
    \newcommand{\OperatorTok}[1]{\textcolor[rgb]{0.40,0.40,0.40}{{#1}}}
    \newcommand{\BuiltInTok}[1]{{#1}}
    \newcommand{\ExtensionTok}[1]{{#1}}
    \newcommand{\PreprocessorTok}[1]{\textcolor[rgb]{0.74,0.48,0.00}{{#1}}}
    \newcommand{\AttributeTok}[1]{\textcolor[rgb]{0.49,0.56,0.16}{{#1}}}
    \newcommand{\InformationTok}[1]{\textcolor[rgb]{0.38,0.63,0.69}{\textbf{\textit{{#1}}}}}
    \newcommand{\WarningTok}[1]{\textcolor[rgb]{0.38,0.63,0.69}{\textbf{\textit{{#1}}}}}
    
    
    % Define a nice break command that doesn't care if a line doesn't already
    % exist.
    \def\br{\hspace*{\fill} \\* }
    % Math Jax compatibility definitions
    \def\gt{>}
    \def\lt{<}
    \let\Oldtex\TeX
    \let\Oldlatex\LaTeX
    \renewcommand{\TeX}{\textrm{\Oldtex}}
    \renewcommand{\LaTeX}{\textrm{\Oldlatex}}
    % Document parameters
    % Document title
    \title{Tarea\_3\_IA}
    
    
    
    
    
% Pygments definitions
\makeatletter
\def\PY@reset{\let\PY@it=\relax \let\PY@bf=\relax%
    \let\PY@ul=\relax \let\PY@tc=\relax%
    \let\PY@bc=\relax \let\PY@ff=\relax}
\def\PY@tok#1{\csname PY@tok@#1\endcsname}
\def\PY@toks#1+{\ifx\relax#1\empty\else%
    \PY@tok{#1}\expandafter\PY@toks\fi}
\def\PY@do#1{\PY@bc{\PY@tc{\PY@ul{%
    \PY@it{\PY@bf{\PY@ff{#1}}}}}}}
\def\PY#1#2{\PY@reset\PY@toks#1+\relax+\PY@do{#2}}

\expandafter\def\csname PY@tok@w\endcsname{\def\PY@tc##1{\textcolor[rgb]{0.73,0.73,0.73}{##1}}}
\expandafter\def\csname PY@tok@c\endcsname{\let\PY@it=\textit\def\PY@tc##1{\textcolor[rgb]{0.25,0.50,0.50}{##1}}}
\expandafter\def\csname PY@tok@cp\endcsname{\def\PY@tc##1{\textcolor[rgb]{0.74,0.48,0.00}{##1}}}
\expandafter\def\csname PY@tok@k\endcsname{\let\PY@bf=\textbf\def\PY@tc##1{\textcolor[rgb]{0.00,0.50,0.00}{##1}}}
\expandafter\def\csname PY@tok@kp\endcsname{\def\PY@tc##1{\textcolor[rgb]{0.00,0.50,0.00}{##1}}}
\expandafter\def\csname PY@tok@kt\endcsname{\def\PY@tc##1{\textcolor[rgb]{0.69,0.00,0.25}{##1}}}
\expandafter\def\csname PY@tok@o\endcsname{\def\PY@tc##1{\textcolor[rgb]{0.40,0.40,0.40}{##1}}}
\expandafter\def\csname PY@tok@ow\endcsname{\let\PY@bf=\textbf\def\PY@tc##1{\textcolor[rgb]{0.67,0.13,1.00}{##1}}}
\expandafter\def\csname PY@tok@nb\endcsname{\def\PY@tc##1{\textcolor[rgb]{0.00,0.50,0.00}{##1}}}
\expandafter\def\csname PY@tok@nf\endcsname{\def\PY@tc##1{\textcolor[rgb]{0.00,0.00,1.00}{##1}}}
\expandafter\def\csname PY@tok@nc\endcsname{\let\PY@bf=\textbf\def\PY@tc##1{\textcolor[rgb]{0.00,0.00,1.00}{##1}}}
\expandafter\def\csname PY@tok@nn\endcsname{\let\PY@bf=\textbf\def\PY@tc##1{\textcolor[rgb]{0.00,0.00,1.00}{##1}}}
\expandafter\def\csname PY@tok@ne\endcsname{\let\PY@bf=\textbf\def\PY@tc##1{\textcolor[rgb]{0.82,0.25,0.23}{##1}}}
\expandafter\def\csname PY@tok@nv\endcsname{\def\PY@tc##1{\textcolor[rgb]{0.10,0.09,0.49}{##1}}}
\expandafter\def\csname PY@tok@no\endcsname{\def\PY@tc##1{\textcolor[rgb]{0.53,0.00,0.00}{##1}}}
\expandafter\def\csname PY@tok@nl\endcsname{\def\PY@tc##1{\textcolor[rgb]{0.63,0.63,0.00}{##1}}}
\expandafter\def\csname PY@tok@ni\endcsname{\let\PY@bf=\textbf\def\PY@tc##1{\textcolor[rgb]{0.60,0.60,0.60}{##1}}}
\expandafter\def\csname PY@tok@na\endcsname{\def\PY@tc##1{\textcolor[rgb]{0.49,0.56,0.16}{##1}}}
\expandafter\def\csname PY@tok@nt\endcsname{\let\PY@bf=\textbf\def\PY@tc##1{\textcolor[rgb]{0.00,0.50,0.00}{##1}}}
\expandafter\def\csname PY@tok@nd\endcsname{\def\PY@tc##1{\textcolor[rgb]{0.67,0.13,1.00}{##1}}}
\expandafter\def\csname PY@tok@s\endcsname{\def\PY@tc##1{\textcolor[rgb]{0.73,0.13,0.13}{##1}}}
\expandafter\def\csname PY@tok@sd\endcsname{\let\PY@it=\textit\def\PY@tc##1{\textcolor[rgb]{0.73,0.13,0.13}{##1}}}
\expandafter\def\csname PY@tok@si\endcsname{\let\PY@bf=\textbf\def\PY@tc##1{\textcolor[rgb]{0.73,0.40,0.53}{##1}}}
\expandafter\def\csname PY@tok@se\endcsname{\let\PY@bf=\textbf\def\PY@tc##1{\textcolor[rgb]{0.73,0.40,0.13}{##1}}}
\expandafter\def\csname PY@tok@sr\endcsname{\def\PY@tc##1{\textcolor[rgb]{0.73,0.40,0.53}{##1}}}
\expandafter\def\csname PY@tok@ss\endcsname{\def\PY@tc##1{\textcolor[rgb]{0.10,0.09,0.49}{##1}}}
\expandafter\def\csname PY@tok@sx\endcsname{\def\PY@tc##1{\textcolor[rgb]{0.00,0.50,0.00}{##1}}}
\expandafter\def\csname PY@tok@m\endcsname{\def\PY@tc##1{\textcolor[rgb]{0.40,0.40,0.40}{##1}}}
\expandafter\def\csname PY@tok@gh\endcsname{\let\PY@bf=\textbf\def\PY@tc##1{\textcolor[rgb]{0.00,0.00,0.50}{##1}}}
\expandafter\def\csname PY@tok@gu\endcsname{\let\PY@bf=\textbf\def\PY@tc##1{\textcolor[rgb]{0.50,0.00,0.50}{##1}}}
\expandafter\def\csname PY@tok@gd\endcsname{\def\PY@tc##1{\textcolor[rgb]{0.63,0.00,0.00}{##1}}}
\expandafter\def\csname PY@tok@gi\endcsname{\def\PY@tc##1{\textcolor[rgb]{0.00,0.63,0.00}{##1}}}
\expandafter\def\csname PY@tok@gr\endcsname{\def\PY@tc##1{\textcolor[rgb]{1.00,0.00,0.00}{##1}}}
\expandafter\def\csname PY@tok@ge\endcsname{\let\PY@it=\textit}
\expandafter\def\csname PY@tok@gs\endcsname{\let\PY@bf=\textbf}
\expandafter\def\csname PY@tok@gp\endcsname{\let\PY@bf=\textbf\def\PY@tc##1{\textcolor[rgb]{0.00,0.00,0.50}{##1}}}
\expandafter\def\csname PY@tok@go\endcsname{\def\PY@tc##1{\textcolor[rgb]{0.53,0.53,0.53}{##1}}}
\expandafter\def\csname PY@tok@gt\endcsname{\def\PY@tc##1{\textcolor[rgb]{0.00,0.27,0.87}{##1}}}
\expandafter\def\csname PY@tok@err\endcsname{\def\PY@bc##1{\setlength{\fboxsep}{0pt}\fcolorbox[rgb]{1.00,0.00,0.00}{1,1,1}{\strut ##1}}}
\expandafter\def\csname PY@tok@kc\endcsname{\let\PY@bf=\textbf\def\PY@tc##1{\textcolor[rgb]{0.00,0.50,0.00}{##1}}}
\expandafter\def\csname PY@tok@kd\endcsname{\let\PY@bf=\textbf\def\PY@tc##1{\textcolor[rgb]{0.00,0.50,0.00}{##1}}}
\expandafter\def\csname PY@tok@kn\endcsname{\let\PY@bf=\textbf\def\PY@tc##1{\textcolor[rgb]{0.00,0.50,0.00}{##1}}}
\expandafter\def\csname PY@tok@kr\endcsname{\let\PY@bf=\textbf\def\PY@tc##1{\textcolor[rgb]{0.00,0.50,0.00}{##1}}}
\expandafter\def\csname PY@tok@bp\endcsname{\def\PY@tc##1{\textcolor[rgb]{0.00,0.50,0.00}{##1}}}
\expandafter\def\csname PY@tok@fm\endcsname{\def\PY@tc##1{\textcolor[rgb]{0.00,0.00,1.00}{##1}}}
\expandafter\def\csname PY@tok@vc\endcsname{\def\PY@tc##1{\textcolor[rgb]{0.10,0.09,0.49}{##1}}}
\expandafter\def\csname PY@tok@vg\endcsname{\def\PY@tc##1{\textcolor[rgb]{0.10,0.09,0.49}{##1}}}
\expandafter\def\csname PY@tok@vi\endcsname{\def\PY@tc##1{\textcolor[rgb]{0.10,0.09,0.49}{##1}}}
\expandafter\def\csname PY@tok@vm\endcsname{\def\PY@tc##1{\textcolor[rgb]{0.10,0.09,0.49}{##1}}}
\expandafter\def\csname PY@tok@sa\endcsname{\def\PY@tc##1{\textcolor[rgb]{0.73,0.13,0.13}{##1}}}
\expandafter\def\csname PY@tok@sb\endcsname{\def\PY@tc##1{\textcolor[rgb]{0.73,0.13,0.13}{##1}}}
\expandafter\def\csname PY@tok@sc\endcsname{\def\PY@tc##1{\textcolor[rgb]{0.73,0.13,0.13}{##1}}}
\expandafter\def\csname PY@tok@dl\endcsname{\def\PY@tc##1{\textcolor[rgb]{0.73,0.13,0.13}{##1}}}
\expandafter\def\csname PY@tok@s2\endcsname{\def\PY@tc##1{\textcolor[rgb]{0.73,0.13,0.13}{##1}}}
\expandafter\def\csname PY@tok@sh\endcsname{\def\PY@tc##1{\textcolor[rgb]{0.73,0.13,0.13}{##1}}}
\expandafter\def\csname PY@tok@s1\endcsname{\def\PY@tc##1{\textcolor[rgb]{0.73,0.13,0.13}{##1}}}
\expandafter\def\csname PY@tok@mb\endcsname{\def\PY@tc##1{\textcolor[rgb]{0.40,0.40,0.40}{##1}}}
\expandafter\def\csname PY@tok@mf\endcsname{\def\PY@tc##1{\textcolor[rgb]{0.40,0.40,0.40}{##1}}}
\expandafter\def\csname PY@tok@mh\endcsname{\def\PY@tc##1{\textcolor[rgb]{0.40,0.40,0.40}{##1}}}
\expandafter\def\csname PY@tok@mi\endcsname{\def\PY@tc##1{\textcolor[rgb]{0.40,0.40,0.40}{##1}}}
\expandafter\def\csname PY@tok@il\endcsname{\def\PY@tc##1{\textcolor[rgb]{0.40,0.40,0.40}{##1}}}
\expandafter\def\csname PY@tok@mo\endcsname{\def\PY@tc##1{\textcolor[rgb]{0.40,0.40,0.40}{##1}}}
\expandafter\def\csname PY@tok@ch\endcsname{\let\PY@it=\textit\def\PY@tc##1{\textcolor[rgb]{0.25,0.50,0.50}{##1}}}
\expandafter\def\csname PY@tok@cm\endcsname{\let\PY@it=\textit\def\PY@tc##1{\textcolor[rgb]{0.25,0.50,0.50}{##1}}}
\expandafter\def\csname PY@tok@cpf\endcsname{\let\PY@it=\textit\def\PY@tc##1{\textcolor[rgb]{0.25,0.50,0.50}{##1}}}
\expandafter\def\csname PY@tok@c1\endcsname{\let\PY@it=\textit\def\PY@tc##1{\textcolor[rgb]{0.25,0.50,0.50}{##1}}}
\expandafter\def\csname PY@tok@cs\endcsname{\let\PY@it=\textit\def\PY@tc##1{\textcolor[rgb]{0.25,0.50,0.50}{##1}}}

\def\PYZbs{\char`\\}
\def\PYZus{\char`\_}
\def\PYZob{\char`\{}
\def\PYZcb{\char`\}}
\def\PYZca{\char`\^}
\def\PYZam{\char`\&}
\def\PYZlt{\char`\<}
\def\PYZgt{\char`\>}
\def\PYZsh{\char`\#}
\def\PYZpc{\char`\%}
\def\PYZdl{\char`\$}
\def\PYZhy{\char`\-}
\def\PYZsq{\char`\'}
\def\PYZdq{\char`\"}
\def\PYZti{\char`\~}
% for compatibility with earlier versions
\def\PYZat{@}
\def\PYZlb{[}
\def\PYZrb{]}
\makeatother


    % For linebreaks inside Verbatim environment from package fancyvrb. 
    \makeatletter
        \newbox\Wrappedcontinuationbox 
        \newbox\Wrappedvisiblespacebox 
        \newcommand*\Wrappedvisiblespace {\textcolor{red}{\textvisiblespace}} 
        \newcommand*\Wrappedcontinuationsymbol {\textcolor{red}{\llap{\tiny$\m@th\hookrightarrow$}}} 
        \newcommand*\Wrappedcontinuationindent {3ex } 
        \newcommand*\Wrappedafterbreak {\kern\Wrappedcontinuationindent\copy\Wrappedcontinuationbox} 
        % Take advantage of the already applied Pygments mark-up to insert 
        % potential linebreaks for TeX processing. 
        %        {, <, #, %, $, ' and ", go to next line. 
        %        _, }, ^, &, >, - and ~, stay at end of broken line. 
        % Use of \textquotesingle for straight quote. 
        \newcommand*\Wrappedbreaksatspecials {% 
            \def\PYGZus{\discretionary{\char`\_}{\Wrappedafterbreak}{\char`\_}}% 
            \def\PYGZob{\discretionary{}{\Wrappedafterbreak\char`\{}{\char`\{}}% 
            \def\PYGZcb{\discretionary{\char`\}}{\Wrappedafterbreak}{\char`\}}}% 
            \def\PYGZca{\discretionary{\char`\^}{\Wrappedafterbreak}{\char`\^}}% 
            \def\PYGZam{\discretionary{\char`\&}{\Wrappedafterbreak}{\char`\&}}% 
            \def\PYGZlt{\discretionary{}{\Wrappedafterbreak\char`\<}{\char`\<}}% 
            \def\PYGZgt{\discretionary{\char`\>}{\Wrappedafterbreak}{\char`\>}}% 
            \def\PYGZsh{\discretionary{}{\Wrappedafterbreak\char`\#}{\char`\#}}% 
            \def\PYGZpc{\discretionary{}{\Wrappedafterbreak\char`\%}{\char`\%}}% 
            \def\PYGZdl{\discretionary{}{\Wrappedafterbreak\char`\$}{\char`\$}}% 
            \def\PYGZhy{\discretionary{\char`\-}{\Wrappedafterbreak}{\char`\-}}% 
            \def\PYGZsq{\discretionary{}{\Wrappedafterbreak\textquotesingle}{\textquotesingle}}% 
            \def\PYGZdq{\discretionary{}{\Wrappedafterbreak\char`\"}{\char`\"}}% 
            \def\PYGZti{\discretionary{\char`\~}{\Wrappedafterbreak}{\char`\~}}% 
        } 
        % Some characters . , ; ? ! / are not pygmentized. 
        % This macro makes them "active" and they will insert potential linebreaks 
        \newcommand*\Wrappedbreaksatpunct {% 
            \lccode`\~`\.\lowercase{\def~}{\discretionary{\hbox{\char`\.}}{\Wrappedafterbreak}{\hbox{\char`\.}}}% 
            \lccode`\~`\,\lowercase{\def~}{\discretionary{\hbox{\char`\,}}{\Wrappedafterbreak}{\hbox{\char`\,}}}% 
            \lccode`\~`\;\lowercase{\def~}{\discretionary{\hbox{\char`\;}}{\Wrappedafterbreak}{\hbox{\char`\;}}}% 
            \lccode`\~`\,\lowercase{\def~}{\discretionary{\hbox{\char`\,}}{\Wrappedafterbreak}{\hbox{\char`\,}}}% 
            \lccode`\~`\?\lowercase{\def~}{\discretionary{\hbox{\char`\?}}{\Wrappedafterbreak}{\hbox{\char`\?}}}% 
            \lccode`\~`\!\lowercase{\def~}{\discretionary{\hbox{\char`\!}}{\Wrappedafterbreak}{\hbox{\char`\!}}}% 
            \lccode`\~`\/\lowercase{\def~}{\discretionary{\hbox{\char`\/}}{\Wrappedafterbreak}{\hbox{\char`\/}}}% 
            \catcode`\.\active
            \catcode`\,\active 
            \catcode`\;\active
            \catcode`\,\active
            \catcode`\?\active
            \catcode`\!\active
            \catcode`\/\active 
            \lccode`\~`\~ 	
        }
    \makeatother

    \let\OriginalVerbatim=\Verbatim
    \makeatletter
    \renewcommand{\Verbatim}[1][1]{%
        %\parskip\z@skip
        \sbox\Wrappedcontinuationbox {\Wrappedcontinuationsymbol}%
        \sbox\Wrappedvisiblespacebox {\FV@SetupFont\Wrappedvisiblespace}%
        \def\FancyVerbFormatLine ##1{\hsize\linewidth
            \vtop{\raggedright\hyphenpenalty\z@\exhyphenpenalty\z@
                \doublehyphendemerits\z@\finalhyphendemerits\z@
                \strut ##1\strut}%
        }%
        % If the linebreak is at a space, the latter will be displayed as visible
        % space at end of first line, and a continuation symbol starts next line.
        % Stretch/shrink are however usually zero for typewriter font.
        \def\FV@Space {%
            \nobreak\hskip\z@ plus\fontdimen3\font minus\fontdimen4\font
            \discretionary{\copy\Wrappedvisiblespacebox}{\Wrappedafterbreak}
            {\kern\fontdimen2\font}%
        }%
        
        % Allow breaks at special characters using \PYG... macros.
        \Wrappedbreaksatspecials
        % Breaks at punctuation characters . , ; ? ! and / need catcode=\active 	
        \OriginalVerbatim[#1,codes*=\Wrappedbreaksatpunct]%
    }
    \makeatother

    % Exact colors from NB
    \definecolor{incolor}{HTML}{303F9F}
    \definecolor{outcolor}{HTML}{D84315}
    \definecolor{cellborder}{HTML}{CFCFCF}
    \definecolor{cellbackground}{HTML}{F7F7F7}
    
    % prompt
    \makeatletter
    \newcommand{\boxspacing}{\kern\kvtcb@left@rule\kern\kvtcb@boxsep}
    \makeatother
    \newcommand{\prompt}[4]{
        \ttfamily\llap{{\color{#2}[#3],\hspace{3pt}#4}}\vspace{-\baselineskip}
    }
    

    
    % Prevent overflowing lines due to hard-to-break entities
    \sloppy 
    % Setup hyperref package
    \hypersetup{
      breaklinks=true,  % so long urls are correctly broken across lines
      colorlinks=true,
      urlcolor=urlcolor,
      linkcolor=linkcolor,
      citecolor=citecolor,
      }
    % Slightly bigger margins than the latex defaults
    
    \geometry{verbose,tmargin=1in,bmargin=1in,lmargin=1in,rmargin=1in}
    
    %---------------------------%
    % Bibliografía con BibLaTeX %
    %---------------------------%
    \usepackage[backend=biber,style=apa]{biblatex}
    \usepackage{csquotes}
    % Usamos el formato APA
    \DeclareLanguageMapping{spanish}{spanish-apa}
    \urlstyle{same}
    % Cargamos la fuente con los datos bibliográficos
    \addbibresource{lib.bib}
\begin{document}
    \title{Tarea 3 IA}
    \author{Roberto Vásquez Martínez \\ Profesor: Arturo Hernández Aguirre}
    \date{17/Noviembre/2021}
    \maketitle
    
    \hypertarget{buxfasqueda-informada-con-algoritmo-a}{%
\section{\texorpdfstring{Búsqueda Informada con algoritmo
\(A^*\)}{Búsqueda Informada con algoritmo A\^{}*}}\label{buxfasqueda-informada-con-algoritmo-a}}

    El objetivo en esta tarea será mejor notablemente la complejidad tanto
en tiempo como en memoria la búsqueda ciega implementada en la Tarea 2.
Lo que haremos será resolver el 8-Imposible con el algoritmo \(A^*\). La
función de costo que utilizaremos para el algoritmo \(A^*\) será de la
forma \begin{equation*}
f(n)=g(n)+h'(n).
\end{equation*}

    para todo nodo \(n\). Aquí \(g\) representa la profunidad del nodo \(n\)
y \(h'\) estimará la distancia del nodo \(n\) a la solución, la cual
especificaremos posteriormente.

Denotaremos por \(h_M(n)\) a la distancia Manhattan del nodo \(n\) a la
solución y por \(h_C(n)\) el número de fichas en posición incorrecta del
tablero \(n\) respecto al tablero solución; a \(h_C\) también se le
conoce como \emph{distancia de Hamming}.

    Por lo tanto, lo que haremos será implementar el algoritmo \(A^*\)
considerando primero la función de costo \(f_M(n)=g(n)+h_M(n)\) y luego
la función \(f_C(n)=g(n)+h_C(n)\) para resolver el 8-imposible.

    En este notebook, al igual que en la Tarea 2, importamos las funciones y
clases que nos permitirán obtener los resultados del archivo
\textsf{heuristic\_search.py} que se adjunta junto con un \textsf{README.md}
en un zip, esto con el fin de darle prioridad a las observaciones y
hacer más sucinto este documento.

El archivo \textsf{heuristic\_search.py} y el notebook de Jupyter
\textsf{Tarea\_3\_IA.ipynb} deben estar en el mismo directorio si se desea ejecutar el notebook de Jupyter.

    \hypertarget{heuruxedsticas-admisibles}{%
\subsection{Heurísticas Admisibles}\label{heuruxedsticas-admisibles}}

    Sabemos que \begin{equation}
        \label{admissible}
h_C(n)\leq h_M(n)\leq h^*(n)
\end{equation}

    para todo estado \(n\) y donde \(h^*\) es el verdadero coste para llegar
de \(n\) al objetivo.

    Como \(h_C\) y \(h_M\) con admisibles entonces por el Teorema 4.2.2 de 
\cite{ginsberg}  el algoritmo \(A^*\) con estas heurísticas nunca regresará
una solución subóptima, por lo que el resultado obtenido con el
algoritmo \(A^*\) será la solución óptima, es decir, la más corta de la
configuración inicial a la final.

    Por lo tanto, anticipamos que con la distancia Manhattan y con la
distancia de Hamming obtendremos soluciones con la misma cantidad de
movimientos, pues en ambos casos la solución es óptima.

    \hypertarget{caso-1}{%
\subsubsection{Caso 1}\label{caso-1}}

    En este caso probaremos con la configuración inicial sugerida
\begin{equation*}\begin{pmatrix}3&2&1\\ 6&5&4\\ 8&7&0 \end{pmatrix},\end{equation*}

    mientras que la configuración final es
\begin{equation*}\begin{pmatrix}1&2&3\\ 4&5&6\\ 7&8&0 \end{pmatrix}.\end{equation*}

    Representamos al hueco del tablero como el número \(0\).

    \begin{tcolorbox}[breakable, size=fbox, boxrule=1pt, pad at break*=1mm,colback=cellbackground, colframe=cellborder]
\prompt{In}{incolor}{1}{\boxspacing}
\begin{Verbatim}[commandchars=\\\{\}]
\PY{k+kn}{import} \PY{n+nn}{sys}
\PY{k+kn}{import} \PY{n+nn}{numpy} \PY{k}{as} \PY{n+nn}{np}
\PY{n}{sys}\PY{o}{.}\PY{n}{path}\PY{o}{.}\PY{n}{append}\PY{p}{(}\PY{l+s+s1}{\PYZsq{}}\PY{l+s+s1}{..}\PY{l+s+s1}{\PYZsq{}}\PY{p}{)}
\PY{k+kn}{from} \PY{n+nn}{heuristic\PYZus{}search} \PY{k+kn}{import} \PY{n}{A\PYZus{}star}
\PY{k+kn}{from} \PY{n+nn}{heuristic\PYZus{}search} \PY{k+kn}{import} \PY{n}{Manhattan\PYZus{}dist}
\PY{k+kn}{from} \PY{n+nn}{heuristic\PYZus{}search} \PY{k+kn}{import} \PY{n}{Counting\PYZus{}dist}
\PY{c+c1}{\PYZsh{} Case 1}
\PY{n}{init\PYZus{}value\PYZus{}1}\PY{o}{=}\PY{n}{np}\PY{o}{.}\PY{n}{array}\PY{p}{(}\PY{p}{[}\PY{p}{[}\PY{l+m+mi}{3}\PY{p}{,}\PY{l+m+mi}{2}\PY{p}{,}\PY{l+m+mi}{1}\PY{p}{]}\PY{p}{,}\PY{p}{[}\PY{l+m+mi}{6}\PY{p}{,}\PY{l+m+mi}{5}\PY{p}{,}\PY{l+m+mi}{4}\PY{p}{]}\PY{p}{,}\PY{p}{[}\PY{l+m+mi}{8}\PY{p}{,}\PY{l+m+mi}{7}\PY{p}{,}\PY{l+m+mi}{0}\PY{p}{]}\PY{p}{]}\PY{p}{)}
\PY{n}{final\PYZus{}value\PYZus{}1}\PY{o}{=}\PY{n}{np}\PY{o}{.}\PY{n}{array}\PY{p}{(}\PY{p}{[}\PY{p}{[}\PY{l+m+mi}{1}\PY{p}{,}\PY{l+m+mi}{2}\PY{p}{,}\PY{l+m+mi}{3}\PY{p}{]}\PY{p}{,}\PY{p}{[}\PY{l+m+mi}{4}\PY{p}{,}\PY{l+m+mi}{5}\PY{p}{,}\PY{l+m+mi}{6}\PY{p}{]}\PY{p}{,}\PY{p}{[}\PY{l+m+mi}{7}\PY{p}{,}\PY{l+m+mi}{8}\PY{p}{,}\PY{l+m+mi}{0}\PY{p}{]}\PY{p}{]}\PY{p}{)}
\PY{c+c1}{\PYZsh{} Heuristic search algorithm}
\PY{n}{A\PYZus{}star\PYZus{}1}\PY{o}{=}\PY{n}{A\PYZus{}star}\PY{p}{(}\PY{n}{init\PYZus{}value\PYZus{}1}\PY{p}{,}\PY{n}{final\PYZus{}value\PYZus{}1}\PY{p}{,}\PY{n}{Counting\PYZus{}dist}\PY{p}{)}
\PY{n}{A\PYZus{}star\PYZus{}1}\PY{o}{.}\PY{n}{main}\PY{p}{(}\PY{n}{Counting\PYZus{}dist}\PY{p}{)}
\end{Verbatim}
\end{tcolorbox}

    \begin{Verbatim}[commandchars=\\\{\}]
No se puede alcanzar el nodo final
    \end{Verbatim}

    A partir de lo anterior vemos que a partir de la configuración incial
proporcionada no es alcanzable la configuración inicial. La heurística
empleada fue considerar como base la configuración
\begin{equation*}\begin{pmatrix}1&2&3\\ 4&5&6\\ 7&8&0 \end{pmatrix},\end{equation*}

    esta se puede puede poner en forma de vector
\begin{equation*}(1,2,3,4,5,6,7,8,0)\end{equation*}

    Cualquier configuración la podemos poner en forma de vector como la
anterior. En cualquier configuración del tablero, una inversión entre
las entradas \(i,j\) , con \(i<j\) , del vector asociado a la
configuración sucede cuando el valor de la celda en la posición \(i\) es
mayor que el valor en la posición \(j\).

    Usamos el hecho de que para poder alcanzar el estado final a partir de un
estado inicial la paridad del número de inversiones del estado inicial y del estado final debe ser la misma.

    Para poder alcanzar la solución lo que haremos será intercambiar dos
fichas de lugar en la configuración inicial de forma que la paridad del
número de inversiones cambie y sea la misma que en la configuración
final. Por lo que usaremos como configuración inicial el siguiente
tablero
\begin{equation*}\begin{pmatrix}3&2&1\\ 6&5&4\\ 7&8&0 \end{pmatrix}.\end{equation*}

    \hypertarget{distancia-de-hamming}{%
\paragraph{Distancia de Hamming}\label{distancia-de-hamming}}

    Mostraremos los resultados correspondientes al algoritmo \(A^*\)
considerando la distancia de Hamming. A continuación presentamos el
código que nos brinda la solución con los resultados.

    \begin{tcolorbox}[breakable, size=fbox, boxrule=1pt, pad at break*=1mm,colback=cellbackground, colframe=cellborder]
\prompt{In}{incolor}{2}{\boxspacing}
\begin{Verbatim}[commandchars=\\\{\}]
\PY{c+c1}{\PYZsh{} Case 1 (Hamming distance)}
\PY{n}{init\PYZus{}value\PYZus{}1}\PY{o}{=}\PY{n}{np}\PY{o}{.}\PY{n}{array}\PY{p}{(}\PY{p}{[}\PY{p}{[}\PY{l+m+mi}{3}\PY{p}{,}\PY{l+m+mi}{2}\PY{p}{,}\PY{l+m+mi}{1}\PY{p}{]}\PY{p}{,}\PY{p}{[}\PY{l+m+mi}{6}\PY{p}{,}\PY{l+m+mi}{5}\PY{p}{,}\PY{l+m+mi}{4}\PY{p}{]}\PY{p}{,}\PY{p}{[}\PY{l+m+mi}{7}\PY{p}{,}\PY{l+m+mi}{8}\PY{p}{,}\PY{l+m+mi}{0}\PY{p}{]}\PY{p}{]}\PY{p}{)}
\PY{n}{final\PYZus{}value\PYZus{}1}\PY{o}{=}\PY{n}{np}\PY{o}{.}\PY{n}{array}\PY{p}{(}\PY{p}{[}\PY{p}{[}\PY{l+m+mi}{1}\PY{p}{,}\PY{l+m+mi}{2}\PY{p}{,}\PY{l+m+mi}{3}\PY{p}{]}\PY{p}{,}\PY{p}{[}\PY{l+m+mi}{4}\PY{p}{,}\PY{l+m+mi}{5}\PY{p}{,}\PY{l+m+mi}{6}\PY{p}{]}\PY{p}{,}\PY{p}{[}\PY{l+m+mi}{7}\PY{p}{,}\PY{l+m+mi}{8}\PY{p}{,}\PY{l+m+mi}{0}\PY{p}{]}\PY{p}{]}\PY{p}{)}
\PY{c+c1}{\PYZsh{} Heuristic search algorithm}
\PY{n}{A\PYZus{}star\PYZus{}1}\PY{o}{=}\PY{n}{A\PYZus{}star}\PY{p}{(}\PY{n}{init\PYZus{}value\PYZus{}1}\PY{p}{,}\PY{n}{final\PYZus{}value\PYZus{}1}\PY{p}{,}\PY{n}{Counting\PYZus{}dist}\PY{p}{)}
\PY{n}{A\PYZus{}star\PYZus{}1}\PY{o}{.}\PY{n}{main}\PY{p}{(}\PY{n}{Counting\PYZus{}dist}\PY{p}{)}
\end{Verbatim}
\end{tcolorbox}

    \begin{Verbatim}[commandchars=\\\{\}]
Numero de movimientos de la solución: 24
Numero de nodos visitados: 12117
Numero de nodos por visitar: 6444
Numero de nodos expandidos: 18561
    \end{Verbatim}

    El tiempo de ejecución en este caso fue de 9m36s y el número de nodos expandidos fue 18561. 
    
    Los caminos completos de la configuración inicial a la
configuración final para cada caso se encuentras en el apéndice \ref{salidas}

    \hypertarget{distancia-manhattan}{%
\paragraph{Distancia Manhattan}\label{distancia-manhattan}}

    A continuación mostramos los resultados obtenidos con el algoritmo
\(A^*\) utilizando la distancia Manhattan.

    \begin{tcolorbox}[breakable, size=fbox, boxrule=1pt, pad at break*=1mm,colback=cellbackground, colframe=cellborder]
\prompt{In}{incolor}{3}{\boxspacing}
\begin{Verbatim}[commandchars=\\\{\}]
\PY{c+c1}{\PYZsh{} Case 1 (Manhattan distance)}
\PY{n}{init\PYZus{}value\PYZus{}1}\PY{o}{=}\PY{n}{np}\PY{o}{.}\PY{n}{array}\PY{p}{(}\PY{p}{[}\PY{p}{[}\PY{l+m+mi}{3}\PY{p}{,}\PY{l+m+mi}{2}\PY{p}{,}\PY{l+m+mi}{1}\PY{p}{]}\PY{p}{,}\PY{p}{[}\PY{l+m+mi}{6}\PY{p}{,}\PY{l+m+mi}{5}\PY{p}{,}\PY{l+m+mi}{4}\PY{p}{]}\PY{p}{,}\PY{p}{[}\PY{l+m+mi}{7}\PY{p}{,}\PY{l+m+mi}{8}\PY{p}{,}\PY{l+m+mi}{0}\PY{p}{]}\PY{p}{]}\PY{p}{)}
\PY{n}{final\PYZus{}value\PYZus{}1}\PY{o}{=}\PY{n}{np}\PY{o}{.}\PY{n}{array}\PY{p}{(}\PY{p}{[}\PY{p}{[}\PY{l+m+mi}{1}\PY{p}{,}\PY{l+m+mi}{2}\PY{p}{,}\PY{l+m+mi}{3}\PY{p}{]}\PY{p}{,}\PY{p}{[}\PY{l+m+mi}{4}\PY{p}{,}\PY{l+m+mi}{5}\PY{p}{,}\PY{l+m+mi}{6}\PY{p}{]}\PY{p}{,}\PY{p}{[}\PY{l+m+mi}{7}\PY{p}{,}\PY{l+m+mi}{8}\PY{p}{,}\PY{l+m+mi}{0}\PY{p}{]}\PY{p}{]}\PY{p}{)}
\PY{c+c1}{\PYZsh{} Heuristic search algorithm}
\PY{n}{A\PYZus{}star\PYZus{}1}\PY{o}{=}\PY{n}{A\PYZus{}star}\PY{p}{(}\PY{n}{init\PYZus{}value\PYZus{}1}\PY{p}{,}\PY{n}{final\PYZus{}value\PYZus{}1}\PY{p}{,}\PY{n}{Manhattan\PYZus{}dist}\PY{p}{)}
\PY{n}{A\PYZus{}star\PYZus{}1}\PY{o}{.}\PY{n}{main}\PY{p}{(}\PY{n}{Manhattan\PYZus{}dist}\PY{p}{)}
\end{Verbatim}
\end{tcolorbox}

    \begin{Verbatim}[commandchars=\\\{\}]
Numero de movimientos de la solución: 24
Numero de nodos visitados: 1628
Numero de nodos por visitar: 875
Numero de nodos expandidos: 2503
    \end{Verbatim}

    El tiempo de ejecución en este caso fue de 12.3s. Por \eqref{admissible} sabemos que el
algoritmo \(A^*\) que usa la distancia Manhattan es \emph{más informado}
que el que usa la distancia de Hamming, lo que quiere decir que si un
nodo es expandido con la distancia de Manhattan es necesariamente
expandido por el algoritmo \(A^*\) con la distancia de Hamming.

Por lo tanto, el número de nodos expandidos por el algoritmo \(A^*\) con
la distancia Manhattan es menor o igual que el número de nodos expandidos por el
algoritmo \(A^*\) con la distancia de Hamming, por lo que con la
distancia Manhattan la búsqueda es más eficiente en tiempo y memoria lo
que se puede corroborar con el ejemplo anterior, pues al algoritmo
\(A^*\) con la distancia de Hamming le toma 9m26s y 18561 nodos
expandidos hallar la solución mientras que al algoritmo \(A^*\) con la distancia de
Manhattan le toma 12.3s y 2503 nodos expandidos.

Sin embargo, al ser ambas distancias admisibles, la solución en ambos
casos es óptima, como lo habíamos anticipado, que en este caso es una
solución en 24 movimientos para ambas heurísticas.

    \hypertarget{caso-2}{%
\subsubsection{Caso 2}\label{caso-2}}

    Como lo solicita el problema haremos dos pruebas más. Para este caso
usaremos la configuración inicial
\begin{equation*}\begin{pmatrix}6&2&8\\ 4&0&5\\ 1&7&3 \end{pmatrix},\end{equation*}

    y como configuración final el siguiente tablero
\begin{equation*}\begin{pmatrix}1&2&3\\ 4&0&5\\ 6&7&8 \end{pmatrix}.\end{equation*}

    \hypertarget{distancia-de-hamming}{%
\paragraph{Distancia de Hamming}\label{distancia-de-hamming}}

    Corremos el algoritmo \(A^*\) con la pareja de tableros anterior
considerando la distancia de Hamming.

    \begin{tcolorbox}[breakable, size=fbox, boxrule=1pt, pad at break*=1mm,colback=cellbackground, colframe=cellborder]
\prompt{In}{incolor}{4}{\boxspacing}
\begin{Verbatim}[commandchars=\\\{\}]
\PY{c+c1}{\PYZsh{} Case 2 (Hamming distance)}
\PY{n}{init\PYZus{}value\PYZus{}2}\PY{o}{=}\PY{n}{np}\PY{o}{.}\PY{n}{array}\PY{p}{(}\PY{p}{[}\PY{p}{[}\PY{l+m+mi}{6}\PY{p}{,}\PY{l+m+mi}{2}\PY{p}{,}\PY{l+m+mi}{8}\PY{p}{]}\PY{p}{,}\PY{p}{[}\PY{l+m+mi}{4}\PY{p}{,}\PY{l+m+mi}{0}\PY{p}{,}\PY{l+m+mi}{5}\PY{p}{]}\PY{p}{,}\PY{p}{[}\PY{l+m+mi}{1}\PY{p}{,}\PY{l+m+mi}{7}\PY{p}{,}\PY{l+m+mi}{3}\PY{p}{]}\PY{p}{]}\PY{p}{)}
\PY{n}{final\PYZus{}value\PYZus{}2}\PY{o}{=}\PY{n}{np}\PY{o}{.}\PY{n}{array}\PY{p}{(}\PY{p}{[}\PY{p}{[}\PY{l+m+mi}{1}\PY{p}{,}\PY{l+m+mi}{2}\PY{p}{,}\PY{l+m+mi}{3}\PY{p}{]}\PY{p}{,}\PY{p}{[}\PY{l+m+mi}{4}\PY{p}{,}\PY{l+m+mi}{0}\PY{p}{,}\PY{l+m+mi}{5}\PY{p}{]}\PY{p}{,}\PY{p}{[}\PY{l+m+mi}{6}\PY{p}{,}\PY{l+m+mi}{7}\PY{p}{,}\PY{l+m+mi}{8}\PY{p}{]}\PY{p}{]}\PY{p}{)}
\PY{c+c1}{\PYZsh{} Heuristic search algorithm}
\PY{n}{A\PYZus{}star\PYZus{}2}\PY{o}{=}\PY{n}{A\PYZus{}star}\PY{p}{(}\PY{n}{init\PYZus{}value\PYZus{}2}\PY{p}{,}\PY{n}{final\PYZus{}value\PYZus{}2}\PY{p}{,}\PY{n}{Counting\PYZus{}dist}\PY{p}{)}
\PY{n}{A\PYZus{}star\PYZus{}2}\PY{o}{.}\PY{n}{main}\PY{p}{(}\PY{n}{Counting\PYZus{}dist}\PY{p}{)}
\end{Verbatim}
\end{tcolorbox}

    \begin{Verbatim}[commandchars=\\\{\}]
Numero de movimientos de la solución: 24
Numero de nodos visitados: 18786
Numero de nodos por visitar: 9378
Numero de nodos expandidos: 28164
    \end{Verbatim}

    El tiempo de ejecución con la distancia de Hamming en este caso fue de
22m46s y el número de nodos expandidos fue de 28164. Ahora
contrastaremos estos resultados con los obtenidos por la distancia
Manhattan.

    \hypertarget{distancia-manhattan}{%
\paragraph{Distancia Manhattan}\label{distancia-manhattan}}

    Para el par de estados inicial y final anterior buscamos la solución
utilizando la distancia Manhattan.

    \begin{tcolorbox}[breakable, size=fbox, boxrule=1pt, pad at break*=1mm,colback=cellbackground, colframe=cellborder]
\prompt{In}{incolor}{5}{\boxspacing}
\begin{Verbatim}[commandchars=\\\{\}]
\PY{c+c1}{\PYZsh{} Case 2 (Hamming distance)}
\PY{n}{init\PYZus{}value\PYZus{}2}\PY{o}{=}\PY{n}{np}\PY{o}{.}\PY{n}{array}\PY{p}{(}\PY{p}{[}\PY{p}{[}\PY{l+m+mi}{6}\PY{p}{,}\PY{l+m+mi}{2}\PY{p}{,}\PY{l+m+mi}{8}\PY{p}{]}\PY{p}{,}\PY{p}{[}\PY{l+m+mi}{4}\PY{p}{,}\PY{l+m+mi}{0}\PY{p}{,}\PY{l+m+mi}{5}\PY{p}{]}\PY{p}{,}\PY{p}{[}\PY{l+m+mi}{1}\PY{p}{,}\PY{l+m+mi}{7}\PY{p}{,}\PY{l+m+mi}{3}\PY{p}{]}\PY{p}{]}\PY{p}{)}
\PY{n}{final\PYZus{}value\PYZus{}2}\PY{o}{=}\PY{n}{np}\PY{o}{.}\PY{n}{array}\PY{p}{(}\PY{p}{[}\PY{p}{[}\PY{l+m+mi}{1}\PY{p}{,}\PY{l+m+mi}{2}\PY{p}{,}\PY{l+m+mi}{3}\PY{p}{]}\PY{p}{,}\PY{p}{[}\PY{l+m+mi}{4}\PY{p}{,}\PY{l+m+mi}{0}\PY{p}{,}\PY{l+m+mi}{5}\PY{p}{]}\PY{p}{,}\PY{p}{[}\PY{l+m+mi}{6}\PY{p}{,}\PY{l+m+mi}{7}\PY{p}{,}\PY{l+m+mi}{8}\PY{p}{]}\PY{p}{]}\PY{p}{)}
\PY{c+c1}{\PYZsh{} Heuristic search algorithm}
\PY{n}{A\PYZus{}star\PYZus{}2}\PY{o}{=}\PY{n}{A\PYZus{}star}\PY{p}{(}\PY{n}{init\PYZus{}value\PYZus{}2}\PY{p}{,}\PY{n}{final\PYZus{}value\PYZus{}2}\PY{p}{,}\PY{n}{Manhattan\PYZus{}dist}\PY{p}{)}
\PY{n}{A\PYZus{}star\PYZus{}2}\PY{o}{.}\PY{n}{main}\PY{p}{(}\PY{n}{Manhattan\PYZus{}dist}\PY{p}{)}
\end{Verbatim}
\end{tcolorbox}

    \begin{Verbatim}[commandchars=\\\{\}]
Numero de movimientos de la solución: 24
Numero de nodos visitados: 1548
Numero de nodos por visitar: 894
Numero de nodos expandidos: 2442
    \end{Verbatim}

    Al igual que con la configuración sugerida tenemos un tiempo de 11.9s y
2442 nodos implementados, en este caso la diferencia de eficiencia entre
el algorimo \(A^*\) con la distancia de Hamming y con la distancia
Manhattan es todavía más notable. Estos dos ejemplos nos enseñan el
beneficio de una heurística admisible más informada. De igual forma que
en el caso 1, ambas heurísticas nos dan la solución óptima.

    \hypertarget{caso-3}{%
\subsubsection{Caso 3}\label{caso-3}}

    Finalmente, correremos un último par de tableros. La configuración
inicial será
\begin{equation*}\begin{pmatrix}1&2&3\\ 8&0&4\\ 6&5&7 \end{pmatrix},\end{equation*}

    mientras que la configuración final será
\begin{equation*}\begin{pmatrix}1&2&3\\ 8&0&4\\ 7&6&5 \end{pmatrix}.\end{equation*}

    \hypertarget{distancia-de-hamming}{%
\paragraph{Distancia de Hamming}\label{distancia-de-hamming}}

    Los resultados con la distancia de Hamming son los siguientes

    \begin{tcolorbox}[breakable, size=fbox, boxrule=1pt, pad at break*=1mm,colback=cellbackground, colframe=cellborder]
\prompt{In}{incolor}{9}{\boxspacing}
\begin{Verbatim}[commandchars=\\\{\}]
\PY{c+c1}{\PYZsh{} Case 3 (Hamming distance)}
\PY{n}{init\PYZus{}value\PYZus{}3}\PY{o}{=}\PY{n}{np}\PY{o}{.}\PY{n}{array}\PY{p}{(}\PY{p}{[}\PY{p}{[}\PY{l+m+mi}{1}\PY{p}{,}\PY{l+m+mi}{2}\PY{p}{,}\PY{l+m+mi}{3}\PY{p}{]}\PY{p}{,}\PY{p}{[}\PY{l+m+mi}{8}\PY{p}{,}\PY{l+m+mi}{0}\PY{p}{,}\PY{l+m+mi}{4}\PY{p}{]}\PY{p}{,}\PY{p}{[}\PY{l+m+mi}{6}\PY{p}{,}\PY{l+m+mi}{5}\PY{p}{,}\PY{l+m+mi}{7}\PY{p}{]}\PY{p}{]}\PY{p}{)}
\PY{n}{final\PYZus{}value\PYZus{}3}\PY{o}{=}\PY{n}{np}\PY{o}{.}\PY{n}{array}\PY{p}{(}\PY{p}{[}\PY{p}{[}\PY{l+m+mi}{1}\PY{p}{,}\PY{l+m+mi}{2}\PY{p}{,}\PY{l+m+mi}{3}\PY{p}{]}\PY{p}{,}\PY{p}{[}\PY{l+m+mi}{8}\PY{p}{,}\PY{l+m+mi}{0}\PY{p}{,}\PY{l+m+mi}{4}\PY{p}{]}\PY{p}{,}\PY{p}{[}\PY{l+m+mi}{7}\PY{p}{,}\PY{l+m+mi}{6}\PY{p}{,}\PY{l+m+mi}{5}\PY{p}{]}\PY{p}{]}\PY{p}{)}
\PY{c+c1}{\PYZsh{} Heuristic search algorithm}
\PY{n}{A\PYZus{}star\PYZus{}3}\PY{o}{=}\PY{n}{A\PYZus{}star}\PY{p}{(}\PY{n}{init\PYZus{}value\PYZus{}3}\PY{p}{,}\PY{n}{final\PYZus{}value\PYZus{}3}\PY{p}{,}\PY{n}{Counting\PYZus{}dist}\PY{p}{)}
\PY{n}{A\PYZus{}star\PYZus{}3}\PY{o}{.}\PY{n}{main}\PY{p}{(}\PY{n}{Counting\PYZus{}dist}\PY{p}{)}
\end{Verbatim}
\end{tcolorbox}

    \begin{Verbatim}[commandchars=\\\{\}]
Numero de movimientos de la solución: 16
Numero de nodos visitados: 525
Numero de nodos por visitar: 362
Numero de nodos expandidos: 887
    \end{Verbatim}

    La idea con esta configuración fue empezar con una configuración con
distancia de Hamming menor que en los casos 1 y 2. Mientras que en los
casos 1 y 2 la distancia de Hamming es de 4, en este caso es de 3, lo que verificamos en el siguiente código.

    \begin{tcolorbox}[breakable, size=fbox, boxrule=1pt, pad at break*=1mm,colback=cellbackground, colframe=cellborder]
\prompt{In}{incolor}{10}{\boxspacing}
\begin{Verbatim}[commandchars=\\\{\}]
\PY{n+nb}{print}\PY{p}{(}\PY{l+s+s1}{\PYZsq{}}\PY{l+s+s1}{Distancia de Hamming Caso 1, }\PY{l+s+si}{\PYZpc{}d}\PY{l+s+s1}{ }\PY{l+s+s1}{\PYZsq{}} \PY{o}{\PYZpc{}}\PY{p}{(}\PY{n}{Counting\PYZus{}dist}\PY{p}{(}\PY{n}{init\PYZus{}value\PYZus{}1}\PY{p}{,}\PY{n}{final\PYZus{}value\PYZus{}1}\PY{p}{)}\PY{p}{)}\PY{p}{)}
\PY{n+nb}{print}\PY{p}{(}\PY{l+s+s1}{\PYZsq{}}\PY{l+s+s1}{Distancia de Hamming Caso 2, }\PY{l+s+si}{\PYZpc{}d}\PY{l+s+s1}{ }\PY{l+s+s1}{\PYZsq{}} \PY{o}{\PYZpc{}}\PY{p}{(}\PY{n}{Counting\PYZus{}dist}\PY{p}{(}\PY{n}{init\PYZus{}value\PYZus{}2}\PY{p}{,}\PY{n}{final\PYZus{}value\PYZus{}2}\PY{p}{)}\PY{p}{)}\PY{p}{)}
\PY{n+nb}{print}\PY{p}{(}\PY{l+s+s1}{\PYZsq{}}\PY{l+s+s1}{Distancia de Hamming Caso 3, }\PY{l+s+si}{\PYZpc{}d}\PY{l+s+s1}{ }\PY{l+s+s1}{\PYZsq{}} \PY{o}{\PYZpc{}}\PY{p}{(}\PY{n}{Counting\PYZus{}dist}\PY{p}{(}\PY{n}{init\PYZus{}value\PYZus{}3}\PY{p}{,}\PY{n}{final\PYZus{}value\PYZus{}3}\PY{p}{)}\PY{p}{)}\PY{p}{)}
\end{Verbatim}
\end{tcolorbox}

    \begin{Verbatim}[commandchars=\\\{\}]
Distancia de Hamming Caso 1: 4
Distancia de Hamming Caso 2: 4
Distancia de Hamming Caso 3: 3
    \end{Verbatim}

    El resultado que obtuvimos fue que en el Caso 3 el algoritmo \(A^*\) con
la distancia de Hamming tiene un mejor desempeño obtienendo la solución
óptima de 16 movimientos en 1.5s y expandiendo 887 nodos.

    \hypertarget{distancia-manhattan}{%
\paragraph{Distancia Manhattan}\label{distancia-manhattan}}

    A continuación analizamos el desempeño del algoritmo \(A^*\) con la
distancia Manhattan con la configuración del Caso 3, esperamos que sea
muy rápida dado el resultado con la distancia de Hamming.

    \begin{tcolorbox}[breakable, size=fbox, boxrule=1pt, pad at break*=1mm,colback=cellbackground, colframe=cellborder]
\prompt{In}{incolor}{11}{\boxspacing}
\begin{Verbatim}[commandchars=\\\{\}]
\PY{c+c1}{\PYZsh{} Case 3 (Hamming distance)}
\PY{n}{init\PYZus{}value\PYZus{}3}\PY{o}{=}\PY{n}{np}\PY{o}{.}\PY{n}{array}\PY{p}{(}\PY{p}{[}\PY{p}{[}\PY{l+m+mi}{1}\PY{p}{,}\PY{l+m+mi}{2}\PY{p}{,}\PY{l+m+mi}{3}\PY{p}{]}\PY{p}{,}\PY{p}{[}\PY{l+m+mi}{8}\PY{p}{,}\PY{l+m+mi}{0}\PY{p}{,}\PY{l+m+mi}{4}\PY{p}{]}\PY{p}{,}\PY{p}{[}\PY{l+m+mi}{6}\PY{p}{,}\PY{l+m+mi}{5}\PY{p}{,}\PY{l+m+mi}{7}\PY{p}{]}\PY{p}{]}\PY{p}{)}
\PY{n}{final\PYZus{}value\PYZus{}3}\PY{o}{=}\PY{n}{np}\PY{o}{.}\PY{n}{array}\PY{p}{(}\PY{p}{[}\PY{p}{[}\PY{l+m+mi}{1}\PY{p}{,}\PY{l+m+mi}{2}\PY{p}{,}\PY{l+m+mi}{3}\PY{p}{]}\PY{p}{,}\PY{p}{[}\PY{l+m+mi}{8}\PY{p}{,}\PY{l+m+mi}{0}\PY{p}{,}\PY{l+m+mi}{4}\PY{p}{]}\PY{p}{,}\PY{p}{[}\PY{l+m+mi}{7}\PY{p}{,}\PY{l+m+mi}{6}\PY{p}{,}\PY{l+m+mi}{5}\PY{p}{]}\PY{p}{]}\PY{p}{)}
\PY{c+c1}{\PYZsh{} Heuristic search algorithm}
\PY{n}{A\PYZus{}star\PYZus{}3}\PY{o}{=}\PY{n}{A\PYZus{}star}\PY{p}{(}\PY{n}{init\PYZus{}value\PYZus{}3}\PY{p}{,}\PY{n}{final\PYZus{}value\PYZus{}3}\PY{p}{,}\PY{n}{Manhattan\PYZus{}dist}\PY{p}{)}
\PY{n}{A\PYZus{}star\PYZus{}3}\PY{o}{.}\PY{n}{main}\PY{p}{(}\PY{n}{Manhattan\PYZus{}dist}\PY{p}{)}
\end{Verbatim}
\end{tcolorbox}

    \begin{Verbatim}[commandchars=\\\{\}]
Numero de movimientos de la solución: 16
Numero de nodos visitados: 165
Numero de nodos por visitar: 111
Numero de nodos expandidos: 276
    \end{Verbatim}

    En este caso, le tomó 0.2s llegar a la solución y 276 nodos expandidos.

    Observamos que cuanto más cerca estemos de \(h^*\) mayor será la tendencia de explorar
en profundidad y viajar directamente por el camino óptimo. Esta es la
razón por la que el número de nodos expandidos se ve reducido
drásticamente usando \(h_M\) que usando \(h_C\), pues con \(h_M\)
estamos más cerca de la búsqueda en profundidad que converge directo a
la solución, que como vimos en la Tarea 2 es más eficiente en
memoria, en consecuencia  al expandir menos nodos se reduce el tiempo de cómputo.

    Sin embargo una limitación de la distancia Manhattan es que considera a
cada ficha independiente de las demás, mientras que en realidad una
ficha puede interferir en el camino de las demás. Hay una mejora a la
distancia Manhattan llamada \emph{Conflicto Lineal} \parencite[pp. 423]{rohit}, que mejoraría aún más la eficiencia del algoritmo \(A^*\).

    \hypertarget{heuruxedstica-no-admisible}{%
\subsection{Heurística no admisible}\label{heuruxedstica-no-admisible}}

    Para el tablero \(n\) consideramos \(P_n\) el número de inversiones del tablero; las inversiones las hemos
definido antes. La heurística que proponemos,
considerando a \(f\) como el tablero final, es \begin{equation*}
h_P(n)=|P_n-P_f|.
\end{equation*}

    Notamos que para el tablero
\begin{equation*}n=\begin{pmatrix}1&2&3\\ 4&5&0\\ 7&8&6 \end{pmatrix},\end{equation*}

    y tablero final
\begin{equation*}f=\begin{pmatrix}1&2&3\\ 4&5&6\\ 7&8&0 \end{pmatrix},\end{equation*}

    se tiene que \(h_P(n)\)=2, lo que podemos verificar con el siguiente
código

    \begin{tcolorbox}[breakable, size=fbox, boxrule=1pt, pad at break*=1mm,colback=cellbackground, colframe=cellborder]
\prompt{In}{incolor}{13}{\boxspacing}
\begin{Verbatim}[commandchars=\\\{\}]
\PY{k+kn}{from} \PY{n+nn}{heuristic\PYZus{}search} \PY{k+kn}{import} \PY{n}{Inversion\PYZus{}dist}
\PY{n}{n}\PY{o}{=}\PY{n}{np}\PY{o}{.}\PY{n}{array}\PY{p}{(}\PY{p}{[}\PY{p}{[}\PY{l+m+mi}{1}\PY{p}{,}\PY{l+m+mi}{2}\PY{p}{,}\PY{l+m+mi}{3}\PY{p}{]}\PY{p}{,}\PY{p}{[}\PY{l+m+mi}{4}\PY{p}{,}\PY{l+m+mi}{5}\PY{p}{,}\PY{l+m+mi}{0}\PY{p}{]}\PY{p}{,}\PY{p}{[}\PY{l+m+mi}{7}\PY{p}{,}\PY{l+m+mi}{8}\PY{p}{,}\PY{l+m+mi}{6}\PY{p}{]}\PY{p}{]}\PY{p}{)}
\PY{n}{f}\PY{o}{=}\PY{n}{np}\PY{o}{.}\PY{n}{array}\PY{p}{(}\PY{p}{[}\PY{p}{[}\PY{l+m+mi}{1}\PY{p}{,}\PY{l+m+mi}{2}\PY{p}{,}\PY{l+m+mi}{3}\PY{p}{]}\PY{p}{,}\PY{p}{[}\PY{l+m+mi}{4}\PY{p}{,}\PY{l+m+mi}{5}\PY{p}{,}\PY{l+m+mi}{6}\PY{p}{]}\PY{p}{,}\PY{p}{[}\PY{l+m+mi}{7}\PY{p}{,}\PY{l+m+mi}{8}\PY{p}{,}\PY{l+m+mi}{0}\PY{p}{]}\PY{p}{]}\PY{p}{)}
\PY{n+nb}{print}\PY{p}{(}\PY{l+s+s1}{\PYZsq{}}\PY{l+s+s1}{h\PYZus{}P(n)= }\PY{l+s+si}{\PYZpc{}d}\PY{l+s+s1}{\PYZsq{}} \PY{o}{\PYZpc{}}\PY{p}{(}\PY{n}{Inversion\PYZus{}dist}\PY{p}{(}\PY{n}{n}\PY{p}{,}\PY{n}{f}\PY{p}{)}\PY{p}{)}\PY{p}{)}
\end{Verbatim}
\end{tcolorbox}

    \begin{Verbatim}[commandchars=\\\{\}]
h\_P(n)= 2
    \end{Verbatim}

    No es díficil ver que podemos obtener \(f\) a partir de \(n\) con un
sólo movimiento, desplazando el \(6\) al espacio vacío en \(n\), luego
\(h^*(n)=1\) por lo que en este caso \(h_P(n)>h^*(n)\).

Por lo tanto \(h_P\) es una heurística no admisible.

    A continuación probaremos cada par de configuraciones en los Casos 1, 2
y 3 con esta heurística \(h_P\) para analizar el efecto que tiene una
heurística no admisible en el algoritmo \(A^*\)

    \hypertarget{caso-1}{%
\subsubsection{Caso 1}\label{caso-1}}

    A continuación presentamos el desempeño del algoritmo \(A^*\) con la
heurística \(h_P\).

    \begin{tcolorbox}[breakable, size=fbox, boxrule=1pt, pad at break*=1mm,colback=cellbackground, colframe=cellborder]
\prompt{In}{incolor}{14}{\boxspacing}
\begin{Verbatim}[commandchars=\\\{\}]
\PY{c+c1}{\PYZsh{} Case 1 (Inversion distance)}
\PY{n}{init\PYZus{}value\PYZus{}1}\PY{o}{=}\PY{n}{np}\PY{o}{.}\PY{n}{array}\PY{p}{(}\PY{p}{[}\PY{p}{[}\PY{l+m+mi}{3}\PY{p}{,}\PY{l+m+mi}{2}\PY{p}{,}\PY{l+m+mi}{1}\PY{p}{]}\PY{p}{,}\PY{p}{[}\PY{l+m+mi}{6}\PY{p}{,}\PY{l+m+mi}{5}\PY{p}{,}\PY{l+m+mi}{4}\PY{p}{]}\PY{p}{,}\PY{p}{[}\PY{l+m+mi}{7}\PY{p}{,}\PY{l+m+mi}{8}\PY{p}{,}\PY{l+m+mi}{0}\PY{p}{]}\PY{p}{]}\PY{p}{)}
\PY{n}{final\PYZus{}value\PYZus{}1}\PY{o}{=}\PY{n}{np}\PY{o}{.}\PY{n}{array}\PY{p}{(}\PY{p}{[}\PY{p}{[}\PY{l+m+mi}{1}\PY{p}{,}\PY{l+m+mi}{2}\PY{p}{,}\PY{l+m+mi}{3}\PY{p}{]}\PY{p}{,}\PY{p}{[}\PY{l+m+mi}{4}\PY{p}{,}\PY{l+m+mi}{5}\PY{p}{,}\PY{l+m+mi}{6}\PY{p}{]}\PY{p}{,}\PY{p}{[}\PY{l+m+mi}{7}\PY{p}{,}\PY{l+m+mi}{8}\PY{p}{,}\PY{l+m+mi}{0}\PY{p}{]}\PY{p}{]}\PY{p}{)}
\PY{c+c1}{\PYZsh{} Heuristic search algorithm}
\PY{n}{A\PYZus{}star\PYZus{}1}\PY{o}{=}\PY{n}{A\PYZus{}star}\PY{p}{(}\PY{n}{init\PYZus{}value\PYZus{}1}\PY{p}{,}\PY{n}{final\PYZus{}value\PYZus{}1}\PY{p}{,}\PY{n}{Inversion\PYZus{}dist}\PY{p}{)}
\PY{n}{A\PYZus{}star\PYZus{}1}\PY{o}{.}\PY{n}{main}\PY{p}{(}\PY{n}{Inversion\PYZus{}dist}\PY{p}{)}
\end{Verbatim}
\end{tcolorbox}

    \begin{Verbatim}[commandchars=\\\{\}]

Numero de movimientos de la solución: 24
Numero de nodos visitados: 4547
Numero de nodos por visitar: 2428
Numero de nodos expandidos: 6975
    \end{Verbatim}

    El algoritmo \(A^*\) en este caso le tomo 1m30s obtener la solución y
expandío 6975 nodos. Como hemos visto \(h_P\) en ocasiones sobreestima
el costo real \(h^*\) lo que puede provocar que que el algoritmo \(A^*\)
con \(h_P\) sea más rápido, en este caso fue más rápido que considerando
la heurística \(h_C\) pero al menos aquí no fue mejor que el algoritmo
\(A^*\) con \(h_M\).

    \hypertarget{caso-2}{%
\subsubsection{Caso 2}\label{caso-2}}

    Ahora probaremos el desempeño del algoritmo \(A^*\) con \(h_P\) y las
configuraciones del caso 2.

    \begin{tcolorbox}[breakable, size=fbox, boxrule=1pt, pad at break*=1mm,colback=cellbackground, colframe=cellborder]
\prompt{In}{incolor}{15}{\boxspacing}
\begin{Verbatim}[commandchars=\\\{\}]
\PY{c+c1}{\PYZsh{} Case 2 (Hamming distance)}
\PY{n}{init\PYZus{}value\PYZus{}2}\PY{o}{=}\PY{n}{np}\PY{o}{.}\PY{n}{array}\PY{p}{(}\PY{p}{[}\PY{p}{[}\PY{l+m+mi}{6}\PY{p}{,}\PY{l+m+mi}{2}\PY{p}{,}\PY{l+m+mi}{8}\PY{p}{]}\PY{p}{,}\PY{p}{[}\PY{l+m+mi}{4}\PY{p}{,}\PY{l+m+mi}{0}\PY{p}{,}\PY{l+m+mi}{5}\PY{p}{]}\PY{p}{,}\PY{p}{[}\PY{l+m+mi}{1}\PY{p}{,}\PY{l+m+mi}{7}\PY{p}{,}\PY{l+m+mi}{3}\PY{p}{]}\PY{p}{]}\PY{p}{)}
\PY{n}{final\PYZus{}value\PYZus{}2}\PY{o}{=}\PY{n}{np}\PY{o}{.}\PY{n}{array}\PY{p}{(}\PY{p}{[}\PY{p}{[}\PY{l+m+mi}{1}\PY{p}{,}\PY{l+m+mi}{2}\PY{p}{,}\PY{l+m+mi}{3}\PY{p}{]}\PY{p}{,}\PY{p}{[}\PY{l+m+mi}{4}\PY{p}{,}\PY{l+m+mi}{0}\PY{p}{,}\PY{l+m+mi}{5}\PY{p}{]}\PY{p}{,}\PY{p}{[}\PY{l+m+mi}{6}\PY{p}{,}\PY{l+m+mi}{7}\PY{p}{,}\PY{l+m+mi}{8}\PY{p}{]}\PY{p}{]}\PY{p}{)}
\PY{c+c1}{\PYZsh{} Heuristic search algorithm}
\PY{n}{A\PYZus{}star\PYZus{}2}\PY{o}{=}\PY{n}{A\PYZus{}star}\PY{p}{(}\PY{n}{init\PYZus{}value\PYZus{}2}\PY{p}{,}\PY{n}{final\PYZus{}value\PYZus{}2}\PY{p}{,}\PY{n}{Inversion\PYZus{}dist}\PY{p}{)}
\PY{n}{A\PYZus{}star\PYZus{}2}\PY{o}{.}\PY{n}{main}\PY{p}{(}\PY{n}{Inversion\PYZus{}dist}\PY{p}{)}
\end{Verbatim}
\end{tcolorbox}

    \begin{Verbatim}[commandchars=\\\{\}]
Numero de movimientos de la solución: 24
Numero de nodos visitados: 1219
Numero de nodos por visitar: 786
Numero de nodos expandidos: 2005
    \end{Verbatim}

    En este caso le tomó al algoritmo \(A^*\) con la heurística \(h_P\) 8.1s
completar el cómputo y necesitó expandir 2005 nodos. Este comportamiento
fue incluso un poco mejor que usando la distancia Manhattan donde se
ocupó 11.9s en el cómputo y se expandieron 2442 nodos.

    \hypertarget{caso-3}{%
\subsubsection{Caso 3}\label{caso-3}}

    Finalmente analizaremos el comportamiento del algoritmo \(A^*\) con la
heurística \(h_P\) y las configuraciones del caso 3.

    \begin{tcolorbox}[breakable, size=fbox, boxrule=1pt, pad at break*=1mm,colback=cellbackground, colframe=cellborder]
\prompt{In}{incolor}{16}{\boxspacing}
\begin{Verbatim}[commandchars=\\\{\}]
\PY{c+c1}{\PYZsh{} Case 3 (Hamming distance)}
\PY{n}{init\PYZus{}value\PYZus{}3}\PY{o}{=}\PY{n}{np}\PY{o}{.}\PY{n}{array}\PY{p}{(}\PY{p}{[}\PY{p}{[}\PY{l+m+mi}{1}\PY{p}{,}\PY{l+m+mi}{2}\PY{p}{,}\PY{l+m+mi}{3}\PY{p}{]}\PY{p}{,}\PY{p}{[}\PY{l+m+mi}{8}\PY{p}{,}\PY{l+m+mi}{0}\PY{p}{,}\PY{l+m+mi}{4}\PY{p}{]}\PY{p}{,}\PY{p}{[}\PY{l+m+mi}{6}\PY{p}{,}\PY{l+m+mi}{5}\PY{p}{,}\PY{l+m+mi}{7}\PY{p}{]}\PY{p}{]}\PY{p}{)}
\PY{n}{final\PYZus{}value\PYZus{}3}\PY{o}{=}\PY{n}{np}\PY{o}{.}\PY{n}{array}\PY{p}{(}\PY{p}{[}\PY{p}{[}\PY{l+m+mi}{1}\PY{p}{,}\PY{l+m+mi}{2}\PY{p}{,}\PY{l+m+mi}{3}\PY{p}{]}\PY{p}{,}\PY{p}{[}\PY{l+m+mi}{8}\PY{p}{,}\PY{l+m+mi}{0}\PY{p}{,}\PY{l+m+mi}{4}\PY{p}{]}\PY{p}{,}\PY{p}{[}\PY{l+m+mi}{7}\PY{p}{,}\PY{l+m+mi}{6}\PY{p}{,}\PY{l+m+mi}{5}\PY{p}{]}\PY{p}{]}\PY{p}{)}
\PY{c+c1}{\PYZsh{} Heuristic search algorithm}
\PY{n}{A\PYZus{}star\PYZus{}3}\PY{o}{=}\PY{n}{A\PYZus{}star}\PY{p}{(}\PY{n}{init\PYZus{}value\PYZus{}3}\PY{p}{,}\PY{n}{final\PYZus{}value\PYZus{}3}\PY{p}{,}\PY{n}{Inversion\PYZus{}dist}\PY{p}{)}
\PY{n}{A\PYZus{}star\PYZus{}3}\PY{o}{.}\PY{n}{main}\PY{p}{(}\PY{n}{Inversion\PYZus{}dist}\PY{p}{)}
\end{Verbatim}
\end{tcolorbox}

    \begin{Verbatim}[commandchars=\\\{\}]
Numero de movimientos de la solución: 16
Numero de nodos visitados: 2531
Numero de nodos por visitar: 1643
Numero de nodos expandidos: 4174
    \end{Verbatim}

    En este caso el cómputo tomó 32.6s y se expendieron 4174 nodos, un
compartamiento peor que la distancia de Hamming donde el cómputo se
completó en 1.5s y 887 nodos expandidos y peor que la distancia
Manhattan que ocupó de 0.2s y 276 nodos expandidos.

Podemos concluir que en promedio el algoritmo \(A^*\) con la heurística
\(h_P\) se comparta mejor que con la heurística \(h_C\), pero es peor
que con la heurística \(h_M\). Esta proposición se basa sólo en los casos analizados. 

Cabe destacar que a pesar de que $h_P$ no es admisible en cada uno de los casos se encontró una solución óptima, pero esto no está garantizado para todo par de configuraciones, es decir, el algoritmo $A^*$ con la heurística $h_P$ puede regresar soluciones subóptimas pues $h_P$ no es admisible.

\clearpage

\appendix

\section{Salidas}
\label{salidas}

Para no interrumpir la lectura del reporte con la sucesión de movimientos que lleva a la configuración final se concentran las soluciones en este apéndice.

En este apéndice se considera a los tableros en forma de vector como se ha dicho antes y el orden de los movimientos es de izquierda a derecha en las filas y moviéndose de arriba a abajo entre filas.
\subsection{Solución Caso 1 Distancia de Hamming}
\setlength{\extrarowheight}{20pt}
\begin{tabular}{c c c c}
(3 , 2 , 1,
6 , 5 , 4,
7 , 8 , 0)

&

(3 , 2 , 1,
6 , 5 , 4,
7 , 0 , 8)

&

(3 , 2 , 1,
6 , 0 , 4,
7 , 5 , 8)

&

(3 , 2 , 1,
0 , 6 , 4,
7 , 5 , 8)

\\

(0 , 2 , 1,
3 , 6 , 4,
7 , 5 , 8)

&

(2 , 0 , 1,
3 , 6 , 4,
7 , 5 , 8)

&

(2 , 6 , 1,
3 , 0 , 4,
7 , 5 , 8)

&

(2 , 6 , 1,
3 , 4 , 0,
7 , 5 , 8)

\\

(2 , 6 , 0,
3 , 4 , 1,
7 , 5 , 8)

&

(2 , 0 , 6,
3 , 4 , 1,
7 , 5 , 8)

&

(2 , 4 , 6,
3 , 0 , 1,
7 , 5 , 8)

&

(2 , 4 , 6,
0 , 3 , 1,
7 , 5 , 8)

\\

(0 , 4 , 6,
2 , 3 , 1,
7 , 5 , 8)

&

(4 , 0 , 6,
2 , 3 , 1,
7 , 5 , 8)

&

(4 , 3 , 6,
2 , 0 , 1,
7 , 5 , 8)

&

(4 , 3 , 6,
2 , 1 , 0,
7 , 5 , 8)

\\

(4 , 3 , 0,
2 , 1 , 6,
7 , 5 , 8)

&

(4 , 0 , 3,
2 , 1 , 6,
7 , 5 , 8)

&

(4 , 1 , 3,
2 , 0 , 6,
7 , 5 , 8)

&

(4 , 1 , 3,
0 , 2 , 6,
7 , 5 , 8)

\\

(0 , 1 , 3,
4 , 2 , 6,
7 , 5 , 8)

&

(1 , 0 , 3,
4 , 2 , 6,
7 , 5 , 8)

&

(1 , 2 , 3,
4 , 0 , 6,
7 , 5 , 8)

&

(1 , 2 , 3,
4 , 5 , 6,
7 , 0 , 8)

\\

(1 , 2 , 3,
4 , 5 , 6,
7 , 8 , 0)

&

&

&

\\
\end{tabular}
\clearpage
\subsection{Solución Caso 1 Distancia Manhattan}
\begin{tabular}{c c c c}
(3 , 2 , 1,
6 , 5 , 4,
7 , 8 , 0)

&

(3 , 2 , 1,
6 , 5 , 4,
7 , 0 , 8)

&

(3 , 2 , 1,
6 , 0 , 4,
7 , 5 , 8)

&


(3 , 2 , 1,
0 , 6 , 4,
7 , 5 , 8)

\\


(0 , 2 , 1,
3 , 6 , 4,
7 , 5 , 8)

&


(2 , 0 , 1,
3 , 6 , 4,
7 , 5 , 8)

&


(2 , 6 , 1,
3 , 0 , 4,
7 , 5 , 8)

&

(2 , 6 , 1,
3 , 4 , 0,
7 , 5 , 8)

\\

(2 , 6 , 0,
3 , 4 , 1,
7 , 5 , 8)

&

(2 , 0 , 6,
3 , 4 , 1,
7 , 5 , 8)

&

(2 , 4 , 6,
3 , 0 , 1,
7 , 5 , 8)

&

(2 , 4 , 6,
0 , 3 , 1,
7 , 5 , 8)

\\

(0 , 4 , 6,
2 , 3 , 1,
7 , 5 , 8)

&

(4 , 0 , 6,
2 , 3 , 1,
7 , 5 , 8)

&

(4 , 3 , 6,
2 , 0 , 1,
7 , 5 , 8)

&

(4 , 3 , 6,
2 , 1 , 0,
7 , 5 , 8)

\\

(4 , 3 , 0,
2 , 1 , 6,
7 , 5 , 8)

&

(4 , 0 , 3,
2 , 1 , 6,
7 , 5 , 8)

&

(4 , 1 , 3,
2 , 0 , 6,
7 , 5 , 8)

&

(4 , 1 , 3,
0 , 2 , 6,
7 , 5 , 8)

\\

(0 , 1 , 3,
4 , 2 , 6,
7 , 5 , 8)

&

(1 , 0 , 3,
4 , 2 , 6,
7 , 5 , 8)

&

(1 , 2 , 3,
4 , 0 , 6,
7 , 5 , 8)

&

(1 , 2 , 3,
4 , 5 , 6,
7 , 0 , 8)

\\

(1 , 2 , 3,
4 , 5 , 6,
7 , 8 , 0)

&

&

&

\\
\end{tabular}
\clearpage

\subsection{Solución Caso 2 Distancia de Hamming}
\begin{tabular}{c c c c}
(6 , 2 , 8,
4 , 0 , 5,
1 , 7 , 3)
&

(6 , 2 , 8,
0 , 4 , 5,
1 , 7 , 3)
&

(0 , 2 , 8,
6 , 4 , 5,
1 , 7 , 3)
&

(2 , 0 , 8,
6 , 4 , 5,
1 , 7 , 3)
\\

(2 , 8 , 0,
6 , 4 , 5,
1 , 7 , 3)
&

(2 , 8 , 5,
6 , 4 , 0,
1 , 7 , 3)
&

(2 , 8 , 5,
6 , 4 , 3,
1 , 7 , 0)
&

(2 , 8 , 5,
6 , 4 , 3,
1 , 0 , 7)
\\

(2 , 8 , 5,
6 , 0 , 3,
1 , 4 , 7)
&

(2 , 0 , 5,
6 , 8 , 3,
1 , 4 , 7)
&

(2 , 5 , 0,
6 , 8 , 3,
1 , 4 , 7)
&

(2 , 5 , 3,
6 , 8 , 0,
1 , 4 , 7)
\\

(2 , 5 , 3,
6 , 0 , 8,
1 , 4 , 7)
&

(2 , 5 , 3,
0 , 6 , 8,
1 , 4 , 7)
&

(2 , 5 , 3,
1 , 6 , 8,
0 , 4 , 7)
&

(2 , 5 , 3,
1 , 6 , 8,
4 , 0 , 7)
\\

(2 , 5 , 3,
1 , 0 , 8,
4 , 6 , 7)
&

(2 , 0 , 3,
1 , 5 , 8,
4 , 6 , 7)
&

(0 , 2 , 3,
1 , 5 , 8,
4 , 6 , 7)
&

(1 , 2 , 3,
0 , 5 , 8,
4 , 6 , 7)
\\

(1 , 2 , 3,
4 , 5 , 8,
0 , 6 , 7)

&
(1 , 2 , 3,
4 , 5 , 8,
6 , 0 , 7)
&

(1 , 2 , 3,
4 , 5 , 8,
6 , 7 , 0)
&

(1 , 2 , 3,
4 , 5 , 0,
6 , 7 , 8)
\\

(1 , 2 , 3,
4 , 0 , 5,
6 , 7 , 8)

& 

&

&

\\
\end{tabular}
\clearpage

\subsection{Solución Caso 2 Distancia Manhattan}
\begin{tabular}{c c c c}
(6 , 2 , 8,
4 , 0 , 5,
1 , 7 , 3)
&

(6 , 2 , 8,
0 , 4 , 5,
1 , 7 , 3)
&

(0 , 2 , 8,
6 , 4 , 5,
1 , 7 , 3)
&

(2 , 0 , 8,
6 , 4 , 5,
1 , 7 , 3)
\\

(2 , 8 , 0,
6 , 4 , 5,
1 , 7 , 3)
&

(2 , 8 , 5,
6 , 4 , 0,
1 , 7 , 3)
&

(2 , 8 , 5,
6 , 4 , 3,
1 , 7 , 0)
&

(2 , 8 , 5,
6 , 4 , 3,
1 , 0 , 7)
\\

(2 , 8 , 5,
6 , 0 , 3,
1 , 4 , 7)
&

(2 , 8 , 5,
0 , 6 , 3,
1 , 4 , 7)
&

(2 , 8 , 5,
1 , 6 , 3,
0 , 4 , 7)
&

(2 , 8 , 5,
1 , 6 , 3,
4 , 0 , 7)
\\

(2 , 8 , 5,
1 , 0 , 3,
4 , 6 , 7)
&

(2 , 0 , 5,
1 , 8 , 3,
4 , 6 , 7,)
&

(2 , 5 , 0,
1 , 8 , 3,
4 , 6 , 7)
&

(2 , 5 , 3,
1 , 8 , 0,
4 , 6 , 7)
\\

(2 , 5 , 3,
1 , 0 , 8,
4 , 6 , 7)
&

(2 , 0 , 3,
1 , 5 , 8,
4 , 6 , 7)
&

(0 , 2 , 3,
1 , 5 , 8,
4 , 6 , 7)
&

(1 , 2 , 3,
0 , 5 , 8,
4 , 6 , 7)

\\


(1 , 2 , 3,
4 , 5 , 8,
0 , 6 , 7)

&
(1 , 2 , 3,
4 , 5 , 8,
6 , 0 , 7)

&
(1 , 2 , 3,
4 , 5 , 8,
6 , 7 , 0)

&
(1 , 2 , 3,
4 , 5 , 0,
6 , 7 , 8)

\\
(1 , 2 , 3,
4 , 0 , 5,
6 , 7 , 8)
&

&

&
\\
\end{tabular}
\clearpage

\subsection{Solución Caso 3 Distancia de Hamming}
\begin{tabular}{c c c c}
(1 , 2 , 3,
8 , 0 , 4,
6 , 5 , 7)
&

(1 , 2 , 3,
8 , 5 , 4,
6 , 0 , 7)
&

(1 , 2 , 3,
8 , 5 , 4,
0 , 6 , 7)
&

(1 , 2 , 3,
0 , 5 , 4,
8 , 6 , 7)
\\


(1 , 2 , 3,
5 , 0 , 4,
8 , 6 , 7)
&

(1 , 2 , 3,
5 , 6 , 4,
8 , 0 , 7)
&

(1 , 2 , 3,
5 , 6 , 4,
8 , 7 , 0)
&


(1 , 2 , 3,
5 , 6 , 0,
8 , 7 , 4)
\\


(1 , 2 , 3,
5 , 0 , 6,
8 , 7 , 4)
&

(1 , 2 , 3,
0 , 5 , 6,
8 , 7 , 4)
&


(1 , 2 , 3,
8 , 5 , 6,
0 , 7 , 4)
&

(1 , 2 , 3,
8 , 5 , 6,
7 , 0 , 4)
\\

(1 , 2 , 3,
8 , 0 , 6,
7 , 5 , 4)

&
(1 , 2 , 3,
8 , 6 , 0,
7 , 5 , 4)
&

(1 , 2 , 3,
8 , 6 , 4,
7 , 5 , 0)
&

(1 , 2 , 3,
8 , 6 , 4,
7 , 0 , 5)
\\

(1 , 2 , 3,
8 , 0 , 4,
7 , 6 , 5)
\end{tabular}
\clearpage

\subsection{Solución Caso 3 Distancia Manhattan}
\begin{tabular}{c c c c}
(1 , 2 , 3,
8 , 0 , 4,
6 , 5 , 7)
&

(1 , 2 , 3,
8 , 4 , 0,
6 , 5 , 7)
&

(1 , 2 , 3,
8 , 4 , 7,
6 , 5 , 0)
&

(1 , 2 , 3,
8 , 4 , 7,
6 , 0 , 5)
\\


(1 , 2 , 3,
8 , 4 , 7,
0 , 6 , 5)
&

(1 , 2 , 3,
0 , 4 , 7,
8 , 6 , 5)
&

(1 , 2 , 3,
4 , 0 , 7,
8 , 6 , 5)
&

(1 , 2 , 3,
4 , 7 , 0,
8 , 6 , 5)
\\

(1 , 2 , 3,
4 , 7 , 5,
8 , 6 , 0)
&

(1 , 2 , 3,
4 , 7 , 5,
8 , 0 , 6)
&

(1 , 2 , 3,
4 , 0 , 5,
8 , 7 , 6)
&

(1 , 2 , 3,
0 , 4 , 5,
8 , 7 , 6)
\\

(1 , 2 , 3,
8 , 4 , 5,
0 , 7 , 6)
&

(1 , 2 , 3,
8 , 4 , 5,
7 , 0 , 6)
&

(1 , 2 , 3,
8 , 4 , 5,
7 , 6 , 0)
&

(1 , 2 , 3,
8 , 4 , 0,
7 , 6 , 5)
\\

(1 , 2 , 3,
8 , 0 , 4,
7 , 6 , 5)
&

&

&

\\
\end{tabular}
\clearpage

\subsection{Solución Caso 1 Heurística no admisible}
\begin{tabular}{c c c c}
(3 , 2 , 1,
6 , 5 , 4,
7 , 8 , 0)
&

(3 , 2 , 1,
6 , 5 , 4,
7 , 0 , 8)
&

(3 , 2 , 1,
6 , 0 , 4,
7 , 5 , 8)
&

(3 , 0 , 1,
6 , 2 , 4,
7 , 5 , 8)
\\

(0 , 3 , 1,
6 , 2 , 4,
7 , 5 , 8)
&

(6 , 3 , 1,
0 , 2 , 4,
7 , 5 , 8)
&

(6 , 3 , 1,
2 , 0 , 4,
7 , 5 , 8)
&

(6 , 0 , 1,
2 , 3 , 4,
7 , 5 , 8)
\\

(6 , 1 , 0,
2 , 3 , 4,
7 , 5 , 8)
&

(6 , 1 , 4,
2 , 3 , 0,
7 , 5 , 8)
&

(6 , 1 , 4,
2 , 0 , 3,
7 , 5 , 8)
&

(6 , 0 , 4,
2 , 1 , 3,
7 , 5 , 8)
\\

(0 , 6 , 4,
2 , 1 , 3,
7 , 5 , 8)
&

(2 , 6 , 4,
0 , 1 , 3,
7 , 5 , 8)
&

(2 , 6 , 4,
1 , 0 , 3,
7 , 5 , 8)
&

(2 , 0 , 4,
1 , 6 , 3,
7 , 5 , 8)
\\

(2 , 4 , 0,
1 , 6 , 3,
7 , 5 , 8)
&

(2 , 4 , 3,
1 , 6 , 0,
7 , 5 , 8)
&

(2 , 4 , 3,
1 , 0 , 6,
7 , 5 , 8)
&

(2 , 0 , 3,
1 , 4 , 6,
7 , 5 , 8)
\\

(0 , 2 , 3,
1 , 4 , 6,
7 , 5 , 8)
&

(1 , 2 , 3,
0 , 4 , 6,
7 , 5 , 8)
&

(1 , 2 , 3,
4 , 0 , 6,
7 , 5 , 8)
&

(1 , 2 , 3,
4 , 5 , 6,
7 , 0 , 8)
\\

(1 , 2 , 3,
4 , 5 , 6,
7 , 8 , 0)

&

&

&

\\
\end{tabular}
\clearpage

\subsection{Solución Caso 2 Heurística no admisible}
\begin{tabular}{c c c c}
(6 , 2 , 8,
4 , 0 , 5,
1 , 7 , 3)
&

(6 , 2 , 8,
0 , 4 , 5,
1 , 7 , 3)
&

(6 , 2 , 8,
1 , 4 , 5,
0 , 7 , 3)
&

(6 , 2 , 8,
1 , 4 , 5,
7 , 0 , 3)
\\

(6 , 2 , 8,
1 , 4 , 5,
7 , 3 , 0)
&

(6 , 2 , 8,
1 , 4 , 0,
7 , 3 , 5)
&

(6 , 2 , 0,
1 , 4 , 8,
7 , 3 , 5)
&

(6 , 0 , 2,
1 , 4 , 8,
7 , 3 , 5)
\\

(6 , 4 , 2,
1 , 0 , 8,
7 , 3 , 5)
&

(6 , 4 , 2,
1 , 3 , 8,
7 , 0 , 5)
&

(6 , 4 , 2,
1 , 3 , 8,
7 , 5 , 0)
&

(6 , 4 , 2,
1 , 3 , 0,
7 , 5 , 8)
\\

(6 , 4 , 2,
1 , 0 , 3,
7 , 5 , 8)
&

(6 , 4 , 2,
0 , 1 , 3,
7 , 5 , 8)
&

(0 , 4 , 2,
6 , 1 , 3,
7 , 5 , 8)
&

(4 , 0 , 2,
6 , 1 , 3,
7 , 5 , 8)
\\

(4 , 1 , 2,
6 , 0 , 3,
7 , 5 , 8)
&

(4 , 1 , 2,
6 , 5 , 3,
7 , 0 , 8)
&

(4 , 1 , 2,
6 , 5 , 3,
0 , 7 , 8)
&

(4 , 1 , 2,
0 , 5 , 3,
6 , 7 , 8)
\\

(0 , 1 , 2,
4 , 5 , 3,
6 , 7 , 8)
&

(1 , 0 , 2,
4 , 5 , 3,
6 , 7 , 8)
&

(1 , 2 , 0,
4 , 5 , 3,
6 , 7 , 8)
&

(1 , 2 , 3,
4 , 5 , 0,
6 , 7 , 8)
\\

(1 , 2 , 3,
4 , 0 , 5,
6 , 7 , 8)
&

&

&
\\
\end{tabular}
\clearpage

\subsection{Solución Caso 3 Heurística no admisible}
\begin{tabular}{c c c c}
(1 , 2 , 3,
8 , 0 , 4,
6 , 5 , 7)
&

(1 , 2 , 3,
0 , 8 , 4,
6 , 5 , 7)
&

(1 , 2 , 3,
6 , 8 , 4,
0 , 5 , 7)
&

(1 , 2 , 3,
6 , 8 , 4,
5 , 0 , 7)
\\

(1 , 2 , 3,
6 , 8 , 4,
5 , 7 , 0)
&

(1 , 2 , 3,
6 , 8 , 0,
5 , 7 , 4)
&

(1 , 2 , 3,
6 , 0 , 8,
5 , 7 , 4)
&

(1 , 2 , 3,
6 , 7 , 8,
5 , 0 , 4)
\\


(1 , 2 , 3,
6 , 7 , 8,
0 , 5 , 4)
&

(1 , 2 , 3,
0 , 7 , 8,
6 , 5 , 4)
&

(1 , 2 , 3,
7 , 0 , 8,
6 , 5 , 4)
&

(1 , 2 , 3,
7 , 8 , 0,
6 , 5 , 4)
\\

(1 , 2 , 3,
7 , 8 , 4,
6 , 5 , 0)
&

(1 , 2 , 3,
7 , 8 , 4,
6 , 0 , 5)
&

(1 , 2 , 3,
7 , 8 , 4,
0 , 6 , 5)
&

(1 , 2 , 3,
0 , 8 , 4,
7 , 6 , 5)
\\

(1 , 2 , 3,
8 , 0 , 4,
7 , 6 , 5)

&

&

&

\\
\end{tabular}

    % Add a bibliography block to the postdoc
    \clearpage
    %%%%%%%%%%%%%%%%
    % Bibliografía %
    %%%%%%%%%%%%%%%%
    \nocite{*}
    \printbibliography
    
    
\end{document}
